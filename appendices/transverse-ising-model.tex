\documentclass[../thesis.tex]{subfiles}
\begin{document}

\chapter{The transverse-field Ising model}

We would like to solve the Hamiltonian
\begin{equation}
  \ham
  = -J\sum_{i \in \ZZ_N} \qty(%
  \opr{S}_i^x\opr{S}_{i+1}^x
  + g \opr{S}_i^z
  )
\end{equation}
which we nondimensionalize as
\begin{equation}
  \frac{4}{J}\ham
  = -\sum_{i \in \ZZ_N} \qty(%
  \pauli_i^x\pauli_{i+1}^x + g \pauli_i^z
  )
  \label{eq:transham}
\end{equation}
for the periodic transverse-field Ising chain with $N$ spins. We will drop the
$4/J$ in what follows. We notice that the operators
\begin{equation}
  \pauli_i^\pm
  = \frac{\pauli_i^x \pm \im\pauli_i^y}{2}
  \label{eq:paulipm}
\end{equation}
satisfy
\begin{equation}
  \pauli_i^z
  = 2\pauli_i^+\pauli_i^- - \idopr
\end{equation}
and have commutators
\begin{align}
  \comm{\pauli_i^+}{\pauli_j^-}
  &= \frac{1}{4}\comm{\pauli_i^x + \im\pauli_i^y}{\pauli_j^x - \im\pauli_j^y} \\
  &= \frac{1}{4} \qty(%
  \comm{\pauli_i^x}{\pauli_j^x} + \comm{\pauli_i^y}{\pauli_j^y}
  + \im\comm{\pauli_i^y}{\pauli_j^x}
  - \im\comm{\pauli_i^x}{\pauli_j^y}
  ) \\
  &= \delta_{ij}\pauli_i^z.
\end{align}
Thus their anticommutators are
\begin{align}
  \acomm{\pauli_i^+}{\pauli_j^-}
  &= 2\pauli_i^+\pauli_j^- - \comm{\pauli_i^+}{\pauli_j^-} \\
  &= 2\pauli_i^+\pauli_j^- - \delta_{ij}\pauli_i^z \\
  &= \delta_{ij}\idopr + 2\pauli_i^+\pauli_j^- (1 - \delta_{ij}).
\end{align}
It could be helpful to think of the $\pauli_i^\pm$ as fermion creation and
annihilation operators, but they do not anticommute at different sites.

How might we construct operators that satisfy the fermionic canonical
anticommutation relations (CARs) from the Pauli operators? Suppose we have such
operators $\opr{c}_i$. Given a tuple $\tq{n} = {\qty(n_i)}_{i\in\ZZ_N}$, we have the
corresponding states
\begin{equation}
  \ket{\tq{n}}
  = \prod_{i\in\ZZ_N} {\qty(\opr{c}_i^\dag)}^{n_i} \ket{\tq{0}},
\end{equation}
where $\ket{\tq{0}}$ denotes the vacuum state. It then follows that
\begin{align}
  \opr{c}_i\ket{\tq{n}}
  &= -n_i {(-1)}^{n_{<i}}\ket{\tq{n}_{i \leftarrow 0}}
  \label{eq:car-action} \\
  \opr{c}_i^\dag\ket{\tq{n}}
  &= -(1 - n_i){(-1)}^{n_{<i}}\ket{\tq{n}_{i \leftarrow 1}},
\end{align}
where $\tq{n}_{i \leftarrow m} = \tq{n}$ with $n_i = m$ and $n_{<i} = \sum_{j <
i} n_j$.

Thus the number operator is
\begin{align}
  \opr{c}_i^\dag \opr{c}_i \ket{\tq{n}}
  &= (1-0){(-1)}^{n_{<i}} n_i{(-1)}^{n_{<i}} \ket{\tq{n}_{i \leftarrow 1}} \\
  &= n_i\ket{\tq{n}}.
\end{align}

This leads us to consider
\begin{align}
  \opr{c}_i
  = -\qty(\prod_{j<i} -\pauli_j^z) \pauli_i^-
\end{align}
acting on the states
\begin{equation}
  \ket{\tq{n}}
  = \prod_{i \in \ZZ_N} {\qty(\pauli_i^+)}^{n_i} \ket{\tq{0}},
\end{equation}
where $\ket{\tq{0}} = \ket{\uparrow}^{\tp N}$ is the state with all $z$-spins
up, or all zero qubits. This gives the same result as \cref{eq:car-action}, so
the $\opr{c}_i$ satisfy the CARs. This process of mapping spin-1/2 sites to
non-local fermions is known as the \term{Jordan-Wigner transformation}. We may
then compute that the inverse transformations are
\begin{align}
  \pauli_i^+\pauli_i^-
  &= \opr{c}_i^\dag\opr{c}_i
  \label{eq:paulipm} \\
  \pauli_i^z
  &= 2\opr{c}_i^\dag\opr{c}_i - \idopr
  \label{eq:pauliz} \\
  \pauli_i^x
  &= -\qty(\prod_{j<i} \qty(\idopr - 2\opr{c}_j^\dag\opr{c}_j))
  \qty(\opr{c}_i^\dag + \opr{c}_i)
  \label{eq:paulix} \\
  \pauli_i^y
  &= \im\qty(\prod_{j<i} \qty(\idopr - 2\opr{c}_j^\dag\opr{c}_j))
  \qty(\opr{c}_i^\dag - \opr{c}_i).
  \label{eq:pauliy}
\end{align}
While $\pauli_i^x$ remains complicated, the product $\pauli_i^x\pauli_{i+1}^x$
does not. For $i < N-1$,
\begin{align}
  \pauli_i^x \pauli_{i+1}^x
  &= \qty(\prod_{j<i} \qty(2\opr{c}_j^\dag\opr{c}_j - \idopr))
  \qty(\opr{c}_i^\dag + \opr{c}_i)
  \qty(\prod_{j<i+1} \qty(2\opr{c}_j^\dag\opr{c}_j - \idopr))
  \qty(\opr{c}_{i+1}^\dag + \opr{c}_{i+1}) \\
  &= \qty(\opr{c}_i^\dag + \opr{c}_i)
  \qty(\idopr - 2\opr{c}_i^\dag\opr{c}_i)
  \qty(\opr{c}_{i+1}^\dag + \opr{c}_{i+1}) \\
  &= \qty(\opr{c}_i^\dag - \opr{c}_i) \qty(\opr{c}_{i+1}^\dag + \opr{c}_{i+1}),
  \label{eq:paulixx}
\end{align}
and for $i = N-1$,
\begin{equation}
  \pauli_{N-1}^x \pauli_0^x
  = \qty(\prod_{j < N-1} \qty(2\opr{c}_j^\dag\opr{c}_j - \idopr))
  \qty(\opr{c}_{N-1}^\dag + \opr{c}_{N-1})
  \qty(\opr{c}_0^\dag + \opr{c}_0).
\end{equation}

We may now perform the Jordan-Wigner transformation of \cref{eq:transham} to
obtain
\begin{equation}
  \begin{aligned}
    \ham
    &= \sum_i \qty(\opr{c}_i - \opr{c}_i^\dag) \qty(\opr{c}_{i+1}^\dag +
    \opr{c}_{i+1})
    - g\sum_i 2\opr{c}_i^\dag\opr{c}_i
    + g N \idopr \\
    &- \qty(\idopr - \prod_{j < N-1}\qty(2\opr{c}_j^\dag \opr{c}_j - \idopr))
    \qty(\opr{c}_{N-1} - \opr{c}_{N-1}^\dag) \qty(\opr{c}_0^\dag + \opr{c}_0).
  \end{aligned}
  \label{eq:cham}
\end{equation}
We now Fourier transform with
\begin{subequations}
\begin{align}
  \opr{c}_i
  &= \frac{1}{\sqrt{N}}\sum_k e^{\im ki} \opr{C}_k
  \label{eq:fourierc} \\
  \opr{C}_k
  &= \frac{1}{\sqrt{N}}\sum_i e^{-\im ki} \opr{c}_i
  \label{eq:fourierC} \\
  \shortintertext{and}
  \opr{c}_i^\dag
  &= \frac{1}{\sqrt{N}}\sum_k e^{-\im ki} \opr{C}_k^\dag
  \label{eq:fourierct} \\
  \opr{C}_k^\dag
  &= \frac{1}{\sqrt{N}}\sum_i e^{\im ki} \opr{c}_i^\dag
  \label{eq:fourierCt}.
\end{align}\label{eq:fourier}
\end{subequations}

We now propagate the periodic boundary conditions to the Fourier-transformed
operators.
\begin{align}
  \opr{c}_0
  &= \frac{1}{\sqrt{N}}\sum_k \opr{C}_k \\
  \opr{c}_N
  &= \frac{1}{\sqrt{N}}\sum_k e^{\im kN} \opr{C}_k.
\end{align}
We then must require that
\begin{equation}
  kN
  \equiv 0 \pmod{2\pi}
\end{equation}
\begin{equation}
  k
  = \frac{2\pi n}{N} - \frac{N - [\text{$N$ odd}]}{N}\pi \qc
  n \in \ZZ_N.
  \label{eq:fourierk}
\end{equation}
For $N$ odd, what is $C_{\pi}$?
\begin{align}
  \opr{C}_{\pi}
  &= \frac{1}{\sqrt{N}} \sum_i e^{-\im \pi i} \opr{c}_i.
\end{align}
Since $e^{-\im\pi i} = e^{\im\pi i}$, $\opr{C}_\pi = \opr{C}_{-\pi}$.

We now verify that this operator Fourier transformation is a unitary operation.
That is, it preserves the fermionic CARs.
\begin{proof}
  Consider $N$ fermionic operators $\opr{c}_i$ and a $N \times N$ unitary matrix
  $\mq{U}$. We may change bases with
  \begin{equation}
    \opr{C}_k^\dag
    = \sum_i U_{ik} \opr{c}_i^\dag.
  \end{equation}
  Then
  \begin{align}
    \acomm{\opr{C}_k}{\opr{C}_{k'}^\dag}
    &= \sum_{ij} U_{ik}^*U_{jk'} \acomm{\opr{c}_i}{\opr{c}_j^\dag} \\
    &= \sum_i U_{ik}^*U_{ik'} \\
    &= {\qty(\mq{U}^\dag\mq{U})}_{kk'} \\
    &= \delta_{kk'},
  \end{align}
  and similar for the other fermionic (anti)-commutation relations.

  For the Fourier transform,
  \begin{equation}
    F_{ik}
    = \frac{1}{\sqrt{N}} e^{\im ki}.
  \end{equation}
  We may then confirm that
  \begin{align}
    {\qty(\mq{F}^\dag\mq{F})}_{kk'}
    &= \sum_i \frac{1}{N} e^{\im (k' - k)i} \\
    &= \delta_{kk'}.
  \end{align}
  Thus the Fourier transform is unitary.
\end{proof}

We are now equipped to Fourier transform \cref{eq:cham} as follows. Since
\begin{equation}
  \frac{1}{N}\sum_{i\in\ZZ_N} e^{\im(k' - k) i}
  = \delta_{kk'},
\end{equation}
and also
\begin{align}
  \opr{C}_{-k}
  &= \opr{C}_k^* \\
  &= \frac{1}{\sqrt{N}} \sum_i e^{-\im (-k)i} \opr{c}_i \\
  &= \frac{1}{N}\sum_{ik'} e^{\im (k' + k)i} \opr{C}_{k'},
\end{align}
we have that
\begin{align}
  \sum_i \opr{c}_i^\dag\opr{c}_i
  &= \frac{1}{N}\sum_{ikk'} e^{\im (k' - k)i}
  \opr{C}_k^\dag \opr{C}_{k'} \\
  &= \sum_k \opr{C}_k^\dag \opr{C}_k,
  \\
  \sum_i \qty(\opr{c}_i^\dag\opr{c}_{i+1} + \opr{c}_{i+1}^\dag\opr{c}_i)
  &= \frac{1}{N}\sum_{ikk'} e^{\im (k' - k)i} \qty(%
  e^{\im k'} + e^{-\im k} 
  ) \opr{C}_k^\dag \opr{C}_{k'} \\
  &= \sum_k 2 \cos k\; \opr{C}_k^\dag \opr{C}_k,
  \\
  \sum_i \qty(\opr{c}_{i+1}\opr{c}_i + \opr{c}_i^\dag\opr{c}_{i+1}^\dag)
  &= \frac{1}{N}\sum_{ikk'} \qty(%
  e^{\im (k' + k)i} e^{\im k} \opr{C}_{k}\opr{C}_{k'}
  + e^{-\im (k' + k)i} e^{-\im k'} \opr{C}_{k}^\dag\opr{C}_{k'}^\dag
  ) \\
  &= \sum_k \qty(%
  e^{-\im k} \opr{C}_{-k} \opr{C}_k
  + e^{\im k} \opr{C}_k^\dag \opr{C}_{-k}^\dag
  ).
\end{align}
Thus \cref{eq:cham} is now
\begin{align}
  \ham
  &= -\sum_k 2 \cos k\; \opr{C}_k^\dag \opr{C}_k
  + \sum_k \qty(%
  e^{-\im k} \opr{C}_{-k} \opr{C}_k
  + e^{\im k} \opr{C}_k^\dag \opr{C}_{-k}^\dag
  )
  - \sum_k 2g\opr{C}_k^\dag\opr{C}_k
  + g N \idopr
  \\
  &= -\sum_k \qty(g + \cos k)
  \qty(%
  \opr{C}_k^\dag \opr{C}_k
  + \opr{C}_{-k}^\dag \opr{C}_{-k}
  )
  + \sum_k \im\sin k
  \qty(%
  \opr{C}_{-k}\opr{C}_k
  - \opr{C}_k^\dag\opr{C}_{-k}^\dag
  )
  + g N \idopr
  \\
  &= -\sum_k \qty(g + \cos k)
  \qty(%
  \opr{C}_k^\dag \opr{C}_k
  - \opr{C}_{-k} \opr{C}_{-k}^\dag
  )
  + \sum_k \im\sin k
  \qty(%
  \opr{C}_{-k}\opr{C}_k
  - \opr{C}_k^\dag\opr{C}_{-k}^\dag
  ) \\
  &= \sum_k \vq{v}_k^\dag \mq{H}_k \vq{v}_k,
  \label{eq:Cham} \\
  \shortintertext{where}
  \mq{H}_k
  &= \bmqty{%
  -(g + \cos k) & -\im\sin k \\
  \im\sin k & g + \cos k
  }, \\
  \vq{v}_k
  &= \bmqty{\opr{C}_k \\ \opr{C}_{-k}^\dag},
\end{align}
and we have used that
\begin{equation}
  \sum_k \cos k = 0.
\end{equation}
Since the $\mq{H}_k$ are Hermitian, they may be diagonalized by a unitary
transformation of the $\vq{v}_k$.\footnote{%
  The unitary transformation of the $\opr{C}_{\pm k}$ to obtain
  $\opr{\eta}_k^\pm$ is an instance of a fermionic \term{Bogoliubov
  transformation}:
  \begin{subequations}
    \begin{align}
      \opr{C}_k
    &= u\opr{f}_k + v\opr{g}_k^\dag \\
    \shortintertext{and}
    \opr{C}_{-k}
    &= -v\opr{f}_k^\dag + u\opr{g}_k.
    \end{align}\label{eq:bogoliubov}
  \end{subequations}
  For these transformations to preserve the CARs,
  \begin{align}
    \acomm{\opr{C}_k^\dag}{\opr{C}_k}
    &= \abs{u}^2 \acomm{\opr{f}_k^\dag}{\opr{f}_k}
    + \abs{v}^2 \acomm{\opr{g}_k}{\opr{g}_k^\dag}
    + u^* v \acomm{\opr{f}_k^\dag}{\opr{g}_k^\dag}
    + v^* u \acomm{\opr{g}_k}{\opr{f}_k} \\
    &= \qty(\abs{u}^2 + \abs{v}^2) \idopr,
  \end{align}
  so we must have
  \begin{equation}
    \abs{u}^2 + \abs{v}^2 = 1.
  \end{equation}
  We may choose
  \begin{subequations}
    \begin{align}
      u &= e^{\im\phi_1}\cos\theta \\
      v &= e^{\im\phi_2}\sin\theta
    \end{align}\label{eq:uv-param}
  \end{subequations}
  for real angles $\phi_1$, $\phi_2$, and $\theta$.
}
The $\mq{H}_k$ are traceless, so they have the eigenvalues
\begin{align}
  E_k^\pm
  &= \pm\sqrt{-\det\mq{H}_k} \\
  &= \pm\sqrt{g^2 + 2g\cos k + \cos^2 k + \sin^2 k} \\
  &= \pm\sqrt{g^2 + 2g\cos k + 1}.
\end{align}
The eigenvectors are then
\begin{equation}
  \vq{q}_k^\pm
  = \bmqty{%
    -\im\sin k \\
    E_k^\pm + g + \cos k
  },
\end{equation}
except if $k = 0$ or $-\pi$, in which case
\begin{align}
  \vq{q}_k^-
  &= \bmqty{1 \\ 0}
  \qand
  \vq{q}_k^+
  = \bmqty{0 \\ 1}.
\end{align}
The $k = -\pi$ case does not appear if $N$ is odd. If also $g =
1$, then $\mq{H}_{-\pi} = \mq{0}$.
To construct the unitary
transformation, we must normalize the $\vq{q}_k^\pm$. We find that
\begin{align}
  \norm{\vq{q}_k^\pm}^2
  &= {\qty(E_k^\pm + g + \cos k)}^2 + \sin^2 k \\
  &= {\qty(E_k^\pm)}^2 + g^2 + \cos^2 k + 2g\cos k + 2E_k^\pm(g + \cos k) + 1 - \cos^2 k \\
  &= 2E_k^\pm \qty(E_k^\pm + g + \cos k).
\end{align}
Now
\begin{align}
  \frac{{\qty(\vq{q}_k^\pm)}_1}{\norm{\vq{q}_k^\pm}}
  &= \frac{-\im\sin k}{\sqrt{2E_k^\pm \qty(E_k^\pm + g + \cos k)}} \\
  &= \frac{-\im\sin k}{\sqrt{2\abs{E_k^\pm} \qty(\abs{E_k^\pm} \pm (g + \cos k))}}
  \\
  \shortintertext{and}
  %
  \frac{{\qty(\vq{q}_k^\pm)}_2}{\norm{\vq{q}_k^\pm}}
  &= \pm\sqrt{\frac{E_k^\pm + (g + \cos k)}{2E_k^\pm}} \\
  &= \pm\sqrt{\frac{\abs{E_k^\pm} \pm (g + \cos k)}{2\abs{E_k^\pm}}}
\end{align}
\begin{equation}
  \mq{U}_k^\dag
  = \bmqty{%
    {\qty(\hat{\vq{q}}_k^-)}^\dag \\
    {\qty(\hat{\vq{q}}_k^+)}^\dag
  }.
\end{equation}
Then with $E_k = \abs{E_k^\pm}$,
\begin{equation}
  \opr{\eta}_k^\pm
  = \frac{\im\sin k}{\sqrt{2E_k\qty(E_k \pm (g + \cos k))}}
  \opr{C}_k
  \pm \sqrt{\frac{E_k \pm (g + \cos k)}{2E_k}}
  \opr{C}^\dag_{-k}
  \label{eq:etas}
\end{equation}
so that
\begin{align}
  \acomm{{\qty(\opr{\eta}_k^\pm)}^\dag}{\opr{\eta}_k^\pm}
  &= \frac{\sin^2 k}{2E_k\qty(E_k \pm (g + \cos k))} \idopr
  + \frac{E_k \pm (g + \cos k)}{2E_k} \idopr \\
  &= \idopr
  \\
  \begin{split}
  \acomm{{\qty(\opr{\eta}_k^\pm)}^\dag}{\opr{\eta}_k^\mp}
  &= \frac{\sin^2 k}{2E_k
    \sqrt{E_k \pm (g + \cos k)}
  \sqrt{E_k \mp (g + \cos k)}} \idopr \\
  &- \frac{\sqrt{E_k \pm (g + \cos k)}\sqrt{E_k \mp (g + \cos k)}}{%
  2 E_k} \idopr
  \end{split}
  \\
  &= \zopr.
\end{align}
Note that \cref{eq:etas} is consistent with the edge cases in the limits $k \to
-\pi$ and $k \to 0$. If also $g = 1$, then we impose that $\opr{\eta}_{-\pi} =
\opr{C}_\pi^\dag$, which is the same as if $g \ne 1$.

\Cref{eq:Cham} becomes
\begin{equation}
  \ham
  = \sum_k E_k^+{\qty(\opr{\eta}_k^+)}^\dag\opr{\eta}_k^+
  + \sum_k E_k^-{\qty(\opr{\eta}_k^-)}^\dag\opr{\eta}_k^-.
  \label{eq:etasham}
\end{equation}
Since
\begin{equation}
  {\qty(\opr{\eta}_{-k}^-)}^\dag
  = \opr{\eta}_k^+
  \eqqcolon \opr{\eta}_k
\end{equation}
and $E_{-k}^\pm = E_k^\pm$, we may reduce \cref{eq:etasham} to
\begin{align}
  \ham
  &= \sum_k E_k\opr{\eta}_k^\dag \opr{\eta}_k
  - \sum_k E_k\qty(\idopr - \opr{\eta}_k^\dag \opr{\eta}_k)
  \\
  &= \sum_k 2E_k\opr{\eta}_k^\dag\opr{\eta}_k
  - \idopr\sum_k E_k.
  \label{eq:etaham}
\end{align}

\end{document}

