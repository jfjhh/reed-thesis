\chapter*{Introduction}
\addcontentsline{toc}{chapter}{Introduction}
\chaptermark{Introduction}
\markboth{Introduction}{Introduction}

The applications of quantum mechanics include exciting topics like the many-body
physics of materials like high-temperature superconductors, or new technologies
for quantum sensing and computation. One barrier towards making progress in
these areas is that quantum systems are notoriously difficult to simulate.
Consider simulating eight interacting spins (\cref{fig:spins}), which can be
thought of as similar to bar magnets. The analog in classical computing is to
manipulate memory on the scale of a few bytes (\SI{8}{bits}), which is trivial
to do. However, the way that we will simulate the spins requires diagonalizing a
matrix with dimension $4^8 = \num{65536}$, which is significantly more
difficult.

\begin{figure}[ht]
  \centering
  \includegraphics[width=0.5\textwidth]{spins}
  \caption{%
    Interacting spins.
  }\label{fig:spins}
\end{figure}

Nonetheless, this thesis must take the classical simulation approach to address
another barrier to these technologies: decoherence. Any novel quantum system
that we create is not isolated from the rest of the world. On one hand, it is
difficult to perfectly isolate a system, and on the other hand, if the system
were perfectly isolated, we could not communicate with it. Due to the inevitable
interaction of a system with its environment, the quantum states that an
experimenter so carefully sets up threaten to be destroyed with time. If the
quantum state of the spins is like a house of cards, then environment-induced
decoherence is a bit like leaving the cards out in the rain. Depending on how
hard it rains, the cards may fall sooner or later. The cards might even buckle
with no rain due to the humidity of being outside. An important aspect of the
different approaches to making quantum computers is the coherence time: how long
the quantum state goes without being destroyed.

In the context of quantum computing, this decoherence is often modeled as a kind
of random error, like a bit flipping unexpectedly. However, this is not the full
story for decoherence. We generally expect the interaction with an environment
at some temperature to result in the system relaxing towards that temperature,
like an ice cube melting in the sun. The rate at which this happens depends on
the details of the interaction with the environment and the system itself. A bit
flip type model for decoherence may be too reductive. The situation is like
timing how long it takes an ice cube to melt in the sun, and then trying to use
that to predict how long it takes an ice sculpture to melt. If the sculpture is
a bigger ice cube, you might predict well, but if the sculpture is a snowman
under an ice umbrella, your calculations might be off by quite a bit. This
thesis will investigate a particular model of interacting spins to assess the
effect of system composition on the relaxation time.

In particular, \cref{ch:open-quantum-systems} will review quantum mechanics and
develop the relevant theory of open quantum systems needed to study the
relaxation of a system due to environment-induced decoherence. \Cref{ch:spins}
will give a simple application of the theory to a two-level atom, and then
discuss a more involved application to our main system of study, the
transverse-field Ising model. These results will then enable numerical
computations for the relaxation rates of the transverse-field Ising model in
\cref{ch:computations}, which we will use to assess the effect of the spin
interactions on relaxation.


\chapter*{Note on Notation}
\addcontentsline{toc}{chapter}{Note on Notation}
\chaptermark{Note on Notation}
\markboth{Note on Notation}{Note on Notation}

The mathematical formalism for quantum mechanics requires three levels of linear
algebra: vectors, operators (maps between vectors), and superoperators (maps
between operators). We will write each object with slightly differerent notation
so that the type of an expression may be inferred if the reader is confused.

\begin{multicols}{2}
  \begin{tabular}{rl}
    Vector & $\ket{v}$ \\
    $\ket{v}$ in coordinates & $\vq{v}$ \\
    Tuple & $\tq{n}$ \\
    Operator & $\opr{A}$ \\
    Vector operator & $\vecopr{B}$ \\
    $\opr{A}$ in coordinates & $\mq{A}$ \\
    Superoperator & $\sopr{A}$ \\
    One & $1$ \\
    Identity operator & $\idopr$ \\
    Identity superoperator & $\idsopr$ \\
    Zero & $0$ \\
    Zero operator & $\zopr$ \\
    Zero superoperator & $\zsopr$ \\
    \\
    Sign of $x$ & $\sgn x$ \\
    Sinc function & $\sinc x = \dfrac{\sin x}{x}$
  \end{tabular}
  \vfill\null\columnbreak%
  \begin{tabular}{rl}
    Real numbers & $\RR$ \\
    Complex numbers & $\CC$ \\
    Integers modulo $n$ & $\ZZ_n$ \\
    Hilbert space & $\hilb$ \\
    Bounded operators & $\bdd(\hilb)$ \\
    Liouville space & $\liou(\hilb)$ \\
    Hamiltonian operator & $\ham$ \\
    Density operator & $\dop$ \\
    Pauli operators & $\pauli_i$ \\
    Spin operators & $\spin_i = \hbar\pauli_i / 2$ \\
    Inner product on $\liou$ & $\lprod{\opr{A}}{\opr{B}}$ \\
    Transpose of $\mq{A}$ & $\mq{A}^\trans$ \\
    Adjoint of $\opr{A}$ & $\opr{A}^\dag$ \\
    Trace of $\opr{A}$ & $\tr A$ \\
    Partial trace over $S$ & $\tr_S$ \\
    Expected value of $\opr{A}$ & $\ev{A} = \tr(\dop\opr{A})$ \\
    Hermitian part of $\opr{A}$ & $\herm\opr{A} = \dfrac{\opr{A} + \opr{A}^\dag}{2}$
  \end{tabular}
\end{multicols}

The text uses \term{natural units} where $\hbar = c = \varepsilon_0 = m_e = 1$.

\ifoptionfinal{}{%
  This is the color version of the document. Plots, code, and links are
  \textcolor{rubric}{colored}. Example: continue to
  \cref{ch:open-quantum-systems}.
}

