\documentclass[11pt]{article} % Document font size and paper size

\usepackage{fontspec} % Allows the use of OpenType fonts
\usepackage{microtype} % Subliminal refinements towards typographical perfection
\usepackage{siunitx} % SI units
\usepackage{physics}
\usepackage{enumitem} % Customize lists
\setlist[itemize]{noitemsep}
\usepackage{geometry} % Allows the configuration of document margins
% \geometry{a4paper} % Document margin settings
\geometry{a4paper, margin=1in} % Document margin settings
% \setlength\parindent{0in} % Remove paragraph indentation
\usepackage[usenames,dvipsnames]{xcolor} % Custom colors
\usepackage{setspace} % For double-spacing
\usepackage[labelfont=bf,center]{caption}
\usepackage[normalem]{ulem} % Custom underlining
\usepackage{xunicode} % Allows generation of unicode characters from accented glyphs
\defaultfontfeatures{Mapping=tex-text} % Converts LaTeX specials (``quotes'' --- dashes etc.) to unicode

\usepackage[style=numeric,doi=false,url=false,isbn=false]{biblatex}
\addbibresource{references.bib}

\newbibmacro{string+doiurlisbn}[1]{%
  \iffieldundef{doi}{%
    \iffieldundef{url}{%
      \iffieldundef{isbn}{%
        \iffieldundef{issn}{%
          #1%
        }{%
          \href{http://books.google.com/books?vid=ISSN\thefield{issn}}{#1}%
        }%
      }{%
        \href{http://books.google.com/books?vid=ISBN\thefield{isbn}}{#1}%
      }%
    }{%
      \href{\thefield{url}}{#1}%
    }%
  }{%
    \href{http://dx.doi.org/\thefield{doi}}{#1}%
  }%
}

\DeclareFieldFormat{title}{\usebibmacro{string+doiurlisbn}{\mkbibemph{#1}}}
\DeclareFieldFormat[article,incollection]{title}%
    {\usebibmacro{string+doiurlisbn}{\mkbibquote{#1}}}


\usepackage{fancyhdr}
\pagestyle{fancy}
\fancyhead[L]{Alex Striff}
\fancyhead[C]{\textsc{Physics Thesis Proposal \liningnums{\thesection}}}
\fancyhead[R]{September 11, 2020}

% Linux Libertine Font (default)
% \setromanfont[Ligatures={Common}, Numbers={OldStyle}]{Linux Libertine O} % Main text font
% \setsansfont[Ligatures={Common}, Numbers={OldStyle}]{Linux Biolinum O} % Main text font
\usepackage[semibold,osf]{libertinus}
\setmonofont[Scale=0.8]{Monaco} % Set mono font (e.g. phone numbers)

\usepackage{sectsty}
\allsectionsfont{\normalfont\sffamily}

\renewcommand{\&}{}

\newcommand{\im}{\mathrm{i}}
\newcommand{\ham}{\mathsf{H}}
\newcommand{\dop}{\mathsf{ρ}}
\newcommand{\spin}{\mathsf{σ}}
\newcommand{\idopr}{\mathsf{I}}

\usepackage[xetex, unicode, bookmarks, colorlinks, breaklinks,
pdftitle={Physics Thesis Proposals},
pdfauthor={Alexander Striff}]{hyperref}
\hypersetup{linkcolor=MidnightBlue,citecolor=MidnightBlue,filecolor=black,urlcolor=MidnightBlue} % Link colors
\urlstyle{same}

\usepackage{kantlipsum}

\begin{document}
\title{Physics Thesis Proposals}
\author{Alex Striff}
\date{September 11, 2020}
% \pagenumbering{gobble}

\section{Toy Systems and Quantum Master Equations
\\ \large \emph{Darrell Schroeter}}

\begin{refsection}
How do quantum systems behave when interacting with their environments? This
question is usually addressed by considering Markovian quantum master equations
like the Lindblad equation. However, the assumptions of memoryless dynamics or
being able to average over some environment behavior are often invalid. To
assess the validity of these assumptions and to understand the more general
dynamics, I propose to find exact (numerical) solutions for at least one toy
system and to evaluate the effect of the environment.

In particular, I propose to start with a transverse-field Ising model with the
Hamiltonian
\[
  \ham
  = -\sum_{i \in \mathbf{Z}_n} \qty(J_i \spin_{iz} \spin_{(i+1)z}
  + \mu_0 h_i \spin_{ix}),
\]
where $\spin_{1z} = \spin_z \otimes \idopr \otimes \cdots \otimes \idopr$, etc.
and $\spin_{i0} = \idopr$ that is coupled to a Markovian bath so that
\[
  \dot\dop
  = \frac{\comm{\ham}{\dop}}{\im\hbar}
  + \sum_{i\in\mathbf{Z}_n,\, a\in\{x,z\}} \gamma_{ia} \qty(\mathsf{L}_{ia}
  \dop \mathsf{L}_{ia}^\dag
  - \frac{1}{2}\acomm{\mathsf{L}_{ia}\mathsf{L}_{ia}^\dag}{\dop}),
\]
where $\mathsf{L}_{iz} = \qty(\spin_{ix} - \im\spin_{iy})/2$ and
$\mathsf{L}_{iz} = \qty(\spin_{iy} - \im\spin_{iz})/2$, similar to
in~\cite{yoshiokaConstructingNeuralStationary2019,jinPhaseDiagramDissipative2018}.

I would consider one or a few spins as a subsystem, and the rest as the (inner)
environment. First, I would exactly diagonalize the closed system with
$\Gamma_{ia} = 0$ and study the behavior of the subsystem as the size $n$ of the
full system grows, which I expect will reach to about 16. What are the
stationary states and what is the time evolution of the expectation values of
quantities like the subsystem Hamiltonian, net magnetization, and von Neumann
entropy? Does the closed system thermalize any of these quantities as it grows?

I would then study how the quantities mentioned behave differently as the
interaction with a Markovian environment is turned on in three regimes where
\begin{itemize}
  \item the isolated subsystem of a few spins interacts with the bath,
  \item the full Ising chain including subsystem interacts with the bath, and
  \item the environment Ising chain but not subsystem interacts with the bath.
\end{itemize}
That is, how do different kinds of composite environments effect the behavior of
the few-spin subsystem in question?

While I have proposed this investigation for the Ising model, the target system
and desired quantities may change as I research the Ising model, various
generalizations like the Heisenberg or Potts models, and other possible target
systems like coupled harmonic oscillators or discrete-level systems, like
in~\cite{bhattacharyaUnderstandingDampingQuantum2012}.

\printbibliography[heading=subbibliography]%
\end{refsection}

\clearpage
\section{Dynamics of an electronic oscillator with step nonlinearity and bandpassed delayed feedback
\\ \large \emph{Lucas Illing}}

\begin{refsection}
This thesis would extend and verify the work of Kees and myself from prior
summers. While Colleen Werkheiser's thesis has shown that fast-oscillating
periodic solutions in a time-delayed system with lowpass filtering are unstable,
it is possible that these fast solutions may be stabilized with the use of
bandpass feedback.

The use of a programmable time-delay device enables the experiments with an
electronic circuit to potentially be automated. Thus, a detailed examination of
the parameter space would be possible. This experimental approach would also be
checked against theoretical predictions and similarly detailed numerical computations.

By forming a full understanding of fast-oscillating solutions in this system, we
inform the study of more complex systems and the conditions which give rise to
so-called virtual chimera states~\cite{largerVirtualChimeraStates2013}.

\printbibliography[heading=subbibliography]%
\end{refsection}

\end{document}

