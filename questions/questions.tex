\documentclass[12pt]{article}

% Core packages
\usepackage{mathtools} % Load before newtxmath
\mathtoolsset{%
  mathic,
  showonlyrefs,
}
% \allowdisplaybreaks%

\usepackage[%
math-style=ISO,
bold-style=ISO,
sans-style=italic,
nabla=upright,
partial=upright,
]{unicode-math}

\usepackage[%
backend=biber,
% backref=true,
sorting=none,
style=phys,
biblabel=brackets,
citestyle=numeric-comp,
doi=false,
url=false,
isbn=false,
]{biblatex}

% \defbibheading{bibliography}[\bibname]{%
%   \section*{#1}
%   \sectionmark{#1}
%   \markboth{#1}{#1}
% }

\AtEveryBibitem{\clearlist{language}\clearfield{note}\clearfield{pagetotal}}
\AtEveryCitekey{\clearlist{language}\clearfield{note}\clearfield{pagetotal}}

\DefineBibliographyStrings{english}{%
  backrefpage = {page},% originally "cited on page"
  backrefpages = {pages},% originally "cited on pages"
}

\newbibmacro{string+doiurlisbn}[1]{%
  \iffieldundef{doi}{%
    \iffieldundef{url}{%
      \iffieldundef{isbn}{%
        \iffieldundef{issn}{%
          #1%
          }{%
          \href{http://books.google.com/books?vid=ISSN\thefield{issn}}{#1}%
        }%
        }{%
        \href{http://books.google.com/books?vid=ISBN\thefield{isbn}}{#1}%
      }%
      }{%
      \href{\thefield{url}}{#1}%
    }%
    }{%
    \href{http://dx.doi.org/\thefield{doi}}{#1}%
  }%
}

\DeclareFieldFormat{title}{\usebibmacro{string+doiurlisbn}{\mkbibemph{#1}}}
% \DeclareFieldFormat[article,incollection]{title}%
% \DeclareFieldFormat[article]{title}%
% {\usebibmacro{string+doiurlisbn}{\mkbibquote{#1}}}

\addbibresource{thesis.bib}

\usepackage{ifdraft} % final: for printing (set bw, etc.)
\usepackage{xstring}
\usepackage{pgffor}
\usepackage{subfiles}
\newcommand{\notebook}[1]{\subfile{../notebooks/tex/#1}}


% Layout and styling
\usepackage{etoolbox}
\usepackage{multicol}
\usepackage[final]{microtype}
\usepackage{booktabs}
\usepackage{enumitem}
\usepackage{rotating}
\usepackage{fnpct}
\setlist[enumerate,1]{label=\textbf{\arabic*.}}

\usepackage[labelfont=bf]{caption}
\usepackage{sectsty}
\allsectionsfont{\normalfont\scshape}
\chapterfont{\normalfont\scshape\liningnums}

% Fonts
\usepackage[full]{textcomp} % to get the right copyright, etc.
\usepackage[semibold]{libertinus-otf}
\usepackage[T1]{fontenc} % LY1 also works
\setmainfont[Numbers={OldStyle,Proportional}]{Libertinus Serif} % We need lots of unicode, like ⊗
\usepackage[supstfm=libertinesups,supscaled=1.2,raised=-.13em]{superiors}
\setmonofont[Scale=MatchLowercase]{JuliaMono} % We need lots of unicode, like ⊗
\usepackage[cal=cm,bb=boondox,frak=boondox]{mathalfa}

% 1.5 / 1.208, where 1.208 is the default baseline skip to font size ratio.
% That is, for 12pt font, the default baseline skip is 14.5pt
% https://www.tug.org/svn/texlive/trunk/Master/texmf-dist/tex/latex/base/size12.clo?view=co
% \linespread{1.24137}

\usepackage{parskip}


% Code highlighting
% \usepackage{textcomp}
% \usepackage{tcolorbox}
% \usepackage[final]{minted}
% \setminted{%
%   breaklines,
%   fontsize=\normalsize,
%   baselinestretch=1,
% }
% \ifoptionfinal{%
%   \setminted{style=bw}
%   }{%
%   \setminted{style=friendly}
%   \setminted[wolfram]{style=mathematica}
% }

% Graphics
\PassOptionsToPackage{final}{graphicx}
\def\figwidth{\linewidth}
\usepackage{xcolor}
\usepackage{graphicx}
\setkeys{Gin}{draft=false}

\graphicspath{{../figs/}{../notebooks/}{../notebooks/tex/}}

% More preamble commands, separated for reuse by notebooks
%% Math packages
\usepackage{physics}
\usepackage{siunitx}


%% Math commands

% Semantic commands
\newcommand{\defeq}{\equiv}
\newcommand{\iso}{\cong}
\newcommand{\by}[1]{\qq{#1}}
\newcommand{\byeq}[1]{\by{\cref{#1}}}

% Variant symbols
\renewcommand{\le}{\leqslant}
\renewcommand{\ge}{\geqslant}
\renewcommand\phi\varphi%
\renewcommand\epsilon\varepsilon%

% Common algebraic objects
\newcommand\ZZ{\mathbb{Z}}
\newcommand\QQ{\mathbb{Q}}
\newcommand\CC{\mathbb{C}}
\newcommand\RR{\mathbb{R}}

% Named operations and special constants
\newcommand\im{\mathrm{i}\mkern1mu}
% \newcommand\im{i\mkern1mu}

% Operators
\newcommand\tp\otimes%
\newcommand\opr\symsf%
\newcommand\vecopr\symbfsf%
% \newcommand{\opr}[1]{\mathnormal{#1}}%
\newcommand\idopr{\opr{1}}
\newcommand\zopr{\opr{0}}
\newcommand\sopr\mathcal%
\newcommand\idsopr{\mathbb{1}}
\newcommand\zsopr{\mathbb{0}}
\newcommand\herm{\operatorname{\mathrm{He}}}
\newcommand\sgn{\operatorname{\mathrm{sgn}}}
\newcommand\sinc{\operatorname{\mathrm{sinc}}}
\newcommand\atanh{\operatorname{\mathrm{atanh}}}
\newcommand\acoth{\operatorname{\mathrm{acoth}}}

% Categories
\newcommand\cat\symsf%

% Linear algebra
\newcommand\tq[1]{\symbf{#1}} % Tuples
\newcommand\vq[1]{\symbf{#1}} % Vectors
\providecommand\vu{}
\renewcommand\vu[1]{\hat{\vq{#1}}} % Unit vectors
\newcommand\mq[1]{\symbf{#1}} % Matrices (not transformations)
\newcommand\trans{{\symsfup{T}}} % this is the one case when we want it upright

% Quantum mechanics
\newcommand\hilb{\mathcal{H}}
\newcommand\liou{\mathcal{L}}
\newcommand\bdd{\mathcal{B}}
\newcommand\ham{\opr{H}}
\newcommand\dop{\opr{ρ}}
\newcommand\pauli{\opr{σ}}
\newcommand\paulivec{\vecopr{σ}}
\newcommand\spin{\opr{S}}
\newcommand\spinvec{\vecopr{S}}
\newcommand\tprod\otimes%
\newcommand\lprod\ip%
% \newcommand\lprod[2]{\qty(#1,#2)}
\newcommand\ensavg[2][\dop]{\ev{#2}_{\mkern-1.5mu{#1}}}
\newcommand\sensavg[2][\dop]{\ev*{#2}_{\mkern-1.5mu{#1}}}


% % ANSI colors (from nbconvert's base.tplx)
% Used to display terminal output from notebooks.

% \ifoptionfinal{% Black and white version
  \definecolor{ansi-black}{HTML}{000000}
  \definecolor{ansi-black-intense}{HTML}{000000}
  \definecolor{ansi-red}{HTML}{000000}
  \definecolor{ansi-red-intense}{HTML}{000000}
  \definecolor{ansi-green}{HTML}{000000}
  \definecolor{ansi-green-intense}{HTML}{000000}
  \definecolor{ansi-yellow}{HTML}{000000}
  \definecolor{ansi-yellow-intense}{HTML}{000000}
  \definecolor{ansi-blue}{HTML}{000000}
  \definecolor{ansi-blue-intense}{HTML}{000000}
  \definecolor{ansi-magenta}{HTML}{000000}
  \definecolor{ansi-magenta-intense}{HTML}{000000}
  \definecolor{ansi-cyan}{HTML}{000000}
  \definecolor{ansi-cyan-intense}{HTML}{000000}
  \definecolor{ansi-white}{HTML}{000000}
  \definecolor{ansi-white-intense}{HTML}{000000}
  \definecolor{ansi-default-inverse-fg}{HTML}{000000}
  \definecolor{ansi-default-inverse-bg}{HTML}{000000}
  % }{% Color version
  % \definecolor{ansi-black}{HTML}{3E424D}
  % \definecolor{ansi-black-intense}{HTML}{282C36}
  % \definecolor{ansi-red}{HTML}{E75C58}
  % \definecolor{ansi-red-intense}{HTML}{B22B31}
  % \definecolor{ansi-green}{HTML}{00A250}
  % \definecolor{ansi-green-intense}{HTML}{007427}
  % \definecolor{ansi-yellow}{HTML}{DDB62B}
  % \definecolor{ansi-yellow-intense}{HTML}{B27D12}
  % \definecolor{ansi-blue}{HTML}{208FFB}
  % \definecolor{ansi-blue-intense}{HTML}{0065CA}
  % \definecolor{ansi-magenta}{HTML}{D160C4}
  % \definecolor{ansi-magenta-intense}{HTML}{A03196}
  % \definecolor{ansi-cyan}{HTML}{60C6C8}
  % \definecolor{ansi-cyan-intense}{HTML}{258F8F}
  % \definecolor{ansi-white}{HTML}{C5C1B4}
  % \definecolor{ansi-white-intense}{HTML}{A1A6B2}
  % \definecolor{ansi-default-inverse-fg}{HTML}{FFFFFF}
  % \definecolor{ansi-default-inverse-bg}{HTML}{000000}
% }



% Reference management (order-sensitive)
\usepackage{xparse}
% \usepackage[pdfusetitle,final,backref,breaklinks]{hyperref}
\usepackage[pdfusetitle,final,breaklinks]{hyperref}
\usepackage[all]{hypcap}
\usepackage{cleveref}

\definecolor{linkcolor}{rgb}{0.0,0.25,0.5}
\definecolor{rubric}{rgb}{0.7,0.05,0.0}
\definecolor{resultbg}{rgb}{0.95,0.95,0.95}
\hypersetup{%
  allcolors=rubric,
}
\urlstyle{same}
\ifoptionfinal{\hypersetup{hidelinks}}{\hypersetup{colorlinks}}

\title{Thesis Defense Questions}
\author{Alexander B{.}~Striff}
\date{May 2021}

\begin{document}
\thispagestyle{empty}
\maketitle

\section{Principal value integrals}

The \emph{principal value integral} of a real function $f:[a, b] \to \RR$ with a
pole at $x_0 \in [a, b]$ is defined as
\begin{equation}
  \pv\int_a^b \dd{x} f(x)
  = \lim_{\varepsilon \to 0^+} \qty(%
  \int_a^{x_0 - \varepsilon} \dd{x} f(x)
  \int_{x_0 + \varepsilon}^b \dd{x} f(x)
  ),
  \label{eq:pv}
\end{equation}
where $\int_a^{x_0} \dd{x} f(x) = \pm\infty$ and $\int_{x_0}^b \dd{x} f(x) =
\mp\infty$. We will integrate a simplified version of eq.~(2.84):
\begin{equation}
  \pv\int_0^b \dd{u} \frac{u^2}{1 - u^2},
\end{equation}
where $b > 1$. First, we will compute the indefinite integral by writing the
integrand as
\begin{equation}
  \frac{u^2}{1 - u^2}
  = \frac{A(u)}{1 - u} + \frac{B(u)}{1 + u}.
\end{equation}
This requires that $A(u) = -B(u) = u/2$, so we find that
\begin{align}
  \int\dd{u}\frac{u^2}{1 - u^2}
  &= \frac{1}{2}\int\dd{u}\qty(\frac{u}{1 - u} - \frac{u}{1 + u}) \\
  &= \frac{1}{2}\qty(-u - \ln(1 - u) - u + \ln(1 + u)) + C \\
  &= -u + \atanh u + C.
\end{align}
Now we substitute the limits of integration to find that
\begin{align}
  &\pv\int_a^b \dd{x} f(x) \\
  &=
  0 - \atanh 0
  -b + \atanh b \\
  &\quad+ \lim_{\varepsilon \to 0^+} \qty(%
  -(1 - \varepsilon) + \atanh(1 - \varepsilon)
  + (1 + \varepsilon) - \atanh(1 + \varepsilon)
  ) \\
  &=
  -b + \atanh b
  + \lim_{\varepsilon \to 0^+} \qty(%
  \atanh(1 - \varepsilon)
  - \atanh(1 + \varepsilon)
  ) \\
  &=
  -b + \frac{1}{2}\ln(\frac{1 + b}{1 - b})
  + \frac{1}{2} \lim_{\varepsilon \to 0^+}
  \ln(\frac{(2-\varepsilon)(-\varepsilon)}{\varepsilon(2+\varepsilon)})
  \\
  &=
  -b + \frac{1}{2}\ln(\frac{1 + b}{1 - b})
  + \frac{1}{2} \ln(-1)
  \\
  &=
  -b + \frac{1}{2}\ln(\frac{b + 1}{b - 1})
  \\
  &= -b + \acoth b,
\end{align}
since $\ln(-z) = \ln z + \im\pi$.

\section{Time-dependent perturbation theory}

There are a few differences between the open system approach in the thesis and
the perturbative calculations
in~\cite{griffithsIntroductionQuantumMechanics2018}.

The weak-coupling Lindblad equation only describes the time evolution on a
coarse-grained time scale, so it will produce incorrect results for time scales
beneath the bath correlation time. In the Markovian regime the reduced dynamics
show exponential decay of populations at small times, while the true populations
are initially unchanging. However, the general idea of reduced dynamics is
consistent with perturbation theory. The two-level system is a special case
where the composite system may be exactly reduced, and the decoherence results
found agree with the quadratic transition probability (eq.~(11.43)
of~\cite{griffithsIntroductionQuantumMechanics2018}) for small times found in
perturbation theory~\cite[p.~230]{opensys}. (This limit demonstrates the quantum
Zeno effect, but the full theory of open systems is required to model an
indirect continuous measurement by the interaction of a system with a probe
particle~\cite[p.~167--171]{opensys}.)

The other difference is in the treatment of the quantum state and the
electromagnetic field. The perturbative approach considers only pure states and
incoherent radiation (see footnote 12
in~\cite[p.~414]{griffithsIntroductionQuantumMechanics2018}). This suffices to
find the absorption rate for a thermal bath, but fails to predict spontaneous
emission directly without recourse to statistical mechanics. The open system
approach requires quantization of the electromagnetic field, and thus enables
all processes including spontaneous emission to be modeled (seen in eq.~(2.11)).
Additionally, one may consider entangled states of light or of the system that
cannot be described by pure states in the perturbative approach. Only the open
system approach shows how a pure state can evolve into a mixed state, and allow
one to study the entropy flux due to dissipative effects of the
bath~\cite[p.~128]{opensys}.

\section{Spaghetti diagrams}

For the purposes of the thesis, a \emph{spaghetti diagram} is a plot that shows
how the eigenvalues of a Hermitian matrix $\ham(\lambda)$ vary as a function of
the parameter $\lambda$. They are used in the thesis to provide insight into the
energy levels of the normalized Ising Hamiltonian (eq.~(2.35)) and the spectrum
of all possible relaxation rates for a system. A simple example of a spaghetti
diagram is given by perturbing the infinite square well. We know that the energy
levels of the square well grow like $n^2$, but what happens when the bump of
\cref{fig:bump} is introduced?
\begin{figure}[h]
  \centering
  \includegraphics[width=0.75\linewidth]{bump}
  \caption{%
    The potential bump added to the square well.
  }\label{fig:bump}
\end{figure}
Let the height of the bump be $\lambda$. We expect that $\lambda = 0$ is the
usual square well and that $\lambda \to \infty$ should be similar to two square
wells with twofold degeneracy in each energy level, but what happens in between?
This is what spaghetti diagrams can show us. Using Darrell's finite difference
code from Quantum II, we may compute the spaghetti diagram for $\lambda$
(\cref{fig:spaghetti}).
\begin{figure}[h]
  \centering
  \includegraphics[width=0.75\linewidth]{spaghetti}
  \caption{%
    A spaghetti diagram for increasing the height of the bump.
  }\label{fig:spaghetti}
\end{figure}
The diagram tells us three things about the transition between the two extremes
of $\lambda$.
\begin{itemize}
  \item The twofold degeneracy as $\lambda \to \infty$ is reproduced.

  \item The bump height $\lambda$ required to join two energy levels increases
    with $n$. (The middle of the transitions for the blue, green, and purple
    levels shifts slightly further to the right as $n$ increases.)

  \item The significant width of the bump causes the energy levels of the two
    new wells in the $\lambda \to \infty$ limit to be higher than those of the
    unperturbed well, since the effective width is smaller than $a / 2$.
\end{itemize}

\printbibliography%

\end{document}

