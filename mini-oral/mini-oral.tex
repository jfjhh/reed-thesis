\documentclass[12pt,c]{beamer}

\usepackage[normalem]{ulem}
\usepackage{multicol}

\usepackage{mathtools} % Load before newtxmath
\mathtoolsset{%
  mathic,
}
\usepackage[%
math-style=ISO,
bold-style=ISO,
sans-style=italic,
nabla=upright,
partial=upright,
]{unicode-math}


\definecolor{rubric}{rgb}{0.7,0.05,0.0}

\usecolortheme[snowy]{owl}
\setbeamercolor{title}{fg=rubric}
\setbeamerfont{frametitle}{shape=\itshape}
\setbeamercolor{frametitle}{bg=white}
\setbeamertemplate{frametitle}[default]%[center]
\setbeamertemplate{caption}[numbered]
\setbeamertemplate{caption label separator}{: }
\setbeamercolor{caption name}{fg=normal text.fg}
\setbeamercolor{section in toc}{fg=normal text.fg}
\setbeamercolor{footnote}{fg=gray}
\setbeamercolor{alerted text}{fg=rubric}
\setbeamercolor*{bibliography item}{fg=normal text.fg}
\setbeamercolor*{bibliography entry author}{fg=normal text.fg}
\setbeamercolor*{bibliography entry title}{fg=rubric}
\setbeamercolor*{bibliography entry location}{fg=normal text.fg}
\setbeamercolor*{bibliography entry note}{fg=normal text.fg}
\addtobeamertemplate{navigation symbols}{}{%
    \usebeamerfont{footline}%
    \color{normal text.fg}
    \hspace{1em}%
    \insertframenumber\,/\,\inserttotalframenumber%
}

\usepackage[sorting=none,doi=false,url=false,isbn=false]{biblatex}
\addbibresource{references.bib}
\renewcommand*{\bibfont}{\footnotesize}

% Fonts
\usefonttheme[stillsansserifsmall]{serif} % So beamer won't change fonts
\usepackage[full]{textcomp} % to get the right copyright, etc.
\usepackage{libertinus-otf}
\usepackage[T1]{fontenc} % LY1 also works
\setmainfont[Numbers={OldStyle,Proportional}]{fbb}
\setsansfont[Scale=0.95]{Cabin}
\usepackage[supstfm=fbb-Regular-sup-t1]{superiors}
% \setmonofont[Scale=MatchLowercase]{DejaVu Sans Mono}
\setmonofont[Scale=MatchLowercase]{Iosevka} % We need unicode
\usepackage[cal=boondoxo,bb=boondox,frak=boondox]{mathalfa}

\newcommand{\spacedsc}{\scshape\addfontfeatures{LetterSpace=5}}


% use microtype if available
\IfFileExists{microtype.sty}{\usepackage{microtype}}{}
\hypersetup{%
  pdfborder={0 0 0},
  breaklinks=true,
}
\urlstyle{same} % don't use monospace font for urls

% \setlength{\parindent}{0pt}
% \setlength{\parskip}{6pt plus 2pt minus 1pt}
% \setlength{\emergencystretch}{3em}  % prevent overfull lines
\providecommand{\tightlist}{%
  \setlength{\itemsep}{0pt}\setlength{\parskip}{0pt}}
\setcounter{secnumdepth}{0}

\usepackage{physics}
\usepackage{siunitx}
\usepackage{pgfplots}
\usepackage{tikz}
\usetikzlibrary{arrows}
\tikzset{% Animate with beamer overlays
  invisible/.style={opacity=0},
  visible on/.style={alt={#1{}{invisible}}},
  alt/.code args={<#1>#2#3}{%
    \alt<#1>{\pgfkeysalso{#2}}{\pgfkeysalso{#3}} % \pgfkeysalso doesn't change the path
  },
}

\graphicspath{{./figs/}}

%% Math packages
\usepackage{physics}
\usepackage{siunitx}


%% Math commands

% Semantic commands
\newcommand{\defeq}{\equiv}
\newcommand{\iso}{\cong}
\newcommand{\by}[1]{\qq{#1}}
\newcommand{\byeq}[1]{\by{\cref{#1}}}

% Variant symbols
\renewcommand{\le}{\leqslant}
\renewcommand{\ge}{\geqslant}
\renewcommand\phi\varphi%
\renewcommand\epsilon\varepsilon%

% Common algebraic objects
\newcommand\ZZ{\mathbb{Z}}
\newcommand\QQ{\mathbb{Q}}
\newcommand\CC{\mathbb{C}}
\newcommand\RR{\mathbb{R}}

% Named operations and special constants
\newcommand\im{\mathrm{i}\mkern1mu}
% \newcommand\im{i\mkern1mu}

% Operators
\newcommand\tp\otimes%
\newcommand\opr\symsf%
\newcommand\vecopr\symbfsf%
% \newcommand{\opr}[1]{\mathnormal{#1}}%
\newcommand\idopr{\opr{1}}
\newcommand\zopr{\opr{0}}
\newcommand\sopr\mathcal%
\newcommand\idsopr{\mathbb{1}}
\newcommand\zsopr{\mathbb{0}}
\newcommand\herm{\operatorname{\mathrm{He}}}
\newcommand\sgn{\operatorname{\mathrm{sgn}}}
\newcommand\sinc{\operatorname{\mathrm{sinc}}}
\newcommand\atanh{\operatorname{\mathrm{atanh}}}
\newcommand\acoth{\operatorname{\mathrm{acoth}}}

% Categories
\newcommand\cat\symsf%

% Linear algebra
\newcommand\tq[1]{\symbf{#1}} % Tuples
\newcommand\vq[1]{\symbf{#1}} % Vectors
\providecommand\vu{}
\renewcommand\vu[1]{\hat{\vq{#1}}} % Unit vectors
\newcommand\mq[1]{\symbf{#1}} % Matrices (not transformations)
\newcommand\trans{{\symsfup{T}}} % this is the one case when we want it upright

% Quantum mechanics
\newcommand\hilb{\mathcal{H}}
\newcommand\liou{\mathcal{L}}
\newcommand\bdd{\mathcal{B}}
\newcommand\ham{\opr{H}}
\newcommand\dop{\opr{ρ}}
\newcommand\pauli{\opr{σ}}
\newcommand\paulivec{\vecopr{σ}}
\newcommand\spin{\opr{S}}
\newcommand\spinvec{\vecopr{S}}
\newcommand\tprod\otimes%
\newcommand\lprod\ip%
% \newcommand\lprod[2]{\qty(#1,#2)}
\newcommand\ensavg[2][\dop]{\ev{#2}_{\mkern-1.5mu{#1}}}
\newcommand\sensavg[2][\dop]{\ev*{#2}_{\mkern-1.5mu{#1}}}



\renewcommand\footnoterule{}
\renewcommand\thefootnote{\textcolor{gray}{\arabic{footnote}}}
% \setbeamertemplate{footnote}{%
%   \scriptsize\noindent\raggedright$^*$\insertfootnotetext\par%
% }

\title{Interaction of the transverse-field Ising model with a bath}
\author{Alex Striff \\ Advisor: Darrell Schroeter}
\institute{\itshape Reed College \\ Thesis Mini Oral}
\date{December 8, 2020}

\begin{document}

{%
  \beamertemplatenavigationsymbolsempty%
  \begin{frame}<article:0>[plain]
    \begin{tikzpicture}[remember picture,overlay,opacity=0.333333]
      \node[at=(current page.center)] {%
          % \includegraphics[keepaspectratio,
          % width=\paperwidth,
          % height=\paperheight]{strogatz}
        };
    \end{tikzpicture}
    \titlepage%
  \end{frame}
}

\begin{frame}{Outline}
  \tableofcontents
\end{frame}

\section{Why care about the transverse-field Ising model?}

\begin{frame}{The transverse-field Ising model}
  \[
    \ham
    = -\sum_{i \in \ZZ_N} \pauli_i^x\pauli_{i+1}^x
    - \lambda\sum_{i \in \ZZ_N} \pauli_i^z
  \]
  \begin{itemize}
    \item Cannot be described with classical statistical mechanics
    \item Show thermalization from microscopic interactions
    \item Undergoes a quantum phase transition at $\lambda = 1$
    \item Learn new techniques
    \item Relatively simple
  \end{itemize}
\end{frame}

\section{What is an open quantum system?}

\begin{frame}{Open quantum systems}
  A quantum system is described by a \alert{density operator} $\dop$
  \vspace{\baselineskip}

  \textbf{Setup}:
  Liouville space $\liou$ of operators with finite norm
  \[
    \lprod{\opr{A}}{\opr{B}}
    = \tr(\opr{A}^\dag \opr{B})
  \]

  \textbf{Expectations}: Self-adjoint $\opr{O}$ has $\ev{\opr{O}} =
  \lprod{\dop}{\opr{O}}$

  \begin{itemize}
    \item $\ev{\alpha\idopr} = \alpha$ for $\alpha \in \CC$ requires $\tr\dop =
      1$

    \item All $\ev{\opr{O}}$ real requires $\dop = \dop^\dag$

    \item All $\ev{\opr{O}}$ positive for positive $\opr{O}$ requires $\dop$
      positive
  \end{itemize}
\end{frame}

\begin{frame}{Composite systems}
  \alert{Epistemically independent} systems $A$ and $B$ form a composite system:
  \[
    \sopr{C}(\dop_A + \dop_A', \dop_B)
    = \sopr{C}(\dop_A, \dop_B) + \sopr{C}(\dop_A', \dop_B)
  \]
  Equivalence classes are called tensors in $\liou_A \tp \liou_B$

  For $\dop_A \tp \dop_B$ to be normalized,
  \begin{align*}
    \ip{\dop_A \tp \dop_B}{\dop_A \tp \dop_B}
    &= \ip{\dop_A}{\dop_A}\ip{\dop_B}{\dop_B} \\
    \shortintertext{This requires}
    (\dop_A \tp \dop_B)(\opr{A} \tp \opr{B})
    &= \dop_A\opr{A} \tp \dop_B\opr{B}
  \end{align*}
  Undo with $\tr_B\dop = \dop_A$.
\end{frame}

\begin{frame}{Time evolution}
  \textit{A priori}, anything can happen to $\dop$\footnote{$\dop$ cannot start
  entangled if it follows a \textsc{cptp} map}
  \[
    \dop \mapsto \sum_i \opr{B}_i \dop \opr{B}_i^\dag,
    \qq{where}
    \sum_i \opr{B}_i \opr{B}_i^\dag \leqslant \idopr
  \]
  Usually $\dop(t)$ depends only on $\dop$ now:
  \begin{align*}
    \dop(t)
    &= \sopr{V}_t\dop(0) \\
    \sopr{V}_{t + t'}
    &= \sopr{V}_t\sopr{V}_{t'}
  \end{align*}
  The \alert{Lindblad equation} generates $\sopr{V}_t$ that are \textsc{cptp}
  \[
    \dot{\dop}
    = -\im\comm{\ham}{\dop}
    + \sum_i \gamma_i \qty(%
    \opr{J}_i\dop\opr{J}_i^\dag
    - \frac{1}{2}\acomm{\opr{J}_i^\dag\opr{J}_i}{\dop}
    )
  \]
\end{frame}

\begin{frame}{Weak-coupling approximations}
  System coupled to stationary bath:
  \begin{align*}
    \ham
    &= \ham_S \tp \idopr + \idopr \tp \ham_B + \ham_I \\
    \ham_I
    &= \sum_i \opr{A}_i \tp \opr{B}_i
  \end{align*}
  Lindblad equation for $\dop_S(t) = \tr_B\dop(t)$ has
  \[
    \gamma_i(\omega)
    = \int_{-\infty}^\infty\dd{s} e^{\im\omega s}
    \ev{\opr{B}_i^\dag(t)\opr{B}_i(t - s)}
  \]
\end{frame}

\section{What effect does the bath have?}

\begin{frame}{Spins in a bosonic bath}
  For a single spin in a one-dimensional cavity:
  \begin{align*}
    \ham_B
    &= \sum_\omega \omega\, \opr{a}_\omega^\dag \opr{a}_\omega \\
    \ham_I
    &= \pauli_x \tp \sum_\omega\sqrt{J(\omega)}\qty(\opr{a}_\omega^\dag +
    \opr{a}_\omega)
  \end{align*}
  Only $J(\omega) \propto \omega$ for small $\omega$ produces interactions.

  With an upper cutoff, this is the \alert{Ohmic spectral density}
  \[
    J(\omega)
    \propto \frac{\omega}{1 + {(\omega/\Omega)}^2}
  \]
  Electric dipole interactions have the same coupling
\end{frame}

\section{How do we solve interacting spin systems?}

\begin{frame}{Solving the transverse-field Ising model}
  \[
    \ham
    = -\sum_{i \in \ZZ_N} \pauli_i^x\pauli_{i+1}^x
    - \lambda\sum_{i \in \ZZ_N} \pauli_i^z
  \]
  \begin{itemize}
    \item A solution by E.\ Leib et.\ al.\ requires deriving and solving a
      $N$-eigenvalue problem.

    \item A series of transformations provide a simpler route.
  \end{itemize}
  \[
    \pauli_i
    \to
    \opr{c}_i
    \to
    \opr{C}_k
    \to
    \opr{\eta}_k
  \]
\end{frame}

\begin{frame}{\color{gray}
    ${\usebeamercolor[fg]{frame title}\pauli_i \to
  \opr{c}_i} \to \opr{C}_k \to \opr{\eta}_k$}
  Fermions at each site with $\pauli_i^+$ and $\pauli_i^-$?
  \[
    \acomm{\pauli_i^+}{\pauli_j^-} \neq \delta_{ij} \idopr
  \]
  Actual fermions act like
  \begin{align*}
    \opr{c}_i\ket{\tq{n}}
    &= -{(-1)}^{\sum_{j<i} n_j}n_i\ket{\tq{n}_{i \leftarrow 0}} \\
    \ket{\tq{n}}
    &= \prod_i {\qty(\opr{c}_i^\dag)}^{n_i} \ket{\tq{0}}
  \end{align*}
  Same effect:
  \begin{align*}
    \opr{c}_i
    &= -\qty(\prod_{j<i} -\pauli_j^z) \pauli_i^- \\
    \ket{\tq{n}}
    &= \prod_i {\qty(\pauli_i^+)}^{n_i} \ket{\tq{0}}
  \end{align*}
\end{frame}

\begin{frame}{\color{gray}
    ${\usebeamercolor[fg]{frame title}\pauli_i \to
  \opr{c}_i} \to \opr{C}_k \to \opr{\eta}_k$}
  After Jordan-Wigner transform:
  \begin{align*}
    \ham
  &= \sum_i \qty(\opr{c}_i - \opr{c}_i^\dag) \qty(\opr{c}_{i+1}^\dag + \opr{c}_{i+1})
  - \lambda\sum_i 2\opr{c}_i^\dag\opr{c}_i
  + \lambda N \idopr \\
  &\color{gray}- \qty(\idopr - \prod_{j < N-1}\qty(2\opr{c}_j^\dag \opr{c}_j - \idopr))
  \qty(\opr{c}_{N-1} - \opr{c}_{N-1}^\dag) \qty(\opr{c}_0^\dag + \opr{c}_0)
  \end{align*}
  $\ham$ is translation invariant without the \textcolor{gray}{\bfseries
  boundary term}.
\end{frame}

\begin{frame}{\color{gray}
    $\pauli_i \to
    {\usebeamercolor[fg]{frame title}\opr{c}_i \to \opr{C}_k}
  \to \opr{\eta}_k$}
  Fourier transform:
  \[
    \opr{c}_i
    = \frac{1}{\sqrt{N}}\sum_k e^{\im ki} \opr{C}_k
  \]
  Now the Hamiltonian is
  \begin{align*}
    \ham
  &= -\sum_k \qty(\lambda + \cos k)
  \qty(%
  \opr{C}_k^\dag \opr{C}_k
  - \opr{C}_{-k} \opr{C}_{-k}^\dag
  )
  + \sum_k \im\sin k
  \qty(%
  \opr{C}_{-k}\opr{C}_k
  - \opr{C}_k^\dag\opr{C}_{-k}^\dag
  ) \\
  &= \sum_k \vq{v}_k^\dag \mq{H}_k \vq{v}_k,
  \end{align*}
  where
  \begin{align*}
    \mq{H}_k
  &= \bmqty{%
    -(\lambda + \cos k) & -\im\sin k \\
    \im\sin k & \lambda + \cos k
  } \\
  \vq{v}_k
  &= \bmqty{\opr{C}_k \\ \opr{C}_{-k}^\dag}
  \end{align*}
\end{frame}

\begin{frame}{\color{gray}
    $\pauli_i \to \opr{c}_i \to {\usebeamercolor[fg]{frame title}\opr{C}_k \to
  \opr{\eta}_k}$}
  Diagonalize the quadratic form:
  \begin{align*}
    E_k
    &= \sqrt{\lambda^2 + 2\lambda\cos k + 1}
    \\
    \opr{\eta}_k
    &= \frac{\im\sin k}{\sqrt{2E_k\qty(E_k + \cos k + \lambda)}}
    \opr{C}_k
    + \sqrt{\frac{E_k + \cos k + \lambda}{2E_k}}
    \opr{C}^\dag_{-k}
  \end{align*}
  Now the Hamiltonian is
  \begin{align*}
    \ham
    &= \sum_k
    \bmqty{\opr{\eta}_k^\dag & \opr{\eta}_{-k}}
    \bmqty{E_k & 0 \\ 0 & -E_k}
    \bmqty{\opr{\eta}_k \\ \opr{\eta}_{-k}^\dag} \\
               &= \sum_k 2E_k\opr{\eta}_k^\dag\opr{\eta}_k
               - \idopr\sum_k E_k
  \end{align*}
  $\opr{c}_i \to \opr{\eta}_k$ is unitary, so we have \alert{independent
  fermions}
\end{frame}

\section{How do we compute the jump operators?}

\begin{frame}{Computing jump operators}
  Jump operators are \alert{superoperators} of the Hamiltonian:
  \[
    \comm{\ham}{\opr{J}_i(\omega)}
    = \omega\opr{J}_i(\omega),
  \]
  where
  \begin{align*}
    \opr{J}_i(\omega)
  &= \sum_{E_2 - E_1 = \omega} \opr{P}(E_1)\pauli_i^x\opr{P}(E_2) \\
  \opr{P}(E)
  &= \sum_{\tq{n} \mathbin{:} E = \sum_k E_k n_k} \op{\tq{n}}
  \end{align*}
  Do inverse transformations \alert{$\opr{\eta}_k \to \pauli_i$} for $\pauli_i^x$
\end{frame}

\begin{frame}
  \centering
  \includegraphics[width=\linewidth]{time-evolution}
\end{frame}

\section{What next?}

\section*{References}

\begin{frame}{References}
  \nocite{opensys,liebTwoSolubleModels1961,pfeutyOnedimensionalIsingModel1970}
  \printbibliography%
\end{frame}

\end{document}

