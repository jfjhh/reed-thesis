\documentclass[12pt,c]{beamer}

\usepackage[normalem]{ulem}
\usepackage{multicol}
\usepackage{wasysym}
\usepackage{multicol}

\usepackage{mathtools} % Load before newtxmath
\mathtoolsset{%
  mathic,
}
\usepackage[%
math-style=ISO,
bold-style=ISO,
sans-style=italic,
nabla=upright,
partial=upright,
]{unicode-math}


\definecolor{rubric}{rgb}{0.7,0.05,0.0}

\usecolortheme[snowy]{owl}
\setbeamercolor{title}{fg=rubric}
% \setbeamerfont{frametitle}{shape=\itshape}
\setbeamercolor{frametitle}{bg=white}
\setbeamertemplate{frametitle}[default]%[center]
\setbeamertemplate{caption}{\insertcaption}
% \setbeamertemplate{caption}[numbered]
% \setbeamertemplate{caption label separator}{: }
\setbeamercolor{caption name}{fg=normal text.fg}
\setbeamercolor{section in toc}{fg=normal text.fg}
\setbeamercolor{footnote}{fg=gray}
\setbeamercolor{alerted text}{fg=rubric}
\setbeamercolor*{bibliography item}{fg=normal text.fg}
\setbeamercolor*{bibliography entry author}{fg=normal text.fg}
\setbeamercolor*{bibliography entry title}{fg=rubric}
\setbeamercolor*{bibliography entry location}{fg=normal text.fg}
\setbeamercolor*{bibliography entry note}{fg=normal text.fg}
\addtobeamertemplate{navigation symbols}{}{%
    \usebeamerfont{footline}%
    \color{normal text.fg}
    \hspace{1em}%
    \insertframenumber\,/\,\inserttotalframenumber%
}

\AtBeginSection[]{%
  \begin{frame}
  \vfill
  \centering
  \begin{beamercolorbox}[sep=8pt,center,shadow=true,rounded=true]{title}
    \usebeamerfont{title}\insertsectionhead\par%
  \end{beamercolorbox}
  \vfill
  \end{frame}
}

\usepackage[sorting=none,doi=false,url=false,isbn=false]{biblatex}
\addbibresource{references.bib}
\renewcommand*{\bibfont}{\footnotesize}

% Fonts
\usefonttheme[stillsansserifsmall]{serif} % So beamer won't change fonts
% \usefonttheme{default}
\usepackage[semibold]{libertinus-otf}
% \usepackage[full]{textcomp} % to get the right copyright, etc.
% \usepackage[T1]{fontenc} % LY1 also works
% \setmainfont[Numbers={OldStyle,Proportional}]{fbb}
% \setsansfont[Scale=0.95]{Cabin}
% \usepackage[supstfm=fbb-Regular-sup-t1]{superiors}
\setmainfont[Numbers={OldStyle,Proportional}]{Libertinus Serif} % We need lots of unicode, like ⊗
\usepackage[supstfm=libertinesups,supscaled=1.2,raised=-.13em]{superiors}
\setmonofont[Scale=MatchLowercase]{JuliaMono} % We need unicode
\usepackage[cal=cm,bb=boondox,frak=boondox]{mathalfa}

\newcommand{\spacedsc}{\scshape\addfontfeatures{LetterSpace=5}}

% Use microtype if available
\IfFileExists{microtype.sty}{\usepackage{microtype}}{}
\hypersetup{%
  pdfborder={0 0 0},
  breaklinks=true,
}
\urlstyle{same} % don't use monospace font for urls

% \setlength{\parindent}{0pt}
% \setlength{\parskip}{6pt plus 2pt minus 1pt}
% \setlength{\emergencystretch}{3em}  % prevent overfull lines
\providecommand{\tightlist}{%
  \setlength{\itemsep}{0pt}\setlength{\parskip}{0pt}}
\setcounter{secnumdepth}{0}

\usepackage{physics}
\usepackage{siunitx}
\usepackage{pgfplots}
\usepackage{tikz}
\usetikzlibrary{arrows}
\tikzset{% Animate with beamer overlays
  invisible/.style={opacity=0},
  visible on/.style={alt={#1{}{invisible}}},
  alt/.code args={<#1>#2#3}{%
    \alt<#1>{\pgfkeysalso{#2}}{\pgfkeysalso{#3}} % \pgfkeysalso doesn't change the path
  },
}

\graphicspath{{./figs/}}

%% Math packages
\usepackage{physics}
\usepackage{siunitx}


%% Math commands

% Semantic commands
\newcommand{\defeq}{\equiv}
\newcommand{\iso}{\cong}
\newcommand{\by}[1]{\qq{#1}}
\newcommand{\byeq}[1]{\by{\cref{#1}}}

% Variant symbols
\renewcommand{\le}{\leqslant}
\renewcommand{\ge}{\geqslant}
\renewcommand\phi\varphi%
\renewcommand\epsilon\varepsilon%

% Common algebraic objects
\newcommand\ZZ{\mathbb{Z}}
\newcommand\QQ{\mathbb{Q}}
\newcommand\CC{\mathbb{C}}
\newcommand\RR{\mathbb{R}}

% Named operations and special constants
\newcommand\im{\mathrm{i}\mkern1mu}
% \newcommand\im{i\mkern1mu}

% Operators
\newcommand\tp\otimes%
\newcommand\opr\symsf%
\newcommand\vecopr\symbfsf%
% \newcommand{\opr}[1]{\mathnormal{#1}}%
\newcommand\idopr{\opr{1}}
\newcommand\zopr{\opr{0}}
\newcommand\sopr\mathcal%
\newcommand\idsopr{\mathbb{1}}
\newcommand\zsopr{\mathbb{0}}
\newcommand\herm{\operatorname{\mathrm{He}}}
\newcommand\sgn{\operatorname{\mathrm{sgn}}}
\newcommand\sinc{\operatorname{\mathrm{sinc}}}
\newcommand\atanh{\operatorname{\mathrm{atanh}}}
\newcommand\acoth{\operatorname{\mathrm{acoth}}}

% Categories
\newcommand\cat\symsf%

% Linear algebra
\newcommand\tq[1]{\symbf{#1}} % Tuples
\newcommand\vq[1]{\symbf{#1}} % Vectors
\providecommand\vu{}
\renewcommand\vu[1]{\hat{\vq{#1}}} % Unit vectors
\newcommand\mq[1]{\symbf{#1}} % Matrices (not transformations)
\newcommand\trans{{\symsfup{T}}} % this is the one case when we want it upright

% Quantum mechanics
\newcommand\hilb{\mathcal{H}}
\newcommand\liou{\mathcal{L}}
\newcommand\bdd{\mathcal{B}}
\newcommand\ham{\opr{H}}
\newcommand\dop{\opr{ρ}}
\newcommand\pauli{\opr{σ}}
\newcommand\paulivec{\vecopr{σ}}
\newcommand\spin{\opr{S}}
\newcommand\spinvec{\vecopr{S}}
\newcommand\tprod\otimes%
\newcommand\lprod\ip%
% \newcommand\lprod[2]{\qty(#1,#2)}
\newcommand\ensavg[2][\dop]{\ev{#2}_{\mkern-1.5mu{#1}}}
\newcommand\sensavg[2][\dop]{\ev*{#2}_{\mkern-1.5mu{#1}}}



\renewcommand\footnoterule{}
\renewcommand\thefootnote{\textcolor{gray}{\arabic{footnote}}}
% \setbeamertemplate{footnote}{%
%   \scriptsize\noindent\raggedright$^*$\insertfootnotetext\par%
% }

\title{Relaxation for an Open System of Interacting Spins}
\author{Alex Striff \\ Advisor: Darrell Schroeter}
\institute{\textit{Reed College \\ Thesis Oral Defense}}
\date{May 4, 2021}

\begin{document}

{%
  \beamertemplatenavigationsymbolsempty%
  \begin{frame}
    \begin{tikzpicture}[remember picture,overlay,opacity=0.125]
      \node[at=(current page.center)] {%
          % \includegraphics[keepaspectratio,
          % width=\paperwidth,
          % height=\paperheight]{spins}
        };
    \end{tikzpicture}
    \titlepage%
  \end{frame}
}

\begin{frame}{Outline}
  \tableofcontents
\end{frame}


\section{Why care about thermalization and relaxation rates?}

\begin{frame}{We know little information about large systems}
  \begin{multicols}{2}
    \begin{figure}[h]
      \centering
      % http://marcmmw.freeshell.org/esp/fisica/intercambio_de_energia.html
      \includegraphics[width=0.5\linewidth]{hard-gas}
      \caption{A hard-sphere gas}\label{fig:hard-gas}
    \end{figure}
    \begin{figure}[h]
      \centering
      % https://tex.stackexchange.com/a/155242
      \includegraphics[width=0.5\linewidth]{diamond}
      \caption{Diamond}\label{fig:diamond}
    \end{figure}
    \vfill\null\columnbreak%
    Thermodynamic quantities:
    \begin{itemize}
      \item Temperature
      \item Energy
      \item Entropy
      \item Heat capacity
      \item Magnetization
    \end{itemize}
    \vspace{\baselineskip}
    Occam's razor: Assign the most uncertain state with known $\ev{E}$ (maximum
    entropy)
    \color{gray}
    \[
      \dop
      = \frac{e^{-\beta\ham}}{\tr e^{-\beta\ham}}
    \]
  \end{multicols}
\end{frame}

\begin{frame}{(The eigenstate thermalization hypothesis)}
  \begin{figure}[h]
    \centering
    % http://marcmmw.freeshell.org/esp/fisica/intercambio_de_energia.html
    \includegraphics[width=0.5\linewidth]{hard-gas}
    \caption{A hard-sphere gas}
  \end{figure}
\end{frame}

\begin{frame}{System-environment interactions}
  \begin{figure}[h]
    \centering
    % https://tex.stackexchange.com/a/155242
    \includegraphics[width=0.3333\linewidth]{diamond}
    \caption{Diamond}
  \end{figure}
  \begin{center}
    $
    \vcenter{\hbox{\includegraphics[width=0.3333\textwidth]{spins}}}
    \overset{\ham_I}{\longleftrightarrow}
    \quad
    \vcenter{\hbox{\includegraphics[width=0.3333\textwidth]{thermal-bath}}}
    $
  \end{center}
\end{frame}

\begin{frame}{Limitations of quantum devices}
  \begin{multicols}{2}
    \begin{figure}[h]
      \centering
      \includegraphics[width=0.85\linewidth]{nv-probe}
      \caption{Scanning NV magnetometer}
    \end{figure}
    \vfill\null\columnbreak%
    \begin{figure}[h]
      \centering
      \includegraphics[width=0.85\linewidth]{ibm-q}
      \caption{IBM Quantum computer}
    \end{figure}
  \end{multicols}
  \begin{figure}[h]
    \centering
    \vspace{-2\baselineskip}
    \includegraphics[width=0.85\linewidth]{nv-coherence}
  \end{figure}
\end{frame}


\section{Review of open quantum systems}

\begin{frame}{The quantum state}
  A \alert{quantum state} assigns probabilities to outcomes.
  \begin{center}
    \includegraphics[width=0.3333\textwidth]{stern-gerlach}
  \end{center}
  $\ket{\psi} \in \hilb$ does not assign even odds for all orientations.
  \frownie%
  
  How to represent half-$\ket{\uparrow}$ and half-$\ket{\downarrow}$?
  \begin{align*}
    \dop
    &= \frac{1}{2}\op{\uparrow} + \frac{1}{2}\op{\downarrow} \\
    &= \sum_i p_i \op{\psi_i}
  \end{align*}
  A \alert{density operator} $\dop$ represents a general quantum state.
\end{frame}

\begin{frame}{Weak coupling to a thermal bath}
  \begin{center}
    $
    \vcenter{\hbox{\includegraphics[width=0.3333\textwidth]{spins}}}
    \overset{\ham_I}{\longleftrightarrow}
    \quad
    \vcenter{\hbox{\includegraphics[width=0.3333\textwidth]{thermal-bath}}}
    $
  \end{center}
  \begin{itemize}
    \item Start with $\dop(0) = \dop_S(0) \tp \dop_B$.

    \item The combined system has unitary evolution from the Hamiltonian
      \[
        \ham
        = \ham_S + \ham_B + \ham_I.
      \]

    \item Then throw away the bath: $\dop_S(t) = \tr_B\dop(t)$.

    \item How to find $\dop_S(t)$ \alert{directly} from $\dop_S(0)$?
  \end{itemize}
\end{frame}

\begin{frame}{Reduced dynamics}
  Whatever it is, the result must also be a density operator.
  \[
    \ket{\Psi(t)} = \opr{U}(t) \ket{\Psi(0)}
    \quad\mapsto\quad
    \dop(t)
    = \sopr{V}(t)\dop(0)
  \]
  \pause%
  Usually $\dop(t + \dd{t})$ depends only on $\dop(t)$, not the past.
  \[
    \sopr{V}(t + \dd{t})\dop(0)
    = \sopr{V}(\dd{t})\sopr{V}(t)\dop(0)
  \]
  Then there is a generator of time evolution.
  \[
    \ket{\dot{\Psi}} = -\im\ham\ket{\Psi}
    \quad\mapsto\quad
    \dot{\dop} = \sopr{L}(\dop)
  \]
  \pause%
  Any \textcolor{gray}{\textsc{cptp}} $\sopr{L}$ may be put into
  \alert{Lindblad form}.
  \[
    \sopr{L}(\dop)
    = -\im\comm{\ham_S + \ham_{LS}}{\dop}
    + \sum_{i,\,\omega} \gamma_i(\omega) \qty(%
    \opr{A}_i(\omega)\dop\opr{A}_i^\dag(\omega)
    - \frac{1}{2}\acomm{\opr{A}_i^\dag(\omega)\opr{A}_i(\omega)}{\dop}
    )
  \]
\end{frame}

\begin{frame}{Example: Atomic emission}
  The eigenstates of hydrogen are \alert{not} stable.
  \textcolor{gray}{(E\&M vacuum)}
  \[
    \vcenter{\hbox{\includegraphics[width=0.5\textwidth]{hydrogen}}}
    \quad
    \vcenter{\hbox{\includegraphics[width=0.125\textwidth]{two-level.png}}}
    \quad
    \ham = \frac{\omega}{2}\,\pauli_z
  \]
  \begin{center}
    \includegraphics[width=0.75\textwidth]{populations}
  \end{center}
\end{frame}

\begin{frame}{Understanding the dissipator $\sopr{D}$}
  For the two-level system
  {\small
    \begin{align*}
      \sopr{D}(\dop)
      = \gamma_0(N + 1) \qty(\pauli^- \dop \pauli^+ -
      \frac{1}{2}\acomm{\pauli^+\pauli^-}{\dop})
      + \gamma_0 N \qty(\pauli^+ \dop \pauli^- -
      \frac{1}{2}\acomm{\pauli^-\pauli^+}{\dop})
    \end{align*}
  }
  \begin{multicols}{2}
    System evolution is a piecewise deterministic process (\textsc{pdp})
    {\color{gray} with $P(\text{jump}) \propto \gamma$}
    \[
      \ket{\psi}
      \mapsto
      \frac{\pauli^+\ket{\psi}}{\norm{\pauli^+\ket{\psi}}}
      \qq{or}
      \frac{\pauli^-\ket{\psi}}{\norm{\pauli^-\ket{\psi}}}
    \]
    \vspace{\baselineskip}
    {\color{gray}
      \[
        \ham_I
        = \vecopr{E} \vdot \vecopr{D}
        \approx -\frac{\Omega}{2}\qty(\pauli^+ + \pauli^-)
      \]
    }
    \vfill\null\columnbreak%
    \vfill
    \begin{figure}[b]
      \centering
      \includegraphics[width=\linewidth]{pdp}
      \caption{%
        A realization of the \textsc{pdp} for the driven two-level atom 
      }
    \end{figure}
  \end{multicols}
\end{frame}

\begin{frame}{The transverse-field Ising chain}
  \[
    \vcenter{\hbox{\includegraphics[width=0.2\linewidth]{tim}}}
    \qquad
    \ham
    = -J\sum_{i \in \ZZ_N} \qty(%
    \opr{S}_i^x\opr{S}_{i+1}^x
    + \frac{g}{2} \opr{S}_i^z
    )
  \]
  \begin{figure}[h]
    \centering
    \includegraphics[width=0.625\linewidth]{ising-modes}
    \caption{Elementary excitation spectra across field strength $g$}
  \end{figure}
\end{frame}


\section{Results}

\begin{frame}{Time evolution}
  \begin{center}
    \includegraphics[width=0.9\textwidth]{time-evolution-2}
  \end{center}
\end{frame}

\begin{frame}{Exponential fit}
  \begin{center}
    \includegraphics[width=0.9\textwidth]{exponential-fit-2}
  \end{center}
\end{frame}

\begin{frame}{Energy levels}
  \begin{center}
    \includegraphics[width=0.9\textwidth]{energy-levels-2}
  \end{center}
\end{frame}

\begin{frame}{Dissipator spectrum (spaghetti diagram)}
  \begin{center}
    \includegraphics[width=0.9\textwidth]{spin-spectrum-2}
  \end{center}
\end{frame}

\begin{frame}
  \begin{center}
    \includegraphics[keepaspectratio,width=0.95\paperwidth,height=0.95\paperheight]{%
    spin-relaxation-2}
  \end{center}
\end{frame}


\begin{frame}{Time evolution}
  \begin{center}
    \includegraphics[width=0.9\textwidth]{time-evolution-5}
  \end{center}
\end{frame}

\begin{frame}{Exponential fit}
  \begin{center}
    \includegraphics[width=0.9\textwidth]{exponential-fit-5}
  \end{center}
\end{frame}

\begin{frame}{Energy levels}
  \begin{center}
    \includegraphics[width=0.9\textwidth]{energy-levels-5}
  \end{center}
\end{frame}

\begin{frame}{Dissipator spectrum (spaghetti diagram)}
  \begin{center}
    \includegraphics[width=0.9\textwidth]{spin-spectrum-5}
  \end{center}
\end{frame}

\begin{frame}
  \begin{center}
    \includegraphics[keepaspectratio,width=0.95\paperwidth,height=0.95\paperheight]{%
    spin-relaxation-5}
  \end{center}
\end{frame}


\section*{References}

\begin{frame}{References}
  \nocite{opensys,liebTwoSolubleModels1961,pfeutyOnedimensionalIsingModel1970}
  \printbibliography%
\end{frame}

\end{document}

