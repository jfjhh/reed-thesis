\documentclass[../thesis.tex]{subfiles}
\begin{document}

\chapter{Open quantum systems}

\dropcap{W}{hat} happens after performing an operation on a physical system?
Our actions are uncertain, so the best we can do is to assign probabilities to
the possible outcomes. Quantum mechanics is a theory for determining the
probabilities of such outcomes. Physical theories model the relevant aspects of
phenomena by abstracting away unnecessary information. Newtonian mechanics
considers point masses while neglecting the material composition of bodies. This
enables a simpler description of motion. Just as Newtonian mechanics models
physical systems as ``Newtonian systems'' of point masses, quantum mechanics
models physical systems with \termalt{quantum state}{quantum states} which
represent only the probabilistic description of operation outcomes. A penny and
a quarter are different physical systems that correspond to the same ``quantum
system{,}'' as far as the outcomes of coin tosses are concerned.\footnote{%
  That is, in an ideal sense. Actual coins made from many atoms are not
  two-outcome systems. For example, the coin atoms could melt into a blob,
  rendering them unflippable.
}

\section{A sketch of quantum mechanics}

What follows is a sketch of quantum theory that helps motivate the mathematical
formalism we use later.

We are interested in performing operations that have consistent effects, for
otherwise we could make little sense of the world. Consider performing an
operation with $m$ outcomes on a quantum system. Given a particular outcome,
this operation is a \term{measurement} if repeating the operation gives the same
outcome with probability one.\footnote{%
  On this view, the result of a measurement is defined operationally. This
  avoids confusion with properties of \emph{physical} systems: an outcome with
  probability one is a different kind of thing than a physical property.
}

We may then characterize a quantum system by its \term{dimension} $N$, which is
the maximum number of outcomes distinguishable by a measurement.\footnote{%
  This supposes that such systems exist, which requires experimental
  verification. For example, Stern-Gerlach type experiments demonstrate the
  existence of a two-dimensional quantum system. Given what we know about
  systems like atoms in quantum physics, it is justified to assert the existence
  of quantum systems with any finite dimension. Infinite dimensional systems
  present some issues, but they may be regarded as being purely mathematical,
  since they can be well-approximated by finite dimensional systems as far as
  measurement outcomes are concerned. We usually consider only
  finite-dimensional systems in what follows.
}
Conversely, an operation with more than $N$ outcomes cannot result in a single
outcome with certainty.

\subsection{Entropy}

Since quantum states encode the probabilities of outcomes, it is relevant to
have a measure for how uncertain we are about which outcome will happen.

To do so, we must quantify the uncertainty expressed by a probability
distribution of outcomes. The most successful definition of this uncertainty due
to Shannon considers the \term{self-information} or \term{surprisal} $I$ of an
event with probability $p$~\cite{shannonMathematicalTheoryCommunication1948}.
This is a function $I$ that has the following intuitive properties:\footnote{%
  Technically, $I$ is defined for a random variable $X$ with support
  $\mathcal{X}$, so that $I:\mathcal{X} \to \RR$ is a function of the
  probability of an event: $I(x) = f(P(x))$.
}
\begin{enumerate}
  \item $I(1) = 0$: Certain events are uninformative or unsurprising.
  \item $I(p) < I(p')$ if $p' < p$: Less probable events are more surprising.
  \item $I(p) \ge 0$: The information learned from an event is nonnegative.
  \item $I(p,\, p') = I(p) + I(p')$: The information learned from independent
    events is the sum of the information learned from each event.
\end{enumerate}
The only functions with these properties are $I(p) = b\log p$ for $b < 0$.
Choosing $b$ amounts to choosing the base of the logarithm and the unit of
information. It is standard to use base two for which the unit of the
self-information
\begin{equation}
  I(p) = -\log p
  \label{eq:self-information}
\end{equation}
is the \term{bit}. The uncertainty expressed by a probability distribution of
outcomes is called the \term{Shannon entropy}, and is defined as the expected
surprisal of an outcome:
\begin{align}
  H
  &= \ev{I} \\
  &= -\sum_i p_i \log p_i,
  \label{eq:entropy}
\end{align}
where $0 \log 0 \coloneqq \lim_{p \to 0} p \log p = 0$. Thus an operation with a
certain outcome has $H = 0$. The most uncertain one can be in a $m$-outcome
operation is to have $p_i = 1/m$. In that case, $H = \log m$.

We now define the \term{von Neumann entropy} $S$ of a quantum state $\dop$ to be
the minimum entropy of any measurement of the state. If $S(\dop) = 0$, there is
a measurement with a definite outcome for $\dop$. Since the other outcomes are
excluded, $\dop$ is called a \term{pure state}, rather than a \term{mixed state}
which has probability spread out over more than one outcome.

\subsection{Hardy's postulates}\label{sec:hardy}

TODO: Rewrite as quantum logic.

Now that we have the notion of a pure state, we may indicate postulates of
quantum theory.

We expect that $K$ real numbers are needed to describe a quantum state $\dop$
and to predict the probabilities of outcomes. In quantum theory, we posulate the
following statements about $N$ and $K$:
\begin{enumerate}
  \item $K$ is a function of $N$, which we may select to be the smallest value
    consistent with the remaining postulates.
  \item A $N$-dimensional system constrained to only $M$ states distinguishable
    by measurement behaves like a system of dimension $M$.
  \item A composite system consisting of subsystems $A$ and $B$ satisfies $N =
    N_A N_B$ and $K = K_A K_B$.
\end{enumerate}
These statements apply equally well to classical probability theory. Quantum
theory is distinguished by the final postulate:
\begin{enumerate}
  \item[\textbf{4.}] There is a continuous and reversible transformation between
    any two pure states of a system.
\end{enumerate}
These postulates are enough to reproduce quantum mechanics as we know it. While
this has been just a sketch, Hardy gives a more precise description and
demonstrates agreement with the formalism to
follow~\cite{hardyQuantumTheoryFive2001}. In particular, Hardy shows that $K =
N^2$, which corresponds to considering a complex Hilbert space.

\section{The mathematical formalism}

Different mathematical formalisms for quantum mechanics differ in how they
represent states and operations, but they agree on the assignment of
probabilities to outcomes. Motivated by \cref{sec:hardy}, we now make the
following postulates:
\begin{post}\label{post:hilb}
  A quantum system is described by a separable complex Hilbert space
  $\hilb$.\footnote{%
    Since a quantum system is identified only by its dimension $d$, we may
    wonder which $d$-dimensional Hilbert space to assign. However, all finite
    $d$-dimensional (separable) complex Hilbert spaces are isometrically
    isomorphic to $\CC^d$, and all infinite-dimensional separable Hilbert spaces
    are isometrically isomorphic to $\ell^2$.
  }
\end{post}

\begin{post}\label{post:effects}
  An outcome corresponds to an \term{effect} $\opr{E}$, which is a self-adjoint
  operator on $\hilb$ such that $\zopr \le \opr{E} \le \idopr$.\footnotemark%
\end{post}
\footnotetext{%
  The notation $\opr{A} \le \opr{B}$ means that $\mel{v}{\opr{A}}{v} \le
  \mel{v}{\opr{B}}{v}$ for all $v \in \hilb$.
}

\begin{post}\label{post:probability_measure}
  A state corresponds to a \term{probability measure} $P$ on effects. That is:
  \begin{enumerate}
    \item $0 \le P(\opr{E}) \le 1$ for all effects $\opr{E}$,
    \item $P(\idopr) = 1$,
    \item $P(\opr{E}_1 + \opr{E}_2 + \cdots) = P(\opr{E}_1) + P(\opr{E}_2)
      + \cdots$ for any sequence of events with $\opr{E}_1 + \opr{E}_2 + \cdots
      \le \idopr$.
  \end{enumerate}
\end{post}

\begin{post}\label{post:convex}
  States form a convex set. If $\sum_i p_i = 1$, the convex sum of states
  $\qty{P_i}$ is defined by
  \begin{equation}
    \qty(\sum_i p_i P_i)(\opr{E})
    = \sum_i p_i P_i(\opr{E}).
    \label{eq:convex}
  \end{equation}
  Such a combination is known as an \term{ensemble}.
\end{post}

It is simple to prove that any such probability measure $P$ on an effect
$\opr{E}$ may be represented by the \term{Born rule}
\begin{equation}
  P(\opr{E})
  = \tr(\dop\opr{E}),
  \label{eq:effect_probability}
\end{equation}
where $\dop \ge \zopr$ is a self-adjoint operator known as a \term{density
  operator}~\cite{buschQuantumStatesGeneralized2003,
  buschOperationalQuantumPhysics1997}.\footnote{%
  The more specific case where effects are restricted to be projections $\op{v}$
  for $v \in \hilb$ is significantly harder, and is known as Gleason's
  theorem~\cite{gleasonMeasuresClosedSubspaces1975}.
}
\Cref{eq:effect_probability} implies that $\tr\dop = 1$ and that a convex sum of
states is represented by the same sum of density operators. The Born rule
uniquely identifies density operators with states, so we will use the term state
to refer to density operators from now on.

An \term{observable} result of an operation is described by an assignment of
each outcome $m$ to an effect $\opr{E}_m$, where $\sum_m \opr{E}_m = \idopr$.
Since effects are positive operators that determine the probabilities of each
outcome, such an observable is called a \term{positive operator valued measure}
(\textsc{povm}). The special case where the effects are projectors is called a
\term{projection valued measure} (\textsc{pvm}). We now describe how operations
change states:

\begin{post}\label{post:operation}
  An \term{operation} with outcome $m$ is described by a map $\sopr{O}_m$. The
  state $\dop$ after the operation becomes
  \begin{equation}
    \dop_m'
    = P{(m)}^{-1} \sopr{O}_m \dop.
    \label{eq:operation}
  \end{equation}
\end{post}
Inherent in \cref{post:operation} is the normalization condition $\tr\dop' = 1$,
which implies that $\tr\sopr{O}_m\dop = \tr(\dop \opr{E}_m)$, as well as the
requirement that $\sopr{O}_m\dop \ge \zopr$. If an operation is performed but
the outcome is unknown, we may assign the state
\begin{equation}
  \dop'
  = \sum_m P(m) \dop_m'
  = \sum_m \sopr{O}_m \dop,
  \label{eq:non-selective}
\end{equation}
so that an effect $\opr{E}$ has the expected probability $\tr(\dop' \opr{E}) =
\ev{\tr(\dop_m' \opr{E})}_m$.

The state of the ensemble $\dop = \sum_i p_i \dop_i$ after an operation with
outcome $m$ is
\begin{equation}
  \frac{\sopr{O}_m \dop}{\tr\sopr{O}_m \dop}
  = \sum_i P(i \,|\, m) \frac{\sopr{O}_m \dop_i}{\tr\sopr{O}_m \dop_i}
  \label{eq:ensemble-operation}
\end{equation}
By Bayes' theorem,
\begin{equation}
  P(i \,|\, m)
  = \frac{P(i) P(m \,|\, i)}{P(m)}
  = \frac{p_i \tr\sopr{O}_m \dop_i}{\tr\sopr{O}_m \dop}.
\end{equation}
Now \cref{eq:ensemble-operation} becomes
\begin{equation}
  \sopr{O}_m \qty(\sum_i p_i \dop_i)
  = \sum_i p_i \sopr{O}_m \dop_i.
  \label{eq:convex-operation}
\end{equation}
Thus operations are convex linear.


\subsection{Composite systems}

The success of physics lies in the apparent lack of causal connections between
phenomena separated in space and time. Different things are different, and the
actions of someone on the other side of the world have no immediate effect on an
experiment performed now. We then expect that some physical systems are composed
of a number of subsystems. The whole system may be affected as different parts
that each respond the same way as if nothing else was there.\footnote{%
  For example, we will later consider each spin in a spin chain to be a
  subsystem, since each spin may be separately affected. However, each of two
  electrons would not be a subsystem, since removal of one electron changes
  where the other electron is likely to be found.
}

Given a quantum system with Hilbert space $\hilb$, a decomposition into
\termalt{subsystem}{subsystems} is described by Hilbert spaces $\hilb_i$ with
$\prod_i\dim\hilb_i = \dim\hilb$ and maps $f_i$, $g_i$ such that
\begin{equation}
  \tr(\dop f_i(\opr{E}_i))
  = \tr(g_i(\dop) \opr{E}_i)
  \label{eq:trcompat}
\end{equation}
for any state $\dop$ on $\hilb$ and effect $\opr{E}_i$ on $\hilb_i$. The maps
$f_i$ lift an effect from the subsystem to the composite system, while the $g_i$
reduce a composite state to a subsystem state.

What are $f_i$ and $g_i$? We will first consider their action on states of
definite composition. Consider states $\dop_A$ and $\dop_B$ on $\hilb_A$ and
$\hilb_B$ where $\dim\hilb_A \dim\hilb_B = \dim\hilb$. How do we represent the
\term{product state} $\dop$, which satisfies $g_A(\dop) = \dop_A$ and $g_B(\dop)
= \dop_B$? Perhaps $\dop$ is just the pair $(\dop_A,\, \dop_B)$. By
\cref{post:convex}, a convex combination of such pairs is also a composite
state. However, suppose that $\dop_A = \sum_i \alpha_i \dop_i^A$ and $\dop_B =
\sum_j \beta_j \dop_j^B$. The ensemble of composites should be the same as the
composite of ensembles, which is the equivalence
\begin{equation}
  \qty(\sum_i \alpha_i \dop_i^A,\, \sum_j \beta_j \dop_j^B)
  \sim \sum_{ij} \alpha_i \beta_j \qty(\dop_i^A,\, \dop_j^B)
  \label{eq:tensorequiv}
\end{equation}
on the composite states. Thus the product state is not a pair, but an
equivalence class $[(\dop_A,\, \dop_B)]$ of $(\bdd(\hilb_A) \times
\bdd(\hilb_B))/{\sim}$. These classes are called \termalt{tensor}{tensors}. The
product state equivalence class is written with the \term{tensor product} as
$\dop_A \tp \dop_B$, and the whole space is called $\bdd(\hilb_A) \tp
\bdd(\hilb_B) \iso \bdd(\hilb)$.\footnote{%
  Is this Hilbert space separable as required by \cref{post:hilb}? Yes, but
  later we will consider a bath described by an infinite tensor product of
  harmonic oscillators. Such a Hilbert space is not separable, but the subspace
  that is physically relevant is separable. Since we will neglect high-frequency
  modes anyway, it also suffices to truncate the product once the energies are
  sufficiently high. For more discussion,
  see~\cite[pp.~84--87]{streaterPCTSpinStatistics1989}.
}
The same argument gives the \term{product effect} $\opr{E}_A \tp \opr{E}_B$. Yet
the same could be said for the operator $\dop_A\opr{E}_A$, which leads us to
define $(\dop_A \tp \dop_B)(\opr{E}_A \tp \opr{E}_B) = \dop_A\opr{E}_A \tp
\dop_B\opr{E}_B$.

What is $\tr(\opr{A} \tp \opr{B})$? Product effects represent independent
events. For the probabilities to multiply, we must have
\begin{equation}
  \tr(\opr{A} \tp \opr{B})
  = \tr\opr{A} \tr\opr{B}.
\end{equation}
Thus an effect $\opr{E}_A$ is lifted as $f_A(\opr{E}_A) = \opr{E}_A \tp \idopr$.

We now extend to non-product states, which are called \term{entangled} states.
Consider a map $f$ of the $f_i$ and the countable sum of events $\opr{E} =
\sum_a \opr{E}_a$. By \cref{eq:trcompat},
\begin{align}
  \tr\qty(\dop \sum_a f\qty(E_a))
  &= \sum_a \tr\qty(\dop f\qty(E_a)) \\
  &= \sum_a \tr\qty(g(\dop) E_a) \\
  &= \tr\qty(g(\dop) \sum_a E_a) \\
  &= \tr\qty(\dop f\qty(\sum_a E_a))
\end{align}
for all $\dop$. Thus $f$ is countably additive. If $\opr{E}_1 \le \opr{E}_2$,
then $0 \le \tr(g(\dop) \qty(\opr{E}_2 - \opr{E}_1)) = \tr(\dop f\qty(\opr{E}_2
- \opr{E}_1))$, so $f\qty(\opr{E}_1) \le f\qty(\opr{E}_2)$. It then follows that
the $f_i$ are convex.\footnote{%
  The argument is the same as that for the real linearity of a probability
  measure on effects given in~\cite{buschQuantumStatesGeneralized2003}.
}
By symmetry, the same argument shows that the $g_i$ are convex.

The map $g_A$ is known as a \term{partial trace}. The partial
trace over $B$ of $\dop_A \tp \dop_B$ is
\begin{equation}
  g_A(\dop)
  = \tr_B(\dop_A \tp \dop_B)
  \defeq \dop_A \tr\dop_B
  = \dop_A,
\end{equation}
and is extended linearly to combinations of product states.


\subsection{Closed dynamics}

Now that we have described the composition of quantum systems, how do they
change in time? We are often interested in the case where the information held
by the quantum state does not change. This is called a \term{closed} quantum
system. While all isolated physical systems are thought to correspond to closed
quantum systems, the converse is not true. For example, atomic spins subject to
external control from lasers are still closed systems.

The change in a closed system is the result of an operation $\sopr{O}$ with a
single, definite outcome. The operation $\sopr{O}$ is convex by
\cref{eq:convex-operation}. Since the information in the state $\dop$ cannot
change, $\sopr{O}$ must be invertible. Kadison's theorem\footnote{%
  Wigner's theorem on symmetries of pure states is a special
  case~\cite[p.~77]{morettiMathematicalFoundationsQuantum2016}. Many similar
  results hold, such as that only unitary transformations preserve the entropy
  (relative or
  not)~\cite{heEntropypreservingMapsQuantum2015,molnarMapsStatesPreserving2010}.
}
states that all such operations have the form
\begin{equation}
  \sopr{O}\dop
  = \opr{U}\dop\opr{U}^\dag,
  \label{eq:unitary-operation}
\end{equation}
where the operator $\opr{U}$ is either \term{unitary} ($\opr{U}^\dag\opr{U} =
\idopr$) or \term{antiunitary} ($\opr{U}^\dag\opr{U} =
\idopr$)~\cite{morettiMathematicalFoundationsQuantum2016,bengtssonGeometryQuantumStates2017}.

In contexts like quantum computation or control, all that matters is the state
before and after the operation. To consider the notion of an isolated physical
system, we must relate the quantum state to the physical time. We then have an
operator $\opr{U}(t)$ which gives the state
\begin{equation}
  \dop(t)
  = \opr{U}(t)\dop(0)\opr{U}^\dag(t)
  \label{eq:unitary-evolution}
\end{equation}
as a function of time. We expect that $\opr{U}(t)$ changes continuously and
satisfies
\begin{equation}
  \opr{U}(t + t')
  = \opr{U}(t) + \opr{U}(t')
\end{equation}
for real $t$, $t'$. Then since $\opr{U}(0) = \idopr$ is unitary, $\opr{U}(t)$ is
unitary. We would like a representation of $\opr{U}(t)$ that does not depend on
time. Stone's theorem~\cite{stoneOneParameterUnitaryGroups1932} is that there is
a self-adjoint operator $\ham$ such that
\begin{equation}
  \opr{U}(t)
  = e^{-\im\ham t}.
\end{equation}
This allows us to differentiate \cref{eq:unitary-evolution} to find
\begin{align}
  \dot{\dop}
  &= \dot{\opr{U}}(t)\dop(0)\opr{U}^\dag(t)
  + \opr{U}(t)\dop(0)\dot{\opr{U}}^\dag(t) \\
  &= -\im\ham\qty(\opr{U}(t)\dop(0)\opr{U}^\dag(t))
  + \qty(\opr{U}(t)\dop(0)\opr{U}^\dag(t))\im\ham \\
  &= -\im\comm{\ham}{\dop},
  \label{eq:liouville}
\end{align}
which is known as the \termalt{Liouville equation}{Liouville} or \term{von
Neumann equation}.

\subsection{Open dynamics}

Assume completely
positive~\cite{cuffaroDebateConcerningProper2013,pechukasReducedDynamicsNeed1994,shajiWhoAfraidNot2005}.

\end{document}

