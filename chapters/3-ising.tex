\documentclass[../thesis.tex]{subfiles}
\begin{document}

\chapter{The Ising model as an open quantum system}

\section{Solution of the transverse Ising model}

We consider the Hamiltonian
\begin{equation}
  \ham
  = -\sum_{i \in \ZZ_N} \pauli_i^x\pauli_{i+1}^x
  + \lambda\sum_{i \in \ZZ_N} \pauli_i^z
  \label{eq:transham}
\end{equation}
for the periodic transverse Ising chain with $N$ spins. We notice that the
operators
\begin{equation}
  \pauli_i^\pm
  = \frac{\pauli_i^x \pm \im\pauli_i^y}{2}
  \label{eq:paulipm}
\end{equation}
satisfy
\begin{equation}
  \pauli_i^z
  = 2\pauli_i^+\pauli_i^- - \idopr
\end{equation}
and have commutators
\begin{align}
  \comm{\pauli_i^+}{\pauli_j^-}
  &= \frac{1}{4}\comm{\pauli_i^x + \im\pauli_i^y}{\pauli_j^x - \im\pauli_j^y} \\
  &= \frac{1}{4} \qty(%
  \comm{\pauli_i^x}{\pauli_j^x} + \comm{\pauli_i^y}{\pauli_j^y}
  + \im\comm{\pauli_i^y}{\pauli_j^x}
  - \im\comm{\pauli_i^x}{\pauli_j^y}
  ) \\
  &= \delta_{ij}\pauli_i^z.
\end{align}
Thus their anticommutators are
\begin{align}
  \acomm{\pauli_i^+}{\pauli_j^-}
  &= 2\pauli_i^+\pauli_j^- - \comm{\pauli_i^+}{\pauli_j^-} \\
  &= 2\pauli_i^+\pauli_j^- - \delta_{ij}\pauli_i^z \\
  &= \delta_{ij}\idopr + 2\pauli_i^+\pauli_j^- (1 - \delta_{ij}).
\end{align}
It could be helpful to think of the $\pauli_i^\pm$ as fermion creation and
annihilation operators, but they do not anticommute at different sites.

How might we construct operators that satisfy the fermionic canonical
anticommutation relations (CARs) from the Pauli operators? Suppose we have such
operators $\opr{c}_i$. Given a tuple $\tq{n} = {\qty(n_i)}_{i\in\ZZ_N}$, we have the
corresponding states
\begin{equation}
  \ket{\tq{n}}
  = \prod_{i\in\ZZ_N} {\qty(\opr{c}_i^\dag)}^{n_i} \ket{\tq{0}},
\end{equation}
where $\ket{\tq{0}}$ denotes the vacuum state. It then follows that
\begin{align}
  \opr{c}_i\ket{\tq{n}}
  &= -n_i {(-1)}^{n_{<i}}\ket{\tq{n}_{i \leftarrow 0}}
  \label{eq:car-action} \\
  \opr{c}_i^\dag\ket{\tq{n}}
  &= -(1 - n_i){(-1)}^{n_{<i}}\ket{\tq{n}_{i \leftarrow 1}},
  \shortintertext{where}
  \tq{n}_{i \leftarrow m}
  &= \tq{n} \qq{with} n_i = m \\
  n_{<i}
  &= \sum_{j < i} n_j.
\end{align}
Then consider
\begin{align}
  \opr{c}_i
  = -\qty(\prod_{j<i} \pauli_j^z) \pauli_j^-
\end{align}
acting on the states
\begin{equation}
  \ket{\tq{n}}
  = \prod_{i \in \ZZ_N} {\qty(\pauli_i^+)}^{n_i} \ket{\tq{0}},
\end{equation}
where $\ket{\tq{0}} = \ket{\uparrow}^{\tp N}$ is the state with all $z$-spins
up, or all zero qubits.
\todo[inline]{Double-check up/down and make so that 1 bit is 1 occupied. I get
the z-eigenstate conventions confused.}
This gives the same result as \cref{eq:car-action}, so the $\opr{c}_i$ satisfy
the CARs.
This change from spin-1/2 sites to (non-local) fermions is known as the
\term{Jordan-Wigner transformation}.
We may then compute that the inverse transformations are
\begin{align}
  \pauli_i^+\pauli_i^-
  &= \opr{c}_i^\dag\opr{c}_i \\
  \pauli_i^z
  &= 2\opr{c}_i^\dag\opr{c}_i - \idopr \\
  \pauli_i^x
  &= -\qty(\prod_{j<i} \qty(2\opr{c}_j^\dag\opr{c}_j - \idopr))
  \qty(\opr{c}_i^\dag + \opr{c}_i) \\
  \pauli_i^y
  &= \im\qty(\prod_{j<i} \qty(2\opr{c}_j^\dag\opr{c}_j - \idopr))
  \qty(\opr{c}_i^\dag - \opr{c}_i).
\end{align}
While $\pauli_i^x$ remains complicated, the product $\pauli_i^x\pauli_{i+1}^x$
does not:
\begin{align}
  \pauli_i^x \pauli_{i+1}^x
  &= \qty(\prod_{j<i} \qty(2\opr{c}_j^\dag\opr{c}_j - \idopr))
  \qty(\opr{c}_i^\dag + \opr{c}_i)
  \qty(\prod_{j<i+1} \qty(2\opr{c}_j^\dag\opr{c}_j - \idopr))
  \qty(\opr{c}_{i+1}^\dag + \opr{c}_{i+1}) \\
  &= \qty(\opr{c}_i^\dag + \opr{c}_i)
  \qty(\idopr - 2\opr{c}_i^\dag\opr{c}_i)
  \qty(\opr{c}_{i+1}^\dag + \opr{c}_{i+1}) \\
  &= \qty(\opr{c}_i^\dag - \opr{c}_i) \qty(\opr{c}_{i+1}^\dag + \opr{c}_{i+1}).
  \label{eq:paulixx}
\end{align}

We may now perform the Jordan-Wigner transformation of \cref{eq:transham} to
obtain\todo[inline]{Where is the boundary term in Pfeuty~(2.4)? I think it is
gone since I have defined the $\opr{c}_i$ for $i \in \ZZ_N$, so that $\opr{c}_N$
is automatically correct and you do not need to add a correction term due to
applying the Jordan-Wigner transform with $i = N \in \ZZ$?}
\begin{align}
  \ham
  &= \sum_i \qty(\opr{c}_i - \opr{c}_i^\dag) \qty(\opr{c}_{i+1}^\dag + \opr{c}_{i+1})
  + \lambda\sum_i 2\opr{c}_i^\dag\opr{c}_i
  - \lambda N \idopr.
  \label{eq:cham}
\end{align}
We now Fourier transform with
\begin{align}
  \opr{c}_i
  &= \frac{1}{\sqrt{N}}\sum_k e^{\im ki} \opr{C}_k \\
  \shortintertext{and}
  \opr{c}_i^\dag
  &= \frac{1}{\sqrt{N}}\sum_k e^{-\im ki} \opr{C}_k^\dag,
  \shortintertext{where}
  k
  &= \frac{2\pi n}{N} - \frac{N - 1}{N}\pi \qc
  n \in \ZZ_N.
\end{align}
(I like this choice of $k$ since it is parity-symmetric. Inverting $k$ gives a
different $\opr{C}_k$ except at zero if $N$ is odd, unlike the usual convention for
Brillouin zones where the $\opr{C}_k$ at the boundary must be identified. The
drawback is that even $N$ has no $k = 0$.)
\begin{proof}
  Consider $N$ fermionic operators $\opr{c}_i$ and a $N \times N$ unitary matrix
  $\mq{U}$. We may change bases with
  \begin{equation}
    \opr{C}_k^\dag
    = \sum_i U_{ik} \opr{c}_i^\dag.
  \end{equation}
  Then
  \begin{align}
    \acomm{\opr{C}_k}{\opr{C}_{k'}^\dag}
    &= \sum_{ij} U_{ik}^*U_{jk'} \acomm{\opr{c}_i}{\opr{c}_j^\dag} \\
    &= \sum_i U_{ik}^*U_{ik'} \\
    &= {\qty(\mq{U}^\dag\mq{U})}_{kk'} \\
    &= \delta_{kk'},
  \end{align}
  and similar for the other fermionic (anti)-commutation relations.
\end{proof}
\begin{proof}
  For the Fourier transform,
  \begin{equation}
    F_{ik}
    = \frac{1}{\sqrt{N}} e^{\im ki}.
  \end{equation}
  We may then confirm that
  \begin{align}
    {\qty(\mq{F}^\dag\mq{F})}_{kk'}
    &= \sum_i \frac{1}{N} e^{\im (k' - k)i} \\
    &= \delta_{kk'}.
  \end{align}
  Thus the Fourier transform is unitary.
\end{proof}
Now since
\begin{equation}
  \frac{1}{N}\sum_{i\in\ZZ_N} e^{\im(k' - k) i}
  = \delta_{kk'},
\end{equation}
and also
\begin{align}
  \opr{C}_{-k}
  &= \opr{C}_k^* \\
  &= \frac{1}{\sqrt{N}} \sum_i e^{-\im (-k)i} \opr{c}_i \\
  &= \frac{1}{N}\sum_{ik'} e^{\im (k' + k)i} \opr{C}_{k'},
\end{align}
we have that
\begin{align}
  \sum_i \opr{c}_i^\dag\opr{c}_i
  &= \frac{1}{N}\sum_{ikk'} e^{\im (k' - k)i}
  \opr{C}_k^\dag \opr{C}_{k'} \\
  &= \sum_k \opr{C}_k^\dag \opr{C}_k,
  \\
  \sum_i \qty(\opr{c}_i^\dag\opr{c}_{i+1} + \opr{c}_{i+1}^\dag\opr{c}_i)
  &= \frac{1}{N}\sum_{ikk'} e^{\im (k' - k)i} \qty(%
  e^{\im k'} + e^{-\im k} 
  ) \opr{C}_k^\dag \opr{C}_{k'} \\
  &= \sum_k 2 \cos k\; \opr{C}_k^\dag \opr{C}_k,
  \\
  \sum_i \qty(\opr{c}_{i+1}\opr{c}_i + \opr{c}_i^\dag\opr{c}_{i+1}^\dag)
  &= \frac{1}{N}\sum_{ikk'} \qty(%
  e^{\im (k' + k)i} e^{\im k} \opr{C}_{k}\opr{C}_{k'}
  + e^{-\im (k' + k)i} e^{-\im k'} \opr{C}_{k}^\dag\opr{C}_{k'}^\dag
  ) \\
  &= \sum_k \qty(%
  e^{-\im k} \opr{C}_{-k} \opr{C}_k
  + e^{\im k} \opr{C}_k^\dag \opr{C}_{-k}^\dag
  ).
\end{align}
Thus \cref{eq:cham} is now
\begin{align}
  \ham
  &= -\sum_k 2 \cos k\; \opr{C}_k^\dag \opr{C}_k
  + \sum_k \qty(%
  e^{-\im k} \opr{C}_{-k} \opr{C}_k
  + e^{\im k} \opr{C}_k^\dag \opr{C}_{-k}^\dag
  )
  + \sum_k 2\lambda\opr{C}_k^\dag\opr{C}_k
  - \lambda N \idopr
  \\
  &= \sum_k \qty(\lambda - \cos k)
  \qty(%
  \opr{C}_k^\dag \opr{C}_k
  + \opr{C}_{-k}^\dag \opr{C}_{-k}
  )
  + \sum_k \im\sin k
  \qty(%
  \opr{C}_{-k}\opr{C}_k
  - \opr{C}_k^\dag\opr{C}_{-k}^\dag
  )
  - \lambda N \idopr
  \\
  &= \sum_k \qty(\lambda - \cos k)
  \qty(%
  \opr{C}_k^\dag \opr{C}_k
  - \opr{C}_{-k} \opr{C}_{-k}^\dag
  )
  + \sum_k \im\sin k
  \qty(%
  \opr{C}_{-k}\opr{C}_k
  - \opr{C}_k^\dag\opr{C}_{-k}^\dag
  )
  - \idopr\sum_k \cos k
  \\
  &= \sum_k \vq{v}_k^\dag \mq{H}_k \vq{v}_k
  - \idopr\sum_k \cos k,
  \label{eq:Cham} \\
  \shortintertext{where}
  \mq{H}_k
  &= \bmqty{%
  \lambda - \cos k & -\im\sin k \\
  \im\sin k & \cos k - \lambda
  } \\
  \shortintertext{and}
  \vq{v}_k
  &= \bmqty{\opr{C}_k \\ \opr{C}_{-k}^\dag}.
\end{align}
\todo[inline]{%
  Instead of halving the sums by using parity symmetry, we can use weighting
  functions of $k$:
  \begin{align}
    \sum_k \opr{C}_k^\dag \opr{C}_k
    &= \sum_k \qty(%
    \alpha_k\opr{C}_k^\dag\opr{C}_k
    + \beta_k\opr{C}_{-k}^\dag\opr{C}_{-k}
    ) \\
    &= \sum_k \qty(%
    \alpha_k\opr{C}_k^\dag\opr{C}_k
    - \beta_k\opr{C}_{-k}\opr{C}_{-k}^\dag
    )
    + \idopr\sum_k \beta_k,
  \end{align}
  where $\alpha_k + \beta_{-k} = 1$.

  Can this be used to make $\mq{H}_k$ singular, or otherwise help so that we
  ultimately have one term like $\opr{\eta}^\dag\opr{\eta}$?
}
Since the $\mq{H}_k$ are Hermitian, they may be diagonalized by a unitary
transformation of the $\vq{v}_k$. The $\mq{H}_k$ are traceless, so they have the
eigenvalues
\begin{align}
  E_k^\pm
  &= \pm\sqrt{-\det\mq{H}_k} \\
  &= \pm\sqrt{\lambda^2 - 2\lambda\cos k + \cos^2 k + \sin^2 k} \\
  &= \pm\sqrt{\lambda^2 - 2\lambda\cos k + 1}.
\end{align}
The eigenvectors are then
\begin{equation}
  \vq{q}_k^\pm
  = \bmqty{%
    -\im\sin k \\
    E_k^\pm - (\lambda - \cos k)
  },
\end{equation}
except if $k = 0$, in which case
\begin{align}
  \mq{H}_k
  &= \bmqty{%
    \lambda - 1 & 0 \\
    0 & 1 - \lambda,
  } \\
  \vq{q}_0^+
  &= \bmqty{1 \\ 0},
  \qand
  \vq{q}_0^-
  = \bmqty{0 \\ 1}.
\end{align}
To disregard this, we consider $N$ even going forward. To construct the unitary
transformation, we must normalize the $\vq{q}_k^\pm$. We find that
\begin{align}
  \norm{\vq{q}_k^\pm}^2
  &= {\qty(E_k^\pm - (\lambda - \cos k))}^2 + \sin^2 k \\
  &= {\qty(E_k^\pm)}^2 + \lambda^2 + \cos^2 k - 2\lambda\cos k - 2E_k^\pm(\lambda - \cos k) + 1 - \cos^2 k \\
  &= 2E_k^\pm \qty(E_k^\pm - (\lambda - \cos k)).
\end{align}
Now
\begin{align}
  \frac{{\qty(\vq{q}_k^\pm)}_1}{\norm{\vq{q}_k^\pm}}
  &= \frac{-\im\sin k}{\sqrt{2E_k^\pm \qty(E_k^\pm - (\lambda - \cos k))}} \\
  &= \frac{-\im\sin k}{\sqrt{2\abs{E_k^\pm} \qty(\abs{E_k^\pm} \mp (\lambda - \cos k))}}
  \\
  \shortintertext{and}
  %
  \frac{{\qty(\vq{q}_k^\pm)}_2}{\norm{\vq{q}_k^\pm}}
  &= \sqrt{\frac{E_k^\pm - (\lambda - \cos k)}{2E_k^\pm}} \\
  &= \sqrt{\frac{\abs{E_k^\pm} \mp (\lambda - \cos k)}{2\abs{E_k^\pm}}}
\end{align}
\begin{equation}
  \mq{U}_k^\dag
  = \bmqty{%
    {\qty(\hat{\vq{q}}_k^+)}^\dag \\
    {\qty(\hat{\vq{q}}_k^-)}^\dag
  }.
\end{equation}
Then with $E_k = \abs{E_k^\pm}$,
\begin{equation}
  \opr{\eta}_k^\pm
  = \frac{\im\sin k}{\sqrt{2E_k\qty(E_k \mp (\lambda - \cos k))}}
  \opr{C}_k
  + \sqrt{\frac{E_k \mp (\lambda - \cos k)}{2E_k}}
  \opr{C}^\dag_{-k},
\end{equation}
so that
\begin{align}
  E_k^\pm {\qty(\opr{\eta}_k^\pm)}^\dag\opr{\eta}_k^\pm
  &= E_k^\pm
  \qty(%
  \frac{-\im\sin k}{\sqrt{2E_k\qty(E_k \pm (\cos k - \lambda))}}
  \opr{C}_k^\dag
  + \sqrt{\frac{E_k \pm (\cos k - \lambda)}{2E_k}}
  \opr{C}_{-k}
  ) \\
  &\phantom{{}={E_k^\pm}}
  \qty(%
  \frac{\im\sin k}{\sqrt{2E_k\qty(E_k \pm (\cos k - \lambda))}}
  \opr{C}_k
  + \sqrt{\frac{E_k \pm (\cos k - \lambda)}{2E_k}}
  \opr{C}_{-k}^\dag
  )
  \\
  &=
  \frac{\sin^2 k}{2\qty(E_k^\pm + (\cos k - \lambda))}
  \opr{C}_k^\dag \opr{C}_k
  %
  + \frac{E_k^\pm + (\cos k - \lambda)}{2}
  \opr{C}_{-k} \opr{C}_{-k}^\dag \\
  %
  &- \frac{\im\sin k}{2}
  \opr{C}_k^\dag \opr{C}_{-k}^\dag
  %
  + \frac{\im\sin k}{2}
  \opr{C}_{-k} \opr{C}_k.
\end{align}
and
\begin{align}
  E_k^\pm \acomm{{\qty(\opr{\eta}_k^\pm)}^\dag}{\opr{\eta}_k^\pm}
  &= \frac{\sin^2 k}{2\qty(E_k^\pm + (\cos k - \lambda))}
  + \frac{E_k^\pm + (\cos k - \lambda)}{2} \\
  &= E_k^\pm \idopr
  \\
  \begin{split}
  E_k^\pm \acomm{{\qty(\opr{\eta}_k^\pm)}^\dag}{\opr{\eta}_k^\mp}
  &= \frac{\sin^2 k}{2\sqrt{E_k \pm (\cos k - \lambda)}
  \sqrt{E_k \mp (\cos k - \lambda)}} \\
  &+ \frac{\sqrt{E_k \pm (\cos k - \lambda)}\sqrt{E_k \mp (\cos k - \lambda)}}{2}
  \end{split}
  \\
  &= \abs{\sin k} \idopr
\end{align}
Thus \cref{eq:Cham} becomes
\begin{align}
  \ham
  &= \sum_k E_k^+{\qty(\opr{\eta}_k^+)}^\dag\opr{\eta}_k^+
  + \sum_k E_k^-{\qty(\opr{\eta}_k^-)}^\dag\opr{\eta}_k^-
  - \idopr\sum_k \cos k.
\end{align}
While the $\opr{\eta}_k^\pm$ are individually fermionic, they are not compatible
with each other.

\section{Eigenoperators for the transverse Ising Hamiltonian}

Pfeuty defines:
\begin{align}
  \lambda
  &= \frac{J}{2\Gamma}
  \label{eq:pflambda} \\
  %
  \opr{a}_i
  &= \opr{S}_{xi} - \im\opr{S}_{yi} 
  \label{eq:pfa} \\
  %
  \opr{a}_i^\dag
  &= \opr{S}_{xi} + \im\opr{S}_{yi} 
  \label{eq:pfadag} \\
  %
  \opr{c}_i
  &= \exp(\pi\im\sum_{j=1}^{i-1} \opr{a}_j^\dag\opr{a}_j)\opr{a}_i
  \label{eq:pfc} \\
  %
  \opr{c}_i^\dag
  &= \opr{a}_i^\dag\exp(-\pi\im\sum_{j=1}^{i-1} \opr{a}_j^\dag\opr{a}_j)
  \label{eq:pfcdag} \\
  %
  \opr{\eta}_k
  &= \sum_i \qty(\frac{\varphi_{ki} + \psi_{ki}}{2}\opr{c}_i
  + \frac{\varphi_{ki} - \psi_{ki}}{2}\opr{c}_i^\dag)
  \label{eq:pfeta} \\
  %
  \varphi_{ki}
  &= \sqrt{\frac{2}{N}}\begin{cases}
    \sin(ki) & k > 0 \\
    \cos(ki) & k \leq 0
  \end{cases}
  \label{eq:pfphi} \\
  %
  \psi_{ki}
  &= -\Lambda_k^{-1} \qty((1 + \lambda\cos k)\varphi_{ki}
  + (\lambda\sin k)\varphi_{-ki})
  \label{eq:pfpsi} \\
  %
  \Lambda_k^2
  &= 1 + \lambda^2 + 2\lambda\cos k
  \label{eq:pfLambda} \\
  %
  k
  &= \frac{2\pi m}{N} \qfor m = -\frac{N}{2},\, \ldots,\, \frac{N}{2} - 1,
  \qq{$N$ even.}
  \label{eq:pfk}
\end{align}
We would like to express the $\opr{S}_{xi}$ in terms of eigenoperators of the
system Hamiltonian
\begin{equation}
  \ham
  = \Gamma\sum_k \Lambda_k \opr{\eta}_k^\dag\opr{\eta}_k
  - \frac{\Gamma}{2}\sum_k \Lambda_k.
  \label{eq:pfH}
\end{equation}
We see that $\idopr$, $\opr{\eta}_k$, $\opr{\eta}_k^\dag$, and
$\opr{\eta}_k^\dag\opr{\eta}_k$ are all eigenoperators of $\ham$. Since these
operators form an orthonormal complete basis for the Liouville space, we then
seek out the coefficients $\ip{\opr{S}_{xi}}{\opr{\eta}_k}$.
\todo[inline]{I don't think this is true! (I.e.\ the $\opr{\eta}_k$ do not
satisfy the fermionic CARs.) All the inner products below are pointless.}

\begin{align}
  \psi_{ki}^2
  &= -\Lambda_k^{-2} \qty(%
  {(1 + \lambda\cos k)}^2\varphi_{ki}^2
  + (1 + \lambda\cos k)(\lambda\sin k)\varphi_{ki}\varphi_{-ki}
  + \lambda^2\sin^2 k\varphi_{-ki}^2
  ) \\
  &= -\Lambda_k^{-2} \qty(%
  \qty(1 + 2\lambda\cos k + \lambda^2\cos^2 k)\varphi_{ki}^2
  + (1 + \lambda\cos k)(\lambda\sin k)\varphi_{ki}\varphi_{-ki}
  + \lambda^2\sin^2 k\varphi_{-ki}^2
  )
\end{align}
For $k > 0$:
\begin{align}
  \psi_{ki}^2
  &= -\frac{2}{N}\sin^2(ki)
  + \frac{2}{N}\Lambda_k^{-2}\qty(
  (1 + \lambda\cos k)(\lambda\sin k)\sin^2(ki)
  - \lambda^2\sin^2 k\sin^2(ki)
  ) \\
  &= -\frac{2}{N}\sin^2(ki)
  + \frac{2}{N}\Lambda_k^{-2}\qty(
  (\Lambda_k^2 - \lambda\cos k - \lambda^2 + \lambda^2\sin^2 k)(\lambda\sin k)\sin^2(ki)
  - \lambda^2\sin^2 k\sin^2(ki)
  ) \\
  &= -\frac{2}{N}\sin^2(ki)
  + \frac{2}{N}\Lambda_k^{-2}\qty(
  (\Lambda_k^2 - \lambda\cos k - \lambda^2)(\lambda\sin k)\sin^2(ki)
  + (\lambda\sin k - 1)\lambda^2\sin^2 k\sin^2(ki)
  ) \\
\end{align}
For $k \leq 0$:
\begin{align}
  \psi_{ki}^2
  &= -\frac{2}{N}\cos^2(ki)
  - \frac{2}{N}\Lambda_k^{-2}\qty(
  (1 + \lambda\cos k)(\lambda\sin k)\cos^2(ki)
  + \lambda^2\sin^2 k\cos^2(ki)
  )
\end{align}

\begin{align}
  \begin{split}
    \acomm{\opr{\eta}_k}{\opr{\eta}_k^\dag}
  &=
  \sum_{ij}
  \frac{\varphi_{ki} + \psi_{ki}}{2}
  \frac{\varphi_{kj} + \psi_{kj}}{2}
  \acomm{\opr{c}_i}{\opr{c}_j}
  +
  \sum_{ij}
  \frac{\varphi_{ki} + \psi_{ki}}{2}
  \frac{\varphi_{kj} - \psi_{kj}}{2}
  \acomm{\opr{c}_i}{\opr{c}_j^\dag} \\
  &+
  \sum_{ij}
  \frac{\varphi_{ki} - \psi_{ki}}{2}
  \frac{\varphi_{kj} + \psi_{kj}}{2}
  \acomm{\opr{c}_i^\dag}{\opr{c}_j}
  +
  \sum_{ij}
  \frac{\varphi_{ki} - \psi_{ki}}{2}
  \frac{\varphi_{kj} - \psi_{kj}}{2}
  \acomm{\opr{c}_i^\dag}{\opr{c}_j^\dag}
  \end{split} \\
  %
  &= \frac{\idopr}{2} \sum_{i}
  \qty(\varphi_{ki} + \psi_{ki})\qty(\varphi_{ki} - \psi_{ki})
  \\
  %
  &= \frac{\idopr}{2} \sum_{i}
  \qty(\varphi_{ki}^2 - \psi_{ki}^2)
  \\
  %
  &= \idopr?
\end{align}
According to Mathematica, not in general.

\begin{align}
  \begin{split}
    \ip{\opr{\eta}_k}{\opr{\eta}_k^\dag}
  &=
  \sum_{ij}
  \frac{\varphi_{ki} + \psi_{ki}}{2}
  \frac{\varphi_{kj} + \psi_{kj}}{2}
  \ip{\opr{c}_i}{\opr{c}_j}
  +
  \sum_{ij}
  \frac{\varphi_{ki} + \psi_{ki}}{2}
  \frac{\varphi_{kj} - \psi_{kj}}{2}
  \ip{\opr{c}_i}{\opr{c}_j^\dag} \\
  &+
  \sum_{ij}
  \frac{\varphi_{ki} - \psi_{ki}}{2}
  \frac{\varphi_{kj} + \psi_{kj}}{2}
  \ip{\opr{c}_i^\dag}{\opr{c}_j}
  +
  \sum_{ij}
  \frac{\varphi_{ki} - \psi_{ki}}{2}
  \frac{\varphi_{kj} - \psi_{kj}}{2}
  \ip{\opr{c}_i^\dag}{\opr{c}_j^\dag}
  \end{split} \\
  &=
  \sum_i \qty(%
  \frac{\varphi_{ki} + \psi_{ki}}{2}
  \frac{\varphi_{ki} + \psi_{ki}}{2}
  +
  \frac{\varphi_{ki} - \psi_{ki}}{2}
  \frac{\varphi_{ki} - \psi_{ki}}{2}
  ) \\
  &=
  \frac{1}{2}
  \sum_i \qty(%
  \varphi_{ki}^2 + \psi_{ki}^2
  )
\end{align}

Note that the $\opr{a}_j^\dag\opr{a}_j$ commute for different $j$, so
\begin{align}
  \opr{c}_i
  = -\qty(\prod_{j=1}^{i-1}\exp(\opr{a}_j^\dag\opr{a}_j))\opr{a}_i \\
  = -\qty(\bigotimes_{j=1}^{i-1}\exp(\opr{a}_j^\dag\opr{a}_j))\opr{a}_i \\
\end{align}
Since $\tr(\opr{A} \tp \opr{B}) = \tr\opr{A}\tr\opr{B}$,
\begin{align}
  \ip{\opr{S}_{xi}}{\opr{c}_i}
  &= -\ip{\opr{S}_{xi}}{\qty(\bigotimes_{j=1}^{i-1}\exp(\opr{a}_j^\dag\opr{a}_j))\opr{a}_i} \\
  &= -\prod_{j=1}^{i-1}\ip{\opr{S}_{xi}\opr{a}_i^\dag}{\exp(\opr{a}_j^\dag\opr{a}_j)} \\
  &= -\prod_{j=1}^{i-1}\ip{\opr{S}_{xi}\opr{a}_i^\dag}{\idopr - 2\opr{a}_j^\dag\opr{a}_j} \\
  &= -\prod_{j=1}^{i-1}\ip{\opr{S}_{xi}\opr{a}_i^\dag}{\idopr - 2\opr{a}_j^\dag\opr{a}_j} \\
  &= -\prod_{j=1}^{i-1}\qty(%
  \ip{\opr{S}_{xi}}{\opr{a}_i}
  + 2\ip{\opr{S}_{xi}}{\opr{a}_j^\dag\opr{a}_j\opr{a}_i} 
  ) \\
\end{align}
OR (note that sign of $\im\pi$ in \cref{eq:pfc} is different in Leeds, but this
makes no difference)
\begin{align}
  \ip{\opr{S}_{xi}}{\opr{c}_j}
  &= \ip{\opr{a}_i + \opr{a}_i^\dag}{\exp(\pi\im\sum_{k=1}^{i-1}
  \opr{a}_k^\dag\opr{a}_k)\opr{a}_j} \\
  &= \ip{\qty(\opr{a}_i + \opr{a}_i^\dag)\opr{a}_j^\dag}{\bigotimes_{k=1}^{i-1}
  \qty(\idopr_2 - 2\opr{a}_k^\dag\opr{a}_k)} \\
  &= \delta_{ij}\ip{\opr{a}_i\opr{a}_i^\dag}{\bigotimes_{k=1}^{i-1}
  \qty(\idopr_2 - 2\opr{a}_k^\dag\opr{a}_k)}
  \\
  &= \delta_{ij}
  \\
  \ip{\opr{S}_{xi}}{\opr{c}_j^\dag}
  &= \delta_{ij}
  \\
  \ip{\opr{S}_{xi}}{\opr{c}_j^\dag\opr{c}_k}
  &= \ip{\opr{a}_i + \opr{a}_i^\dag}{\opr{a}_j^\dag\opr{a}_k} \\
  &= (\delta_{ij} + \delta_{jk} + \delta_{ki})(1 - \delta_{ij}\delta_{jk})
  \\
  \ip{\opr{S}_{xi}}{\opr{c}_j\opr{c}_k^\dag}
  &= (\delta_{ij} + \delta_{jk} + \delta_{ki})(1 - \delta_{ij}\delta_{jk}).
\end{align}
Thus
\begin{align}
  \ip{\opr{S}_{xi}}{\opr{\eta}_k}
  &= \frac{\varphi_{ki} + \psi_{ki}}{2} + \frac{\varphi_{ki} - \psi_{ki}}{2} \\
  &= \varphi_{ki}
  \\
  \ip{\opr{S}_{xi}}{\opr{\eta}_k^\dag}
  &= \varphi_{ki}
  \\
  \ip{\opr{S}_{xi}}{\opr{\eta}_k^\dag\opr{\eta}_k}
  &=
  \sum_{ab}
  \qty(%
  \frac{\varphi_{ka} + \psi_{ka}}{2}
  \opr{c}_a^\dag
  +
  \frac{\varphi_{ka} - \psi_{ka}}{2}
  \opr{c}_a
  )
  \qty(%
  \frac{\varphi_{kb} + \psi_{kb}}{2}
  \opr{c}_b
  +
  \frac{\varphi_{kb} - \psi_{kb}}{2}
  \opr{c}_b^\dag
  ) \\
  &=
  \sum_{ab}
  \qty(%
  \frac{\varphi_{ka} + \psi_{ka}}{2}
  \frac{\varphi_{kb} + \psi_{kb}}{2}
  +
  \frac{\varphi_{ka} - \psi_{ka}}{2}
  \frac{\varphi_{kb} - \psi_{kb}}{2}
  )
  (\delta_{ab} + \delta_{bi} + \delta_{ia})(1 - \delta_{ab}\delta_{bi})
  \\
  &=
  \sum_a
  \qty(%
  {\qty(\frac{\varphi_{ka} + \psi_{ka}}{2})}^2
  +
  {\qty(\frac{\varphi_{ka} - \psi_{ka}}{2})}^2
  )(1 - \delta_{ia}) \\
  &+
  2\sum_a
  \qty(%
  \frac{\varphi_{ka} + \psi_{ka}}{2}
  \frac{\varphi_{ki} + \psi_{ki}}{2}
  +
  \frac{\varphi_{ka} - \psi_{ka}}{2}
  \frac{\varphi_{ki} - \psi_{ki}}{2}
  )
  (1 - \delta_{ia})
  \\
\end{align}

\end{document}

