\documentclass[../thesis.tex]{subfiles}
\begin{document}

\chapter{Computing jump operators}

Here we go.\footnote{%
  A spaghetti diagram~\cite{bernasconiGiorgioBenedekExtraordinary2012}.
  \\
  \begin{center}
    \includegraphics[width=0.25\linewidth]{spaghetti}
  \end{center}
}

% Notebooks
% \notebook{transverse-ising}
% \notebook{jump-operators}

\notebook{sym-two-spin-jumps}

\foreach\n in {2,3,4,5}{%
  \begin{figure}[ht]
    \centering
    \includegraphics[width=0.85\linewidth]{time-evolution-\n}
    \caption{%
      Example time evolution for $N = \n$ spins.
    }\label{fig:time-evolution-\n}
  \end{figure}
  \begin{figure}[ht]
    \centering
    \includegraphics[width=0.85\linewidth]{spin-spectrum-\n}
    \caption{%
      Dissipation rate spaghetti diagram for $N = \n$ spins.
    }\label{fig:spin-spectrum-\n}
  \end{figure}
}

\foreach\n in {2,3,4,5}{%
  \begin{figure}[ht]
    \centering
    \includegraphics[width=\linewidth]{spin-relaxations-\n}
    \caption{%
      Single-spin relaxation rates in different temperature regimes for $N = \n$
      spins.
    }\label{fig:spin-relaxations-\n}
  \end{figure}
}

\end{document}

