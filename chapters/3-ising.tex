\documentclass[../thesis.tex]{subfiles}
\begin{document}

\chapter{The Ising model as an open quantum system}

\section{Solution of the transverse Ising model}

We consider the Hamiltonian
\begin{equation}
  \ham
  = -\sum_{i \in \ZZ_N} \pauli_i^x\pauli_{i+1}^x
  + \lambda\sum_{i \in \ZZ_N} \pauli_i^z
  \label{eq:transham}
\end{equation}
for the periodic transverse Ising chain with $N$ spins. We notice that the
operators
\begin{equation}
  \pauli_i^\pm
  = \frac{\pauli_i^x \pm \im\pauli_i^y}{2}
  \label{eq:paulipm}
\end{equation}
satisfy
\begin{equation}
  \pauli_i^z
  = 2\pauli_i^+\pauli_i^- - \idopr
\end{equation}
and have commutators
\begin{align}
  \comm{\pauli_i^+}{\pauli_j^-}
  &= \frac{1}{4}\comm{\pauli_i^x + \im\pauli_i^y}{\pauli_j^x - \im\pauli_j^y} \\
  &= \frac{1}{4} \qty(%
  \comm{\pauli_i^x}{\pauli_j^x} + \comm{\pauli_i^y}{\pauli_j^y}
  + \im\comm{\pauli_i^y}{\pauli_j^x}
  - \im\comm{\pauli_i^x}{\pauli_j^y}
  ) \\
  &= \delta_{ij}\pauli_i^z.
\end{align}
Thus their anticommutators are
\begin{align}
  \acomm{\pauli_i^+}{\pauli_j^-}
  &= 2\pauli_i^+\pauli_j^- - \comm{\pauli_i^+}{\pauli_j^-} \\
  &= 2\pauli_i^+\pauli_j^- - \delta_{ij}\pauli_i^z \\
  &= \delta_{ij}\idopr + 2\pauli_i^+\pauli_j^- (1 - \delta_{ij}).
\end{align}
It could be helpful to think of the $\pauli_i^\pm$ as fermion creation and
annihilation operators, but they do not anticommute at different sites.

How might we construct operators that satisfy the fermionic canonical
anticommutation relations (CARs) from the Pauli operators? Suppose we have such
operators $\opr{c}_i$. Given a tuple $\tq{n} = {\qty(n_i)}_{i\in\ZZ_N}$, we have the
corresponding states
\begin{equation}
  \ket{\tq{n}}
  = \prod_{i\in\ZZ_N} {\qty(\opr{c}_i^\dag)}^{n_i} \ket{\tq{0}},
\end{equation}
where $\ket{\tq{0}}$ denotes the vacuum state. It then follows that
\begin{align}
  \opr{c}_i\ket{\tq{n}}
  &= -n_i {(-1)}^{n_{<i}}\ket{\tq{n}_{i \leftarrow 0}}
  \label{eq:car-action} \\
  \opr{c}_i^\dag\ket{\tq{n}}
  &= -(1 - n_i){(-1)}^{n_{<i}}\ket{\tq{n}_{i \leftarrow 1}},
  \shortintertext{where}
  \tq{n}_{i \leftarrow m}
  &= \tq{n} \qq{with} n_i = m \\
  n_{<i}
  &= \sum_{j < i} n_j.
\end{align}
Then consider
\begin{align}
  \opr{c}_i
  = -\qty(\prod_{j<i} \pauli_j^z) \pauli_j^-
\end{align}
acting on the states
\begin{equation}
  \ket{\tq{n}}
  = \prod_{i \in \ZZ_N} {\qty(\pauli_i^+)}^{n_i} \ket{\tq{0}},
\end{equation}
where $\ket{\tq{0}} = \ket{\uparrow}^{\tp N}$ is the state with all $z$-spins
up, or all zero qubits.
\todo[inline]{Double-check up/down and make so that 1 bit is 1 occupied. I get
the z-eigenstate conventions confused.}
This gives the same result as \cref{eq:car-action}, so the $\opr{c}_i$ satisfy
the CARs.
This change from spin-1/2 sites to (non-local) fermions is known as the
\term{Jordan-Wigner transformation}.
We may then compute that the inverse transformations are
\begin{align}
  \pauli_i^+\pauli_i^-
  &= \opr{c}_i^\dag\opr{c}_i
  \label{eq:paulipm} \\
  \pauli_i^z
  &= 2\opr{c}_i^\dag\opr{c}_i - \idopr
  \label{eq:pauliz} \\
  \pauli_i^x
  &= -\qty(\prod_{j<i} \qty(2\opr{c}_j^\dag\opr{c}_j - \idopr))
  \qty(\opr{c}_i^\dag + \opr{c}_i)
  \label{eq:paulix} \\
  \pauli_i^y
  &= \im\qty(\prod_{j<i} \qty(2\opr{c}_j^\dag\opr{c}_j - \idopr))
  \qty(\opr{c}_i^\dag - \opr{c}_i).
  \label{eq:pauliy}
\end{align}
While $\pauli_i^x$ remains complicated, the product $\pauli_i^x\pauli_{i+1}^x$
does not:
\begin{align}
  \pauli_i^x \pauli_{i+1}^x
  &= \qty(\prod_{j<i} \qty(2\opr{c}_j^\dag\opr{c}_j - \idopr))
  \qty(\opr{c}_i^\dag + \opr{c}_i)
  \qty(\prod_{j<i+1} \qty(2\opr{c}_j^\dag\opr{c}_j - \idopr))
  \qty(\opr{c}_{i+1}^\dag + \opr{c}_{i+1}) \\
  &= \qty(\opr{c}_i^\dag + \opr{c}_i)
  \qty(\idopr - 2\opr{c}_i^\dag\opr{c}_i)
  \qty(\opr{c}_{i+1}^\dag + \opr{c}_{i+1}) \\
  &= \qty(\opr{c}_i^\dag - \opr{c}_i) \qty(\opr{c}_{i+1}^\dag + \opr{c}_{i+1}).
  \label{eq:paulixx}
\end{align}

We may now perform the Jordan-Wigner transformation of \cref{eq:transham} to
obtain\todo[inline]{Where is the boundary term in Pfeuty~(2.4)? I think it is
gone since I have defined the $\opr{c}_i$ for $i \in \ZZ_N$, so that $\opr{c}_N$
is automatically correct and you do not need to add a correction term due to
applying the Jordan-Wigner transform with $i = N \in \ZZ$?}
\begin{align}
  \ham
  &= \sum_i \qty(\opr{c}_i - \opr{c}_i^\dag) \qty(\opr{c}_{i+1}^\dag + \opr{c}_{i+1})
  + \lambda\sum_i 2\opr{c}_i^\dag\opr{c}_i
  - \lambda N \idopr.
  \label{eq:cham}
\end{align}
We now Fourier transform with
\begin{subequations}
\begin{align}
  \opr{c}_i
  &= \frac{1}{\sqrt{N}}\sum_k e^{\im ki} \opr{C}_k
  \label{eq:fourierc} \\
  \opr{C}_k
  &= \frac{1}{\sqrt{N}}\sum_i e^{-\im ki} \opr{c}_i
  \label{eq:fourierC} \\
  \shortintertext{and}
  \opr{c}_i^\dag
  &= \frac{1}{\sqrt{N}}\sum_k e^{-\im ki} \opr{C}_k^\dag
  \label{eq:fourierct} \\
  \opr{C}_k^\dag
  &= \frac{1}{\sqrt{N}}\sum_i e^{\im ki} \opr{c}_i^\dag
  \label{eq:fourierCt}.
\end{align}\label{eq:fourier}
\end{subequations}
\begin{proof}
  \begin{align}
    \opr{c}_0
    &= \frac{1}{\sqrt{N}}\sum_k \opr{C}_k \\
    \opr{c}_N
    &= \frac{1}{\sqrt{N}}\sum_k e^{\im kN} \opr{C}_k.
  \end{align}
  We then must require that
  \begin{equation}
    kN
    \equiv 0 \pmod{2\pi}
  \end{equation}
  \begin{equation}
    k
    = \frac{2\pi n}{N} - \frac{N - [\text{$N$ odd}]}{N}\pi \qc
    n \in \ZZ_N.
    \label{eq:fourierk}
  \end{equation}
  For $N$ odd, what is $C_{\pi}$?
  \begin{align}
    % \opr{C}_\pi
    \opr{C}_{\pi}
    &= \frac{1}{\sqrt{N}} \sum_i e^{-\im \pi i} \opr{c}_i.
  \end{align}
  Since $e^{-\im\pi i} = e^{\im\pi i}$, $\opr{C}_\pi = \opr{C}_{-\pi}$.
\end{proof}
\begin{proof}
  Consider $N$ fermionic operators $\opr{c}_i$ and a $N \times N$ unitary matrix
  $\mq{U}$. We may change bases with
  \begin{equation}
    \opr{C}_k^\dag
    = \sum_i U_{ik} \opr{c}_i^\dag.
  \end{equation}
  Then
  \begin{align}
    \acomm{\opr{C}_k}{\opr{C}_{k'}^\dag}
    &= \sum_{ij} U_{ik}^*U_{jk'} \acomm{\opr{c}_i}{\opr{c}_j^\dag} \\
    &= \sum_i U_{ik}^*U_{ik'} \\
    &= {\qty(\mq{U}^\dag\mq{U})}_{kk'} \\
    &= \delta_{kk'},
  \end{align}
  and similar for the other fermionic (anti)-commutation relations.
\end{proof}
\begin{proof}
  For the Fourier transform,
  \begin{equation}
    F_{ik}
    = \frac{1}{\sqrt{N}} e^{\im ki}.
  \end{equation}
  We may then confirm that
  \begin{align}
    {\qty(\mq{F}^\dag\mq{F})}_{kk'}
    &= \sum_i \frac{1}{N} e^{\im (k' - k)i} \\
    &= \delta_{kk'}.
  \end{align}
  Thus the Fourier transform is unitary.
\end{proof}
Now since
\begin{equation}
  \frac{1}{N}\sum_{i\in\ZZ_N} e^{\im(k' - k) i}
  = \delta_{kk'},
\end{equation}
and also
\begin{align}
  \opr{C}_{-k}
  &= \opr{C}_k^* \\
  &= \frac{1}{\sqrt{N}} \sum_i e^{-\im (-k)i} \opr{c}_i \\
  &= \frac{1}{N}\sum_{ik'} e^{\im (k' + k)i} \opr{C}_{k'},
\end{align}
we have that
\begin{align}
  \sum_i \opr{c}_i^\dag\opr{c}_i
  &= \frac{1}{N}\sum_{ikk'} e^{\im (k' - k)i}
  \opr{C}_k^\dag \opr{C}_{k'} \\
  &= \sum_k \opr{C}_k^\dag \opr{C}_k,
  \\
  \sum_i \qty(\opr{c}_i^\dag\opr{c}_{i+1} + \opr{c}_{i+1}^\dag\opr{c}_i)
  &= \frac{1}{N}\sum_{ikk'} e^{\im (k' - k)i} \qty(%
  e^{\im k'} + e^{-\im k} 
  ) \opr{C}_k^\dag \opr{C}_{k'} \\
  &= \sum_k 2 \cos k\; \opr{C}_k^\dag \opr{C}_k,
  \\
  \sum_i \qty(\opr{c}_{i+1}\opr{c}_i + \opr{c}_i^\dag\opr{c}_{i+1}^\dag)
  &= \frac{1}{N}\sum_{ikk'} \qty(%
  e^{\im (k' + k)i} e^{\im k} \opr{C}_{k}\opr{C}_{k'}
  + e^{-\im (k' + k)i} e^{-\im k'} \opr{C}_{k}^\dag\opr{C}_{k'}^\dag
  ) \\
  &= \sum_k \qty(%
  e^{-\im k} \opr{C}_{-k} \opr{C}_k
  + e^{\im k} \opr{C}_k^\dag \opr{C}_{-k}^\dag
  ).
\end{align}
Thus \cref{eq:cham} is now
\begin{align}
  \ham
  &= -\sum_k 2 \cos k\; \opr{C}_k^\dag \opr{C}_k
  + \sum_k \qty(%
  e^{-\im k} \opr{C}_{-k} \opr{C}_k
  + e^{\im k} \opr{C}_k^\dag \opr{C}_{-k}^\dag
  )
  + \sum_k 2\lambda\opr{C}_k^\dag\opr{C}_k
  - \lambda N \idopr
  \\
  &= \sum_k \qty(\lambda - \cos k)
  \qty(%
  \opr{C}_k^\dag \opr{C}_k
  + \opr{C}_{-k}^\dag \opr{C}_{-k}
  )
  + \sum_k \im\sin k
  \qty(%
  \opr{C}_{-k}\opr{C}_k
  - \opr{C}_k^\dag\opr{C}_{-k}^\dag
  )
  - \lambda N \idopr
  \\
  &= \sum_k \qty(\lambda - \cos k)
  \qty(%
  \opr{C}_k^\dag \opr{C}_k
  - \opr{C}_{-k} \opr{C}_{-k}^\dag
  )
  + \sum_k \im\sin k
  \qty(%
  \opr{C}_{-k}\opr{C}_k
  - \opr{C}_k^\dag\opr{C}_{-k}^\dag
  ) \\
  &= \sum_k \vq{v}_k^\dag \mq{H}_k \vq{v}_k,
  \label{eq:Cham} \\
  \shortintertext{where}
  \mq{H}_k
  &= \bmqty{%
  \lambda - \cos k & -\im\sin k \\
  \im\sin k & \cos k - \lambda
  }, \\
  \vq{v}_k
  &= \bmqty{\opr{C}_k \\ \opr{C}_{-k}^\dag},
\end{align}
and we have used that
\begin{equation}
  \sum_k \cos k = 0.
\end{equation}
Since the $\mq{H}_k$ are Hermitian, they may be diagonalized by a unitary
transformation of the $\vq{v}_k$. The $\mq{H}_k$ are traceless, so they have the
eigenvalues
\begin{align}
  E_k^\pm
  &= \pm\sqrt{-\det\mq{H}_k} \\
  &= \pm\sqrt{\lambda^2 - 2\lambda\cos k + \cos^2 k + \sin^2 k} \\
  &= \pm\sqrt{\lambda^2 - 2\lambda\cos k + 1}.
\end{align}
The eigenvectors are then
\begin{equation}
  \vq{q}_k^\pm
  = \bmqty{%
    -\im\sin k \\
    E_k^\pm - (\lambda - \cos k)
  },
\end{equation}
except if $k = 0$, in which case
\begin{align}
  \mq{H}_k
  &= \bmqty{%
    \lambda - 1 & 0 \\
    0 & 1 - \lambda,
  } \\
  \vq{q}_0^+
  &= \bmqty{1 \\ 0},
  \qand
  \vq{q}_0^-
  = \bmqty{0 \\ 1}.
\end{align}
To disregard this, we consider $N$ even going forward. To construct the unitary
transformation, we must normalize the $\vq{q}_k^\pm$. We find that
\begin{align}
  \norm{\vq{q}_k^\pm}^2
  &= {\qty(E_k^\pm - (\lambda - \cos k))}^2 + \sin^2 k \\
  &= {\qty(E_k^\pm)}^2 + \lambda^2 + \cos^2 k - 2\lambda\cos k - 2E_k^\pm(\lambda - \cos k) + 1 - \cos^2 k \\
  &= 2E_k^\pm \qty(E_k^\pm - (\lambda - \cos k)).
\end{align}
Now
\begin{align}
  \frac{{\qty(\vq{q}_k^\pm)}_1}{\norm{\vq{q}_k^\pm}}
  &= \frac{-\im\sin k}{\sqrt{2E_k^\pm \qty(E_k^\pm - (\lambda - \cos k))}} \\
  &= \frac{-\im\sin k}{\sqrt{2\abs{E_k^\pm} \qty(\abs{E_k^\pm} \mp (\lambda - \cos k))}}
  \\
  \shortintertext{and}
  %
  \frac{{\qty(\vq{q}_k^\pm)}_2}{\norm{\vq{q}_k^\pm}}
  &= \pm\sqrt{\frac{E_k^\pm - (\lambda - \cos k)}{2E_k^\pm}} \\
  &= \pm\sqrt{\frac{\abs{E_k^\pm} \mp (\lambda - \cos k)}{2\abs{E_k^\pm}}}
\end{align}
\begin{equation}
  \mq{U}_k^\dag
  = \bmqty{%
    {\qty(\hat{\vq{q}}_k^+)}^\dag \\
    {\qty(\hat{\vq{q}}_k^-)}^\dag
  }.
\end{equation}
Then with $E_k = \abs{E_k^\pm}$,
\begin{equation}
  \opr{\eta}_k^\pm
  = \frac{-\im\sin k}{\sqrt{2E_k\qty(E_k \pm (\cos k - \lambda))}}
  \opr{C}_k
  \pm \sqrt{\frac{E_k \pm (\cos k - \lambda)}{2E_k}}
  \opr{C}^\dag_{-k}
  \label{eq:etas}
\end{equation}
so that
\begin{align}
  \acomm{{\qty(\opr{\eta}_k^\pm)}^\dag}{\opr{\eta}_k^\pm}
  &= \frac{\sin^2 k}{2E_k\qty(E_k \pm (\cos k - \lambda))} \idopr
  + \frac{E_k \pm (\cos k - \lambda)}{2E_k} \idopr \\
  &= \idopr
  \\
  \begin{split}
  \acomm{{\qty(\opr{\eta}_k^\pm)}^\dag}{\opr{\eta}_k^\mp}
  &= \frac{\sin^2 k}{2E_k
    \sqrt{E_k \pm (\cos k - \lambda)}
  \sqrt{E_k \mp (\cos k - \lambda)}} \idopr \\
  &- \frac{\sqrt{E_k \pm (\cos k - \lambda)}\sqrt{E_k \mp (\cos k - \lambda)}}{%
  2 E_k} \idopr
  \end{split}
  \\
  &= \zopr.
\end{align}
Thus \cref{eq:Cham} becomes
\begin{equation}
  \ham
  = \sum_k E_k^+{\qty(\opr{\eta}_k^+)}^\dag\opr{\eta}_k^+
  + \sum_k E_k^-{\qty(\opr{\eta}_k^-)}^\dag\opr{\eta}_k^-.
  \label{eq:etasham}
\end{equation}
Since
\begin{equation}
  {\qty(\opr{\eta}_{-k}^-)}^\dag
  = \opr{\eta}_k^+
  \eqqcolon \opr{\eta}_k
\end{equation}
and $E_{-k}^\pm = E_k^\pm$, we may reduce \cref{eq:etasham} to
\begin{align}
  \ham
  &= \sum_k E_k\opr{\eta}_k^\dag \opr{\eta}_k
  - \sum_k E_k\qty(\idopr - \opr{\eta}_k^\dag \opr{\eta}_k)
  \\
  &= \sum_k 2E_k\opr{\eta}_k^\dag\opr{\eta}_k
  - \idopr\sum_k E_k.
  \label{eq:etaham}
\end{align}

The unitary transformation of the $\opr{C}_{\pm k}$ to obtain $\opr{\eta}_k^\pm$
is an instance of a fermionic \term{Bogoliubov transformation}:
\begin{subequations}
  \begin{align}
    \opr{C}_k
  &= u\opr{f}_k + v\opr{g}_k^\dag \\
  \shortintertext{and}
  \opr{C}_{-k}
  &= -v\opr{f}_k^\dag + u\opr{g}_k.
  \end{align}\label{eq:bogoliubov}
\end{subequations}
For these transformations to preserve the CARs,
\begin{align}
  \acomm{\opr{C}_k^\dag}{\opr{C}_k}
  &= \abs{u}^2 \acomm{\opr{f}_k^\dag}{\opr{f}_k}
  + \abs{v}^2 \acomm{\opr{g}_k}{\opr{g}_k^\dag}
  + u^* v \acomm{\opr{f}_k^\dag}{\opr{g}_k^\dag}
  + v^* u \acomm{\opr{g}_k}{\opr{f}_k} \\
  &= \qty(\abs{u}^2 + \abs{v}^2) \idopr,
\end{align}
so we must have
\begin{equation}
  \abs{u}^2 + \abs{v}^2 = 1.
\end{equation}
We may choose
% \begin{subequations}
  \begin{align}
    u &= e^{\im\phi_1}\cos\theta \\
    v &= e^{\im\phi_2}\sin\theta
  \end{align}\label{eq:uv-param}
% \end{subequations}
for real angles $\phi_1$, $\phi_2$, and $\theta$.

It remains to perform the inverse transformations to obtain the $\pauli_i^x$ in
terms of the $\opr{\eta}_k^\pm$. Shift of perspective: Instead of decomposing
the Hilbert space as the tensor product of spin operators, we decompose
it as the tensor product of fermionic operators, where the fermionic operators
are restricted to their subspaces in order to form an orthonormal basis for the
Liouville space. We may then recover $\pauli_i^x$ by the corresponding
isomorphism of Liouville spaces. We choose
\begin{equation}
  \opr{c}
  \qc
  \opr{c}^\dag
  \qc
  \opr{c}^\dag \opr{c},
  \qand
  \opr{c} \opr{c}^\dag
\end{equation}
as our basis, since they are both orthonormal and eigenoperators of
$\opr{c}^\dag \opr{c} \angle$ with respective eigenvalues
\[
  -1 \qc 1 \qc 2, \qand 0.
\]
We may then write
\begin{equation}
  \pauli_x
  = \sum_{ia} x_{ia} \opr{c}_{ia},
\end{equation}
where
\begin{equation}
  x_{ia}
  = \ip{\pauli_x}{\opr{c}_{ia}}
\end{equation}
and $\opr{c}_{ia}$ is restricted to subspace $i$. That is,
\begin{align}
  \opr{c}_{1a}
  &= \opr{c}_a \otimes \zopr^{\otimes(N-1)}, \\
  \shortintertext{not}
  \opr{c}_{1a}
  &= \opr{c}_a \otimes \idopr^{\otimes(N-1)}
\end{align}
as usual.

Combining \cref{eq:etas,eq:fourier} gives
\begin{align}
  \opr{\eta}_k
  &= \frac{-\im\sin k}{\sqrt{2E_k\qty(E_k + (\cos k - \lambda))}}
  \frac{1}{\sqrt{N}}\sum_i e^{-\im ki} \opr{c}_i \\
  &+ \sqrt{\frac{E_k + (\cos k - \lambda)}{2E_k}}
  \frac{1}{\sqrt{N}}\sum_i e^{-\im ki} \opr{c}_i^\dag.
  \\
  \opr{\eta}_k^\dag \opr{\eta}_k
  &= \frac{\sin^2 k}{2E_k\qty(E_k + (\cos k - \lambda))}
  \frac{1}{N}\sum_{ij} e^{\im k (i - j)} \opr{c}_i^\dag \opr{c}_j \\
  &+ \frac{E_k + (\cos k - \lambda)}{2E_k}
  \frac{1}{N}\sum_{ij} e^{\im k (i - j)} \opr{c}_i \opr{c}_j^\dag \\
  &+ \frac{\im\sin k}{2 E_k}
  \frac{1}{N}\sum_{ij} e^{\im k (i - j)}
  \qty(\opr{c}_i^\dag \opr{c}_j^\dag - \opr{c}_i \opr{c}_j)
\end{align}
\todo[inline]{Should the $\opr{\eta}_k$ match Pfeuty? If so, then hmm: these coefficients are
complex while Pfeuty's are real.}
Then using \cref{eq:paulix} as
\begin{equation}
  \pauli_i^x
  = -\qty(\prod_{j<i} \qty(\opr{c}_j^\dag\opr{c}_j - \opr{c}_j\opr{c}_j^\dag))
  \qty(\opr{c}_i^\dag + \opr{c}_i),
\end{equation}
\begin{align}
  \opr{\eta}_k
  &= \sum_{i} a_{ik} \opr{c}_i + b_{ik} \opr{c}_i^\dag.
  \\
  \opr{\eta}_k^\dag \opr{\eta}_k
  &= \sum_{ij} \qty(a_{ik}^* \opr{c}_i^\dag + b_{ik}^* \opr{c}_i)
  \qty(a_{jk} \opr{c}_j + b_{jk} \opr{c}_j^\dag) \\
  &= \sum_{ij} \qty(%
  a_{ik}^* a_{jk} \opr{c}_i^\dag \opr{c}_j
  + b_{ik}^* b_{jk} \opr{c}_i\opr{c}_j^\dag
  + a_{ik}^* b_{jk} \opr{c}_i^\dag \opr{c}_j^\dag
  + b_{ik}^* a_{jk} \opr{c}_i \opr{c}_j
  ) \\
  &= \sum_i \qty(%
  \abs{a_{ik}}^2 \opr{c}_i^\dag \opr{c}_i
  + \abs{b_{ik}}^2 \opr{c}_i \opr{c}_i^\dag
  ) \\
  &+ \sum_{i \neq j} \qty(%
  a_{ik}^* a_{jk} \opr{c}_i^\dag \opr{c}_j
  + b_{ik}^* b_{jk} \opr{c}_i\opr{c}_j^\dag
  + a_{ik}^* b_{jk} \opr{c}_i^\dag \opr{c}_j^\dag
  + b_{ik}^* a_{jk} \opr{c}_i \opr{c}_j
  ).
\end{align}
Note that for $\opr{c}_i$ and $\opr{d}_i$ from an orthonormal basis for $\liou_i$,
\begin{align}
  \ip{\bigotimes\nolimits_i a_i \opr{c}_i}{\bigotimes\nolimits_i b_i \opr{d}_i}
  &= \tr(\bigotimes\nolimits_i{a_i^* b_i\, \opr{c}_i^\dag \opr{d}_i}) \\
  &= \prod_i \tr(a_i^* b_i\, \opr{c}_i^\dag \opr{d}_i) \\
  &= \prod_i a_i^* b_i \ip{\opr{c}_i}{\opr{d}_i} \\
  &= \begin{cases}
    \prod_i a_i^* b_i & \opr{c}_i = \opr{d}_i \\
    0 & \text{otherwise.}
  \end{cases}
\end{align}
We compute that
\begin{align}
  \ip{\pauli_i^x}{\opr{\eta}_k}
  &= -\ip{\opr{c}_i^\dag + \opr{c}_i}{%
    \sum_{j} a_{jk} \opr{c}_j + b_{jk} \opr{c}_j^\dag
  } \\
  &= a_{ik} + b_{ik}
  \\
  \ip{\pauli_i^x}{\opr{\eta}_k^\dag}
  &= \ip{\opr{\eta}_k}{\pauli_i^x} \\
  &= a_{ik}^* + b_{ik}^*
  \\
  \ip{\pauli_i^x}{\opr{\eta}_k^\dag\opr{\eta}_k}
  &= 0 \by{since trace $i$ vanishes} \\
  \ip{\pauli_i^x}{\opr{\eta}_k\opr{\eta}_k^\dag}
  &= \ip{\opr{\eta}_k^\dag\opr{\eta}_k}{\pauli_i^x} \\
  &= 0.
\end{align}
Thus
\begin{align}
  \pauli_i^x
  &= \sum_k \opr{A}_{ik} + \opr{A}_{ik}^\dag
  = 2\herm \sum_k \opr{A}_{ik},
  \label{eq:paulix-etas} \\
  \shortintertext{where}
  \opr{A}_{ik}
  &= \qty(a_{ik} + b_{ik})\opr{\eta}_k.
  \label{eq:paulix-As}
\end{align}
According to Pfeuty, if $\lambda \ne 0$ then
\begin{equation}
  a_{ik} + b_{ik}
  = \sqrt{\frac{2}{N}}\begin{cases}
    \sin(ki) & k > 0 \\
    \cos(ki) & k \le 0,
  \end{cases}
\end{equation}
and if $\lambda = 0$ then $a_{ik} + b_{ik} = 1 / \sqrt{N}$.

% \begin{align}
%   \ip{\pauli_i^x}{\opr{\eta}_k}
%   &= -\ip{\opr{c}_i^\dag + \opr{c}_i}{%
%     \frac{-\im\sin k}{\sqrt{2E_k\qty(E_k + (\cos k - \lambda))}}
%     \frac{1}{\sqrt{N}}\sum_j e^{-\im kj} \opr{c}_j
%   } \\
%   &\phantom{{}=}-\ip{\opr{c}_i^\dag + \opr{c}_i}{%
%     \sqrt{\frac{E_k + (\cos k - \lambda)}{2E_k}}
%     \frac{1}{\sqrt{N}}\sum_j e^{-\im kj} \opr{c}_j^\dag
%   }
%   \\
%   &= \qty(%
%   \frac{\im\sin k}{\sqrt{2E_k\qty(E_k + (\cos k - \lambda))}}
%   - \sqrt{\frac{E_k + (\cos k - \lambda)}{2E_k}}
%   )
%   \frac{1}{\sqrt{N}} e^{-\im ki}
%   \\
%   \ip{\pauli_i^x}{\opr{\eta}_k^\dag}
%   &= \ip{\opr{\eta}_k}{\pauli_i^x} \\
%   &= \qty(%
%   -\frac{\im\sin k}{\sqrt{2E_k\qty(E_k + (\cos k - \lambda))}}
%   - \sqrt{\frac{E_k + (\cos k - \lambda)}{2E_k}}
%   )
%   \frac{1}{\sqrt{N}} e^{\im ki}
%   \\
%   \ip{\pauli_i^x}{\opr{\eta}_k^\dag\opr{\eta}_k}
%   &= 0 \by{since trace $i$ vanishes} \\
%   \ip{\pauli_i^x}{\opr{\eta}_k\opr{\eta}_k^\dag}
%   &= \ip{\opr{\eta}_k^\dag\opr{\eta}_k}{\pauli_i^x} \\
%   &= 0.
% \end{align}

Now since $\opr{\eta}_k$ is an eigenoperator of the Hamiltonian with eigenvalue
$\omega_k = -E_k$, and $\opr{\eta}_k^\dag$ similarly has $\omega_k = E_k$, we
may use \cref{eq:dissipator,eq:ohmicJ,eq:gamma-corr} to find that the dissipator
is
\begin{align}
  \begin{split}
    \sopr{D} \dop
    &= \sum_{ik}
    \frac{2L}{c} {C(-E_k)}^2 n_B(E_k)
    \qty(%
    \opr{A}_{ik} \dop \opr{A}_{ik}^\dag
    - \frac{1}{2} \acomm{\opr{A}_{ik}^\dag
    \opr{A}_{ik}}{\dop}
    ) \\
    &+ \sum_{ik}
    \frac{2L}{c} {C(E_k)}^2 \qty(n_B(E_k) + 1)
    \qty(%
    \opr{A}_{ik}^\dag \dop \opr{A}_{ik}
    - \frac{1}{2} \acomm{\opr{A}_{ik}
    \opr{A}_{ik}^\dag}{\dop}
    )
  \end{split}
  \\
  \begin{split}
    &= \sum_{ik}
    \frac{2L}{c} {C(-E_k)}^2 n_B(E_k)
    \abs{a_{ik} + b_{ik}}^2
    \qty(%
    \opr{\eta}_k \dop \opr{\eta}_k^\dag
    - \frac{1}{2} \acomm{\opr{\eta}_k^\dag
    \opr{\eta}_k}{\dop}
    ) \\
    &+ \sum_{ik}
    \frac{2L}{c} {C(E_k)}^2 \qty(n_B(E_k) + 1)
    \abs{a_{ik} + b_{ik}}^2
    \qty(%
    \opr{\eta}_k^\dag \dop \opr{\eta}_k
    - \frac{1}{2} \acomm{\opr{\eta}_k
    \opr{\eta}_k^\dag}{\dop}
    )
  \end{split}
  \\
  \begin{split}
    &= \sum_k
    \frac{2L}{Nc} {C(-E_k)}^2 n_B(E_k)
    \qty(%
    \opr{\eta}_k \dop \opr{\eta}_k^\dag
    - \frac{1}{2} \acomm{\opr{\eta}_k^\dag
    \opr{\eta}_k}{\dop}
    ) \\
    &+ \sum_k
    \frac{2L}{Nc} {C(E_k)}^2 \qty(n_B(E_k) + 1)
    \qty(%
    \opr{\eta}_k^\dag \dop \opr{\eta}_k
    - \frac{1}{2} \acomm{\opr{\eta}_k
    \opr{\eta}_k^\dag}{\dop}
    ),
  \end{split}
  \label{eq:ising-diss}
\end{align}
since
\begin{equation}
  \sum_i
  \abs{a_{ik} + b_{ik}}^2
  = \frac{2}{N} \sum_i \sin^2(ki)
  = \frac{1}{N}.
\end{equation}
If $\dop_0 = \op{\tq{0}}{\tq{0}} = \otimes_k \opr{\eta}_k \opr{\eta}_k^\dag$,
then at $t = 0$
\begin{equation}
    \sopr{D} \dop
    = \sum_{ik}
    \frac{2L}{c} {C(E_k)}^2 \qty(n_B(E_k) + 1)
    \opr{A}_{ik}^\dag \opr{A}_{ik}
    \ne 0,
\end{equation}
so the ground state of the closed system is not stationary?


\section{Characterization of two-level dissipators}

If we have a two-level system with density operator $\rho$, a general form of
the dissipator is
\begin{align}
  \begin{split}
    \sopr{D}\dop
    &= \gamma_- \abs{h_-}^2 \qty(\pauli^- \dop \pauli^+
    - \frac{1}{2}\acomm{\pauli^+\pauli^-}{\dop}) \\
    &+ \gamma_+ \abs{h_+}^2 \qty(\pauli^+ \dop \pauli^-
    - \frac{1}{2}\acomm{\pauli_- \pauli^+}{\dop}) \\
    &+ \gamma_z \abs{h_z}^2 \qty(\pauli^z \dop \pauli^z
    - \dop),
  \end{split}
  \label{eq:two-level-diss}
\end{align}
where each of the Pauli operators $\pauli^a$ is assumed to be an eigenoperator
of some Hamiltonian with eigenvalue $\omega_a$, and $\gamma_a =
\gamma(\omega_a)$. We also have coefficients from expressing part of an
interaction Hamiltonian as $\sum_a h_a \pauli^a$.

We evaluate \cref{eq:two-level-diss} by substituting
\begin{equation}
  \dop
  = \rho_{00} \pauli^-\pauli^+
  + \rho_{01} \pauli^-
  + \rho_{10} \pauli^+
  + \rho_{11} \pauli^+ \pauli^-.
\end{equation}
This gives
\begin{align}
  \begin{split}
    \sopr{D}\dop
    &= \gamma_+ \abs{h_+}^2 \qty(\pauli^+ \pauli^-
    - \pauli_- \pauli^+) \rho_{00}
    + \gamma_- \abs{h_-}^2 \qty(\pauli^- \pauli^+
    - \pauli^+\pauli^-) \rho_{11}
    \\
    &+ \gamma_- \abs{h_-}^2 \qty(%
    - \frac{1}{2}\acomm{\pauli^+\pauli^-}{\pauli^-}) \rho_{01}
    + \gamma_+ \abs{h_+}^2 \qty(%
    - \frac{1}{2}\acomm{\pauli_- \pauli^+}{\pauli^-}) \rho_{01} \\
    &+ \gamma_- \abs{h_-}^2 \qty(%
    - \frac{1}{2}\acomm{\pauli^+\pauli^-}{\pauli^+}) \rho_{10}
    + \gamma_+ \abs{h_+}^2 \qty(%
    - \frac{1}{2}\acomm{\pauli_- \pauli^+}{\pauli^+}) \rho_{10} \\
    &+ \gamma_z \abs{h_z}^2 \qty(\pauli^z \pauli^- \pauli^z
    - \pauli^-) \rho_{01}
    + \gamma_z \abs{h_z}^2 \qty(\pauli^z \pauli^+ \pauli^z
    - \pauli^+) \rho_{10}
  \end{split}
  \\
  \begin{split}
    &= \gamma_+ \abs{h_+}^2 \qty(\pauli^+ \pauli^-
    - \pauli_- \pauli^+) \rho_{00}
    + \gamma_- \abs{h_-}^2 \qty(\pauli^- \pauli^+
    - \pauli^+\pauli^-) \rho_{11}
    \\
    &- \gamma_- \abs{h_-}^2
    \frac{1}{2}\pauli^- \rho_{01}
    - \gamma_+ \abs{h_+}^2
    \frac{1}{2}\pauli^- \rho_{01}
    - \gamma_z \abs{h_z}^2 2\pauli^- \rho_{01}
    \\
    &- \gamma_- \abs{h_-}^2
    \frac{1}{2}\pauli^+ \rho_{10}
    - \gamma_+ \abs{h_+}^2
    \frac{1}{2}\pauli^+ \rho_{10}
    - \gamma_z \abs{h_z}^2 2\pauli^+ \rho_{10}.
  \end{split}
\end{align}
Thus ignoring the Lamb shift, the density matrix in the interaction picture
evolves according to
\begin{subequations}
  \begin{align}
    \dot{\rho}_{00}
    &= -\gamma_+ \abs{h_+}^2 \rho_{00}
    + \gamma_- \abs{h_-}^2 \rho_{11}
    \\
    \dot{\rho}_{01}
    &= -\frac{1}{2}\qty(%
    \gamma_- \abs{h_-}^2
    + \gamma_+ \abs{h_+}^2
    + 4\gamma_z \abs{h_z}^2
    ) \rho_{01}
    \\
    \dot{\rho}_{10}
    &= -\frac{1}{2}\qty(%
    \gamma_- \abs{h_-}^2
    + \gamma_+ \abs{h_+}^2
    + 4\gamma_z \abs{h_z}^2
    ) \rho_{10}
    \\
    \dot{\rho}_{11}
    &= \gamma_+ \abs{h_+}^2 \rho_{00}
    - \gamma_- \abs{h_-}^2 \rho_{11}
  \end{align}\label{eq:pauli-diss}
\end{subequations}
In the case where $h_z = 0$ and $h_- = h_+ \eqqcolon h$, \cref{eq:pauli-diss}
reduces to
\begin{subequations}
  \begin{align}
    \dot{\rho}_{00} \abs{h}^{-2}
    &= -\gamma_+ \rho_{00}
    + \gamma_- \rho_{11}
    \\
    \dot{\rho}_{01} \abs{h}^{-2}
    &= -\frac{1}{2}\qty(%
    \gamma_-
    + \gamma_+
    ) \rho_{01}
    \\
    \dot{\rho}_{10} \abs{h}^{-2}
    &= -\frac{1}{2}\qty(%
    \gamma_-
    + \gamma_+
    ) \rho_{10}
    \\
    \dot{\rho}_{11} \abs{h}^{-2}
    &= \gamma_+ \rho_{00}
    - \gamma_- \rho_{11}.
  \end{align}\label{eq:pauli-diss-noz}
\end{subequations}


% \begin{align}
%   \begin{split}
%     \sopr{D}\dop
%     &= \gamma(\omega_-) \abs{h_-}^2 \comm{\pauli^-}{\pauli^+} \dop_0 \\
%     &+ \gamma(\omega_+) \abs{h_+}^2 \comm{\pauli^+}{\pauli^-} \dop_0
%     \\
%     &+ \gamma(\omega_-) \abs{h_-}^2 \qty(%
%     - \frac{1}{2}\acomm{\pauli^+\pauli^-}{\pauli^-}) \dop_- \\
%     &+ \gamma(\omega_+) \abs{h_+}^2 \qty(%
%     - \frac{1}{2}\acomm{\pauli_- \pauli^+}{\pauli^-}) \dop_- \\
%     &+ \gamma(\omega_z) \abs{h_z}^2 \qty(\pauli^z \pauli^- \pauli^z
%     - \pauli^-) \rho_-
%     \\
%     &+ \gamma(\omega_-) \abs{h_-}^2 \qty(%
%     - \frac{1}{2}\acomm{\pauli^+\pauli^-}{\pauli^+}) \dop_+ \\
%     &+ \gamma(\omega_+) \abs{h_+}^2 \qty(%
%     - \frac{1}{2}\acomm{\pauli_- \pauli^+}{\pauli^+}) \dop_+ \\
%     &+ \gamma(\omega_z) \abs{h_z}^2 \qty(\pauli^z \pauli^+ \pauli^z
%     - \pauli^+) \rho_+
%     \\
%     &+ \gamma(\omega_-) \abs{h_-}^2 \qty(\pauli^- \pauli^z \pauli^+
%     - \frac{1}{2}\acomm{\pauli^+\pauli^-}{\pauli^z}) \dop_z \\
%     &+ \gamma(\omega_+) \abs{h_+}^2 \qty(\pauli^+ \pauli^z \pauli^-
%     - \frac{1}{2}\acomm{\pauli_- \pauli^+}{\pauli^z}) \dop_z
%   \end{split}
%   \\
%   \begin{split}
%     &= -\gamma(\omega_-) \abs{h_-}^2 \pauli^z \dop_0 \\
%     &+ \gamma(\omega_+) \abs{h_+}^2 \pauli^z \dop_0
%     \\
%     &- \gamma(\omega_-) \abs{h_-}^2
%     \frac{1}{2}\pauli^- \dop_- \\
%     &- \gamma(\omega_+) \abs{h_+}^2
%     \frac{1}{2}\pauli^- \dop_- \\
%     &- \gamma(\omega_z) \abs{h_z}^2 2\pauli^- \rho_-
%     \\
%     &- \gamma(\omega_-) \abs{h_-}^2
%     \frac{1}{2}\pauli^+ \dop_+ \\
%     &- \gamma(\omega_+) \abs{h_+}^2
%     \frac{1}{2}\pauli^+ \dop_+ \\
%     &- \gamma(\omega_z) \abs{h_z}^2 2\pauli^+ \rho_+
%     \\
%     &+ \gamma(\omega_-) \abs{h_-}^2 \qty(\pauli^- \pauli^z \pauli^+
%     - \frac{1}{2}\acomm{\pauli^+\pauli^-}{\pauli^z}) \dop_z \\
%     &+ \gamma(\omega_+) \abs{h_+}^2 \qty(\pauli^+ \pauli^z \pauli^-
%     - \frac{1}{2}\acomm{\pauli_- \pauli^+}{\pauli^z}) \dop_z
%   \end{split}
% \end{align}

\section{Pfeuty scratch work}

I have
\begin{equation}
  \lambda
  = -\frac{1}{\lambda_\text{Pf}}
  = -\frac{2\Gamma}{J},
\end{equation}
and my Hamiltonian \cref{eq:transham} should have
\begin{equation}
  \ham
  \mapsto \frac{4}{J}\ham
\end{equation}
to be correctly nondimensionalized and match Pfeuty's results. Then
\begin{equation}
  E_k
  = \frac{\Lambda_k}{\lambda_\text{Pf}}
  = -\Lambda_k \lambda.
\end{equation}

Pfeuty defines:
\begin{align}
  \lambda
  &= \frac{J}{2\Gamma}
  \label{eq:pflambda} \\
  %
  \opr{a}_i
  &= \opr{S}_{xi} - \im\opr{S}_{yi} 
  \label{eq:pfa} \\
  %
  \opr{a}_i^\dag
  &= \opr{S}_{xi} + \im\opr{S}_{yi} 
  \label{eq:pfadag} \\
  %
  \opr{c}_i
  &= \exp(\pi\im\sum_{j=1}^{i-1} \opr{a}_j^\dag\opr{a}_j)\opr{a}_i
  \label{eq:pfc} \\
  %
  \opr{c}_i^\dag
  &= \opr{a}_i^\dag\exp(-\pi\im\sum_{j=1}^{i-1} \opr{a}_j^\dag\opr{a}_j)
  \label{eq:pfcdag} \\
  %
  \opr{\eta}_k
  &= \sum_i \qty(\frac{\varphi_{ki} + \psi_{ki}}{2}\opr{c}_i
  + \frac{\varphi_{ki} - \psi_{ki}}{2}\opr{c}_i^\dag)
  \label{eq:pfeta} \\
  %
  \varphi_{ki}
  &= \sqrt{\frac{2}{N}}\begin{cases}
    \sin(ki) & k > 0 \\
    \cos(ki) & k \le 0
  \end{cases}
  \label{eq:pfphi} \\
  %
  \psi_{ki}
  &= -\Lambda_k^{-1} \qty((1 + \lambda\cos k)\varphi_{ki}
  + (\lambda\sin k)\varphi_{-ki})
  \label{eq:pfpsi} \\
  %
  \Lambda_k^2
  &= 1 + \lambda^2 + 2\lambda\cos k
  \label{eq:pfLambda} \\
  %
  k
  &= \frac{2\pi m}{N} \qfor m = -\frac{N}{2},\, \ldots,\, \frac{N}{2} - 1,
  \qq{$N$ even.}
  \label{eq:pfk}
\end{align}
We would like to express the $\opr{S}_{xi}$ in terms of eigenoperators of the
system Hamiltonian
\begin{equation}
  \ham
  = \Gamma\sum_k \Lambda_k \opr{\eta}_k^\dag\opr{\eta}_k
  - \frac{\Gamma}{2}\sum_k \Lambda_k.
  \label{eq:pfH}
\end{equation}

\begin{align}
  \begin{split}
    \acomm{\opr{\eta}_k}{\opr{\eta}_k^\dag}
  &=
  \sum_{ij}
  \frac{\varphi_{ki} + \psi_{ki}}{2}
  \frac{\varphi_{kj} - \psi_{kj}}{2}
  \acomm{\opr{c}_i}{\opr{c}_j}
  +
  \sum_{ij}
  \frac{\varphi_{ki} + \psi_{ki}}{2}
  \frac{\varphi_{kj} + \psi_{kj}}{2}
  \acomm{\opr{c}_i}{\opr{c}_j^\dag} \\
  &+
  \sum_{ij}
  \frac{\varphi_{ki} - \psi_{ki}}{2}
  \frac{\varphi_{kj} - \psi_{kj}}{2}
  \acomm{\opr{c}_i^\dag}{\opr{c}_j}
  +
  \sum_{ij}
  \frac{\varphi_{ki} - \psi_{ki}}{2}
  \frac{\varphi_{kj} + \psi_{kj}}{2}
  \acomm{\opr{c}_i^\dag}{\opr{c}_j^\dag}
  \end{split} \\
  %
  &= \frac{\idopr}{4} \sum_{i}
  {\qty(\varphi_{ki} + \psi_{ki})}^2
  + {\qty(\varphi_{ki} - \psi_{ki})}^2
  \\
  %
  &= \frac{\idopr}{2} \sum_{i}
  \varphi_{ki}^2 + \psi_{ki}^2
  \\
  %
  &= \idopr?
\end{align}
According to Mathematica, no? But~\cite[pp.~452--454]{liebTwoSolubleModels1961}
is probably right.

\end{document}

