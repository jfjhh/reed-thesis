\documentclass[../thesis.tex]{subfiles}
\begin{document}

\chapter{Computing jump operators}\label{ch:computations}

Now that we have obtained $\tilde{\gamma}$ in \cref{eq:gamma-nondim}, it remains
to compute the jump operators \cref{eq:Aprojectors} for the interpolating
Hamiltonian \cref{eq:Hinterp} and solve the Lindblad equation to find the
relaxation rates of the system. For simplicity, we will aim to estimate a single
exponential relaxation rate for the expected value $\ev{\pauli_1^z}$ of a single
spin.\footnote{%
  Since are using periodic boundary conditions for the transverse-field Ising
  model, this expected value is the same for all spins.
}
These tasks will be done by numerically diagonalizing the dissipator of the
system as follows. Just as we usually consider the eigenstates of a Hamiltonian
to solve a closed system, we may consider the eigenoperators of the Liouvillian
$\sopr{L}$ to solve an open system. We decompose the initial system density
operator as
\begin{equation}
  \dop(0)
  = \sum_i c_i \opr{V}_i,
\end{equation}
where
\begin{equation}
  \sopr{L}(\opr{V}_i)
  = \lambda_i \opr{V}_i.
\end{equation}
The reduced density operator for the system then evolves in time as
\begin{equation}
  \dop(\tilde{t})
  = \sum_i c_i e^{\lambda_i \tilde{t}} \opr{V}_i,
\end{equation}
and the desired observable is
\begin{equation}
  \ev{\pauli_1^z}
  = \tr(\dop(\tilde{t})\pauli_1^z)
  = \sum_i c_i e^{\lambda_i \tilde{t}} \tr(\opr{V}_i \pauli_1^z).
  \label{eq:multi-exponential}
\end{equation}
Note that the eigenoperators $\opr{V}_i$ are generally not valid density
operators. In this case, the diagonalization is a formal technique for solving
the system, rather than a way of identifying stationary quantum states. Thus an
observable like $\ev{\pauli_1^z}$ is seen to decay according to a sum of
decaying exponentials with rates $\lambda_i$. The system and interaction
Hamiltonians determine the $\opr{V}_i$, while the initial state determines the
$c_i$. The combination of these sources can be seen as filtering the full
spectrum of the Liouvillian to produce the time evolution of $\ev{\pauli_1^z}$.
To merely determine the decay rates and not the full time evolution, it suffices
to diagonalize just the dissipator (\cref{eq:dissipator}). To understand this
situation, we will borrow from solid state physics and consider \term{spaghetti
diagrams}: plots of the dissipator eigenvalues as we vary the Hamiltonian
parameter $g$. \Cref{fig:spaghetti} explains the name.

\begin{figure}[ht]
  \centering
  \includegraphics[width=0.5\linewidth]{spaghetti}
  \caption{%
    A spaghetti diagram~\cite{bernasconiGiorgioBenedekExtraordinary2012}.
  }\label{fig:spaghetti}
\end{figure}

In addition to varying the parameter $g$, we also would like to vary the
dimensionless inverse temperature $\eta$. It is a bit confusing to decipher
spaghetti diagrams across $g$ and temperature, so we will instead perform a
least squares fit of the single exponential model
\begin{equation}
  \ev{\pauli_1^z}(\tilde{t})
  = A e^{-s \tilde{t}} + \ev{\pauli_1^z}(\infty)
  \label{eq:single-exp}
\end{equation}
to the true multiple exponential envelope \cref{eq:multi-exponential} over the
time interval $[0,\, \tilde{t}_f]$, where
\begin{equation}
  \tilde{t}_f
  = \frac{1}{10\tilde{\gamma}(1)},
  \label{eq:time-interval}
\end{equation}
and instead inspect the effective relaxation rate $s$. While this rate depends
on the particular observable and initial condition, it will still give a rough
picture of what happens as we vary $g$.

With this setup in mind, we may now go over the rate computation process. In the
interest of transparency, code snippets are given in
\href{https://julialang.org}{Julia}, and the information on packages and
hardware used is given in \cref{ch:computer-details}.

\notebook{document-computations}

\section{Results}

The results of the preceeding computations for $N$ from \numrange{2}{5} are
given on the following pages. For easy comparison, each set of plots for a given
number of spins spans an even and odd page. Note that in
\cref{fig:spin-relaxation-2,fig:spin-relaxation-3,fig:spin-relaxation-4,fig:spin-relaxation-5}
the first three plots show relaxation rates decreasing with temperature, while
the next three plots flip the order.

\newcommand*\cleartoleftpage{%
  \clearpage
  \ifodd\value{page}\hbox{}\newpage\fi
}

\cleartoleftpage%

\foreach\n in {2,3,4,5}{%
  \begin{figure}[H]
    \centering
    \includegraphics[width=0.8\linewidth]{time-evolution-\n}
    \caption{%
      Example time evolution for $N = \n$ spins.
    }\label{fig:time-evolution-\n}
  \end{figure}
  \begin{figure}[H]
    \centering
    \includegraphics[width=0.8\linewidth]{exponential-fit-\n}
    \caption{%
      Exponential fit for $N = \n$ spins.
    }\label{fig:exponential-fit-\n}
  \end{figure}
  
  \begin{figure}[H]
    \centering
    \includegraphics[width=0.8\linewidth]{energy-levels-\n}
    \caption{%
      Energy levels of \cref{eq:Hinterp} for $N = \n$ spins.
    }\label{fig:energy-levels-\n}
  \end{figure}
  \begin{figure}[H]
    \centering
    \includegraphics[width=0.8\linewidth]{spin-spectrum-\n}
    \caption{%
      Dissipation rate spaghetti diagram for $N = \n$ spins.
    }\label{fig:spin-spectrum-\n}
  \end{figure}
}

\foreach\n in {2,3,4,5}{%
  \begin{figure}[ht]
    \centering
    \includegraphics[width=\linewidth]{spin-relaxation-\n}
    \caption{%
      Single-spin relaxation rates in different temperature regimes for $N = \n$
      spins.
    }\label{fig:spin-relaxation-\n}
  \end{figure}
}

\clearpage

\end{document}

