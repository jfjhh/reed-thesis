\documentclass[../thesis.tex]{subfiles}
\begin{document}

\chapter{Density operator theory}

\dropcap{I}{f quantum mechanics} is so weird, then why aren't we? The answer
lies in the phenomenon of \term{decoherence}, which was first considered in
depth in the 1970's by Zeh~\cite{zeh}. Despite being a late comer to the history
of quantum mechanics, the theory of decoherence is crucial to understanding how
classical results are obtained from many interacting quantum systems. While
there are other routes to decoherence, the most common is through interaction
with a memoryless (Markovian) environment~\cite{decoherence}. This leads to the
theory of quantum Markovian master equations, which describe transformations of
a system in the presence of such an environment. The most general form of these
transformations was first extended to quantum mechanics by Gorini, Kossakowski,
Lindblad, and Sudarshan to give the \textsc{\term{gkls}} or \term{Lindblad
equation}~\cite{gks,lindblad}.

This chapter will explain the relevant theoretical background in
\cref{sec:dop,sec:composite}, before presenting the general theory which leads
to the Lindblad equation in \cref{sec:lindblad}, following the text of Breuer
and Petruccione~\cite{opensys} and to a lesser extent~\cite{intro} (which has
some flaws). This leads us to consider the weak-coupling limit in
\cref{sec:interaction} and an application to atomic physics in
\cref{sec:example}. Issues with the method are discussed briefly in
\cref{sec:limitations}.

\section{Different perspectives on density operators\label{sec:dop}}

For a statistical perspective on quantum mechanics, we will make two postulates.
The mathematical background is the \term{Liouville space} $\liou(\hilb)$ for the
Hilbert space $\hilb$.
\begin{defn}[Liouville space]
  The space $\liou(\hilb)$ is the complex Hilbert space of operators $\opr{A}$
  on $\hilb$ for which the norm induced by the inner product
  $\lprod{\opr{A}}{\opr{B}} \defeq \tr(\opr{A}^\dag \opr{B})$ is finite.
\end{defn}

\begin{post}\label{post:ensemble}
  A quantum system may be understood as a statistical ensemble
  $\dop$ with observables $\opr{O}$ which are both described by elements of
  $\liou(\hilb)$, where the ensemble average of $\opr{O}$ is $\ensavg{\opr{O}}
  \equiv \ip{\opr{O}}{\dop}$ and $\opr{O}$ is Hermitian.
\end{post}
The usual properties of the \term{density operator} $\dop$ follow from
considering various averages. The only way for $\ensavg{\alpha\idopr} = \alpha$
for all physical constants $\alpha \in \CC$ is if $\tr\dop = 1$. For
$\ensavg{\opr{O}}$ to be real, $\dop$ must be \emph{self-adjoint}, and if
$\opr{O}$ is also positive, then $\dop$ must be \emph{positive} for
$\ensavg{\opr{O}}$ to be
positive~\cite{weinbergQuantumMechanicsState2014}.

\begin{post}\label{post:time}
  The density operator for an isolated system with Hamiltonian $\ham$ evolves
  unitarily in time according to the \term{Liouville-von Neumann equation}
  \begin{equation}
    \dot{\dop}
    = \flatfrac{\comm{\ham}{\dop}}{\im\hbar}\!.
    \label{eq:liouville}
  \end{equation}
\end{post}

While we usually consider the density operator to change in time, the time
dependence may be shifted onto the observables. Consider a quantum system with
unitary time-evolution operator $\opr{U}(t)$, so that we may express the density
operator for the system as $\dop(t) = \opr{U}(t) \dop \opr{U}^\dag(t)$, where
$\dop = \dop(0)$. If we compute the ensemble average of an observable
$\opr{O}(t)$ and cycle the trace, we find
\begin{align}
  \ensavg[\dop(t)]{\opr{O}(t)}
  &= \tr(\opr{O}(t)\opr{U}(t)\dop\opr{U}^\dag(t)) \label{eq:preheis} \\
  &= \tr(\opr{U}^\dag(t)\opr{O}(t)\opr{U}(t)\dop)
  \equiv \ensavg{\opr{O}_H(t)}
  \label{eq:heisenbergpic}
\end{align}
where $\opr{O}_H$ is the observable in the \term{Heisenberg picture}, as opposed
to the \term{Schr\"odinger picture}, where the operators are time-independent.
If we can split the Hamiltonian into the form $\ham = \ham_0 + \ham_I(t)$, then
$\opr{U}(t)$ splits into the product of $\opr{U}_0(t) = e^{\ham_0 t / \im\hbar}$
and $\opr{U}_I(t) = \opr{U}_0^\dag(t)\opr{U}(t)$. Cycling over only
$\opr{U}_I(t)$ in \cref{eq:preheis} gives the \term{interaction picture}
operators
\begin{subequations}\label{eq:intpic}
\begin{align}
  \opr{O}_I(t)
  &= \opr{U}_0^\dag(t) \opr{O}(t) \opr{U}_0(t)
  \label{eq:intobs} \\
  \dop_I(t)
  &= \opr{U}_I(t) \dop \opr{U}_I^\dag(t).
  \label{eq:intdop}
\end{align}
\end{subequations}
Without $\ham_I(t)$, \cref{eq:intpic} reduces to the Schr\"odinger picture, and
without $\ham_0$, \cref{eq:intpic} reduces to the Heisenberg picture. The
time-dependence of the interaction picture density operator from
differentiating \cref{eq:intdop} is (suppressing time dependences)
\begin{align}
  \im\hbar\mkern1mu\dot{\dop}_I(t)
  &= \im\hbar\dv{t} \qty(\opr{U}_0^\dag\dop(t)\opr{U}_0) \nonumber \\
  &= -\opr{U}_0^\dag\ham_0^\dag\dop(t)\opr{U}_0
  + \opr{U}_0^\dag\dop(t)\ham_0\opr{U}_0 \nonumber \\
  &+ \opr{U}_0^\dag\comm{\ham_0}{\dop(t)}\opr{U}_0
  + \opr{U}_0^\dag\comm{\ham_I}{\dop(t)}\opr{U}_0 \nonumber \\
  &= \opr{U}_0^\dag\comm{\opr{U}_0\ham_I'\opr{U}_0^\dag}{\dop(t)}\opr{U}_0
  \nonumber \\
  &= \comm{\ham_I'}{\dop_I(t)},
  \label{eq:intliou}
\end{align}
where $\ham_I'$ denotes the interaction Hamiltonian $\ham_I(t)$ in the
interaction picture. This is just \cref{eq:liouville} with the interaction
Hamiltonian.

With this understanding of the behavior of isolated systems, it may be
surprising that \cref{post:ensemble,post:time} are actually
insufficient to describe common systems. For example, the allowed energies for
the harmonic oscillator are unbounded, so the Hamiltonian is not an element of
the Liouville space. We will see later how this issue is related to the dynamics
of a composite quantum system in \cref{sec:limitations}, but will now move
on to considering the more general dynamics of interacting quantum systems.

\section{Composite quantum systems\label{sec:composite}}

\todo[inline]{Clean up inner product notation.}

Consider two quantum systems described by Hilbert spaces $\hilb_A$ and
$\hilb_B$. We should be able to view both systems as parts of a composite
quantum system described by some Hilbert space $\hilb$. For the moment, we
will only consider the Hilbert spaces as vector spaces. Mathematically, we have
a map $\phi:\hilb_A \times \hilb_B \to \hilb$. The simplest choice would be
$\hilb = \hilb_A \times \hilb_B$, but this does not make physical sense. For
the assignment of composite states to be consistent, if a linear combination is
taken in $\hilb_A$ then joined with $\hilb_B$, the result must be the same as
first joining and then taking the linear combination. If $\ket{a_1}$, $\ket{a_2}
\in \hilb_A$, $\ket{b} \in \hilb_B$, and $\alpha$, $\beta \in \CC$, then
\begin{equation}
  \phi\qty(\alpha\ket{a_1} + \beta\ket{a_2},\, \ket{b})
  = \alpha\phi\qty(\ket{a_1},\, \ket{b}) + \beta\phi\qty(\ket{a_2},\, \ket{b}),
  \label{eq:consistent-composite}
\end{equation}
so that $\phi$ is bilinear.

Taking $\hilb = \hilb_A \times \hilb_B$ by doing operations elementwise is
known as the \term{direct sum} $\hilb_A \oplus \hilb_B$. Applying $\phi':\hilb_A
\times \hilb_B \to \hilb_A \oplus \hilb_B$ gives
\begin{align}
  \alpha\phi'\qty(\ket{a_1},\, \ket{b}) + \beta\phi'\qty(\ket{a_2},\, \ket{b})
  &= \alpha\qty(\ket{a_1},\, \ket{b}) + \beta\qty(\ket{a_2},\, \ket{b}) \\
  &= \qty(\alpha\ket{a_1} + \beta\ket{a_2},\, 2\ket{b}) \\
  &\ne \qty(\alpha\ket{a_1} + \beta\ket{a_2},\, \ket{b}) \\
  &= \phi'\qty(\alpha\ket{a_1} + \beta\ket{a_2},\, \ket{b}),
\end{align}
since $\ket{b}$ is normalized in $\hilb_B$. Thus the direct sum is not a
physically consistent construction of a composite system.

Instead, we may define the composite system and corresponding map so that they
satisfy \cref{eq:consistent-composite}. We consider the combination of two
states first as a symbol, giving the set of symbols
\begin{equation}
  B
  = \qty{\tilde{\phi}\qty(\ket{a}, \ket{b}) : \ket{a} \in \hilb_A,\, \ket{b} \in
  \hilb_B}.
\end{equation}
The formal complex linear combinations of elements of $B$ form a vector space,
known as the \term{free vector space} $F(\hilb_A \times \hilb_B)$. The composite
system is then the quotient vector space $\hilb = F(\hilb_A \times \hilb_B)
/\sim$, where $\sim$ is an equivalence relation that identifies linear
combinations like in \cref{eq:consistent-composite}. These equivalence classes
are known as \term{tensors}. We write the class $\phi\qty(\ket{a}, \ket{b}) =
[\tilde{\phi}(\ket{a}, \ket{b})]$ as $\ket{a} \tp \ket{b}$, which is often
abbreviated to $\ket{a}\ket{b}$ or $\ket{a,b}$.

To make $\hilb$ a Hilbert space, we may complete it with a suitable inner
product. Consider the special case when the states are normalized and
$\ip{a_1}{a_2} = 0$. We have another consistency criterion: the composite state
should also be normalized. We find
\begin{align}
  1
  &= \ip{\qty(\alpha\ket{a_1} + \beta\ket{a_1}) \tp \ket{b}}{%
  \qty(\alpha\ket{a_1} + \beta\ket{a_1}) \tp \ket{b}} \\
  &= \abs{\alpha}^2 \ip{\ket{a_1} \tp \ket{b}}{\ket{a_1} \tp \ket{b}}
  + \abs{\beta}^2 \ip{\ket{a_2} \tp \ket{b}}{\ket{a_2} \tp \ket{b}} \\
  &+ \alpha\beta^* \ip{\ket{a_1} \tp \ket{b}}{\ket{a_2} \tp \ket{b}}
  + \alpha^*\beta \ip{\ket{a_2} \tp \ket{b}}{\ket{a_1} \tp \ket{b}} \\
  &= \abs{\alpha}^2 + \abs{\beta}^2
  + \alpha\beta^* \ip{\ket{a_1} \tp \ket{b}}{\ket{a_2} \tp \ket{b}}
  + \alpha^*\beta \ip{\ket{a_2} \tp \ket{b}}{\ket{a_1} \tp \ket{b}}.
\end{align}
This requires that
\begin{equation}
  \alpha\beta^* \ip{\ket{a_1} \tp \ket{b}}{\ket{a_2} \tp \ket{b}}
  = -\alpha^*\beta \ip{\ket{a_1} \tp \ket{b}}{\ket{a_2} \tp \ket{b}}^*,
\end{equation}
so if $\alpha$, $\beta \neq 0$, the inner product $\ip{\ket{a_1} \tp
\ket{b}}{\ket{a_2} \tp \ket{b}}$ must be purely imaginary. But since the
superposition state is to 1, this establishes that $\ket{a_1} \tp \ket{b}$ and
$\ket{a_2} \tp \ket{b}$ must be orthonormal. In general, consistency requires
that
\begin{equation}
  \ip{\ket{a_i} \tp \ket{b_m}}{\ket{a_j} \tp \ket{b_n}}
  = \delta_{ij}\delta_{mn}.
\end{equation}
Hence consistency has determined the inner product. For arbitrary vectors
$\ket{a} = \sum_i c_i \ket{a_i}$, $\ket{\tilde{a}} = \sum_j c_j' \ket{a_j}$,
$\ket{b} = \sum_m d_m \ket{b_m}$, and $\ket{\tilde{b}} = \sum_n d_n' \ket{b_n}$,
\begin{align}
  \ip{\ket{a} \tp \ket{b}}{\ket{\tilde a} \tp \ket{\tilde b}}
  &= \sum_{ijmn} c_i \tilde{c}_j^* d_m \tilde{d}_n^*
  \ip{\ket{a_i} \tp \ket{b_m}}{\ket{a_j} \tp \ket{b_n}}
  \\
  &= \sum_{ijmn} c_i \tilde{c}_j^* d_m \tilde{d}_n^*
  \delta_{ij}\delta_{mn}
  \\
  &= \qty(\sum_{ij} c_i \tilde{c}_j^* \ip{a_i}{a_j})
  \qty(\sum_{mn} d_m \tilde{d}_n^* \ip{b_m}{b_n})
  \\
  &= \ip{a}{\tilde{a}} \ip{b}{\tilde{b}}.
  \label{eq:composite-ip}
\end{align}
\Cref{eq:composite-ip} is then a valid definition for the inner product, with
consistency upheld by extending to linear combinations of tensors.

Given an operator $\opr{A}$ on $\hilb_A$, what is the corresponding operator
$\tilde{\opr{A}}$ on $\hilb$? For $\ev{\tilde{\opr{A}}}{a,b} = \ev{\opr{A}}{a}$,
we must have
\begin{equation}
  \tilde{\opr{A}}
  = \opr{A} \tp \idopr,
\end{equation}
where action of the operator $\opr{A} \tp \opr{B}$ on $\hilb$ is defined by
\begin{equation}
  (\opr{A} \tp \opr{B})(\ket{a} \tp \ket{b})
  = \opr{A}\ket{a} \tp \opr{B}\ket{b}.
\end{equation}
The \term{adjoint} $\opr{T}^\dag$ of $\opr{T}:\hilb_A \to \hilb_B$ satisfies that
\begin{equation}
  \ip{\opr{T}\ket{a}}{\ket{b}}
  = \ip{\ket{a}}{\opr{T}^\dag\ket{b}}
  \qfor \ket{a} \in \hilb_A,\, \ket{b} \in \hilb_B.
\end{equation}
Thus for $\opr{A} \tp \opr{B}: \hilb \to \hilb$,
\begin{align}
  \ip{(\opr{A} \tp \opr{B})(\ket{a} \tp \ket{b})}{\ket{c} \tp \ket{d}}
  &= \ip{\opr{A}\ket{a} \tp \opr{B}\ket{b}}{\ket{c} \tp \ket{d}} \\
  &= \ip{\opr{A}\ket{a}}{\ket{c}} \ip{\opr{B}\ket{b}}{\ket{d}} \\
  &= \ip{\ket{a}}{\opr{A}^\dag\ket{c}} \ip{\ket{b}}{\opr{B}^\dag\ket{d}} \\
  &= \ip{\ket{a} \tp \ket{b}}{\opr{A}^\dag\ket{c} \tp \opr{B}^\dag\ket{d}} \\
  &= \ip{\ket{a} \tp \ket{b}}{\qty(\opr{A}^\dag \tp \opr{B}^\dag)(\ket{c} \tp
  \ket{d})}.
\end{align}
Hence
\begin{equation}
  {\qty(\opr{A} \tp \opr{B})}^\dag
  = \opr{A}^\dag \tp \opr{B}^\dag.
\end{equation}
Given $\ket{a,b} \in \hilb$, we will often use an operator on part of $\hilb$ as
if it were an operator on $\hilb$. For example, $\opr{A}\ket{a,b}$ is taken to
mean $(\opr{A} \tp \idopr)\ket{a,b}$. This can be dangerous: we also write
expressions like $(\opr{A} + \opr{B})\ket{a,b}$, which we intend to mean
$\opr{A}\ket{a} \tp \opr{B}\ket{b}$, but this conflicts with using
$\opr{A}\opr{B}$ to do the same thing. Additionally, using the product notation
makes $\opr{B}\opr{A} = (\idopr \tp \opr{B})(\opr{A} \tp \idopr) =
\opr{A}\opr{B}$ so that the ``commutator'' $\comm{\opr{A}}{\opr{B}} = \zopr$.

How might a density operator $\dop$ for the composite system admit a
\term{reduced density operator} $\dop_A$ for system $A$? Consider an observable
$\opr{O}$ of system $A$, for which the corresponding composite observable is
$\opr{O} \tp \idopr$. Regardless of the representation (composite or subsystem),
the ensemble average of $\opr{O}$ should be the same:
\begin{equation}
  \ensavg{\opr{O}_A \tp \idopr}
  = \ensavg[\dop_A]{\opr{O}_A}.
  \label{eq:sameavg}
\end{equation}

To see what $\dop_A$ is, take an orthonormal complete basis $\qty{\opr{A}_j}$ of
Hermitian operators for $\liou(\hilb_A)$ and $\qty{\opr{B}_k}$ for
$\liou(\hilb_B)$, so the density operator may be expressed as
\begin{equation}
  \dop
  = \sum_{jk}
  \opr{A}_j \tp \opr{B}_k
  \lprod{\opr{A}_j \tp \opr{B}_k}{\dop}.
  \label{eq:ptracedopdef}
\end{equation}
We may then compute that
\begin{align}
  \dop_A
  &= \sum_i \opr{A}_i \lprod{\opr{A}_i}{\dop_A}
  \\
  &= \sum_i \opr{A}_i \lprod{\opr{A}_i \tp \idopr}{\dop}
  \byeq{eq:sameavg}
  \\
  &= \sum_i \opr{A}_i \lprod{\opr{A}_i \tp \idopr}{%
    \textstyle \sum_{jk}
    \opr{A}_j \tp \opr{B}_k
    \lprod{\opr{A}_j \tp \opr{B}_k}{\dop}
  }
  \byeq{eq:ptracedopdef}
  \\
  &= \sum_{ijk}
  \opr{A}_i \lprod{\opr{A}_i \tp \idopr}{\opr{A}_j \tp \opr{B}_k}
  \lprod{\opr{A}_j \tp \opr{B}_k}{\dop}
  \\
  &= \sum_{ijk}
  \opr{A}_i \lprod{\opr{A}_i}{\opr{A}_j}\lprod{\opr{B}_k}{\idopr}
  \lprod{\opr{A}_j \tp \opr{B}_k}{\dop}
  \\
  &= \sum_{ijk}
  \opr{A}_i \delta_{ij}\tr\opr{B}_k
  \lprod{\opr{A}_j \tp \opr{B}_k}{\dop}
  \\
  &= \sum_{jk}
  \opr{A}_j \tr\opr{B}_k
  \lprod{\opr{A}_j \tp \opr{B}_k}{\dop}.
  \label{eq:partialptrace}
\end{align}
\begin{defn}\label{def:ptrace}
  At this point, it makes sense to define the \term{partial trace} by
  \begin{equation}
    \tr_B\qty(\opr{A} \tp \opr{B})
    = \opr{A} \tr\opr{B}
    \label{eq:ptrace}
  \end{equation}
  and extending linearly.
\end{defn}
With this definition, we continue from \cref{eq:partialptrace} to find that
\begin{align}
  \dop_A
  &= \sum_{jk}
  \tr_B\qty(\opr{A}_j \tp \opr{B}_k)
  \lprod{\opr{A}_j \tp \opr{B}_k}{\dop}
  \\
  &= \tr_B\qty(\textstyle \sum_{jk}
  \opr{A}_j \tp \opr{B}_k
  \lprod{\opr{A}_j \tp \opr{B}_k}{\dop})
  \\
  &= \tr_B\dop.
\end{align}
Thus the reduced density matrix for system $A$ is obtained from the full density
matrix by taking the partial trace over system $B$.

However, the reduction of the density operator by ``tracing out'' $B$ comes at
the cost of losing information about the correlation between $A$ and $B$.
Quantitatively, the \term{relative entropy} between the correlated and
uncorrelated density operators is
\begin{align}
  &S(\dop \| \dop_A \tp \dop_B) \\
  &\equiv \tr\dop(\ln\dop - \ln(\dop_A \tp \dop_B)) \\
  &= \tr(\dop\ln\dop)
  - \tr_A\tr_B\qty(\dop\ln(\dop_A \tp \idopr))
  - \tr_B\tr_A\qty(\dop\ln(\idopr \tp \dop_B)) \\
  &= S(\dop_A) + S(\dop_B) - S(\dop).
  \label{eq:relentropy}
\end{align}
Together with the Klein inequality which states that relative entropies are
non-negative (\cref{thm:klein}), we have that
\begin{equation}
  S(\dop)
  \leq S(\dop_A) + S(\dop_B),
  \label{eq:infoloss}
\end{equation}
with equality when $\dop = \dop_A \tp \dop_B$. Other expected properties
hold, such as that the relative entropy is invariant under unitary
transformations (\cref{thm:relSunitary}), or that the relative entropy
between subsystems is less than that between combined systems
(\cref{thm:relSsubsys}).

We would like to know the \term{reduced dynamics} of the quantum system $S$ when
in contact with an \term{environment} system $B$. We suppose that the composite
system has a Hamiltonian of the form
\begin{equation}
  \ham_{SB}(t)
  = \ham_S \tp \idopr
  + \idopr \tp \ham_B
  + \ham_I(t)
  \label{eq:fullham}
\end{equation}
and that the environment is in equilibrium, so the composite density operator is
$\dop(t) = \dop_S(t) \tp \dop_B$. In terms of the unitary time-evolution
operator $\opr{U}(t)$ for the system, \cref{eq:liouville} becomes $\dop(t) =
\opr{U}(t)\dop(0)\opr{U}^\dag(t)$. Taking the partial trace over the environment
gives the time-evolved system density operator
\begin{equation}
  \dop_S(t)
  = \tr_B\qty(\opr{U}(t)\qty(\dop_S(0) \tp \dop_B)\opr{U}^\dag(t)).
  \label{eq:evolveddop}
\end{equation}

Whatever \cref{eq:evolveddop} evaluates to, it will be an example of a
\term{dynamical map} $\sopr{V}(t)$ that time-evolves the system according to
$\dop_S(t) = \sopr{V}(t)\dop_S(0)$. While $\sopr{V}(t)$ seems abstract, we know
it should output a density operator. Then for $\sopr{V}(t)$ to be a valid map on
system density operators, it should \emph{preserve the trace} of the input
density operator. In fact, as a valid map $\sopr{V}(t) \tp \idsopr$ on the
composite system, the composite density operator should remain positive. This
property of $\sopr{V}(t)$ is called \term{complete positivity}. Thus the valid
maps on system density operators are \term{completely positive and
trace-preserving} (\textsc{\term{cptp}}).

\section{The Lindblad equation\label{sec:lindblad}}

With the idea of random interactions with an environment in mind, we will assume
that the maps $\qty{\sopr{V}(t) : t \geq 0}$ are also \emph{memoryless} or
\term{Markovian}, so that they form a \term{quantum dynamical semigroup}
satisfying
\begin{equation}
  \sopr{V}(t_1)\sopr{V}(t_2)
  = \sopr{V}(t_1 + t_2)
  \qfor t_1,\, t_2 \geq 0.
  \label{eq:semigroup}
\end{equation}
The action of the dynamical semigroup on the system describes an irreversible
process. As such, the relative entropy between an arbitrary system ensemble
$\dop(t)$ and an equilibrium ensemble $\dop_0$ cannot decrease
(by \cref{eq:evolveddop} and \cref{thm:relSunitary,thm:relSsubsys}):
\begin{align}
  S\qty(\sopr{V}(t)\dop \| \sopr{V}(t)\dop_0)
  &= S\qty(\tr_B\qty[\opr{U}(t)\qty(\dop \tp \dop_B)\opr{U}^\dag(t)] \| \dop_0) \\
  &\leq S\qty(\opr{U}(t)\qty(\dop \tp \dop_B)\opr{U}^\dag(t) \| \dop_0 \tp \dop_B) \\
  &= S\qty(\dop \tp \dop_B \| \dop_0 \tp \dop_B) \\
  &= S\qty(\dop \| \dop_0).
\end{align}

We would like to determine the \term{infinitesimal generator} $\liou$ for the
quantum dynamical semigroup which allows the dynamical maps to be expressed as
$\sopr{V}(t) = e^{\liou t}$, analogously to how a time-independent Hamiltonian
is a generator for the unitary time-evolution operator $e^{\ham t/\im\hbar}$.
Following this analogy, the Schr\"odinger equation is replaced by the
\term{Markovian quantum master equation} $\dot{\dop}_S = \liou\dop_S$, which
generalizes \cref{eq:liouville} to typically non-unitary \textsc{ctcp} maps of
density operators, provided that they are Markovian.

We will find that the most general form of $\liou$ is given by the
\term{Lindblad equation} \cref{eq:lindblad}. To obtain this result, first
consider diagonalizing $\dop_B$ as $\dop_B = \sum_j \lambda_j \op*{\phi_j}$ with
orthonormal vectors $\phi_j \in \hilb_B$, where $\sum_j\lambda_j = 1$.
Then \cref{eq:evolveddop} becomes (writing $\dop_S$ as $\dop$)
\begin{align}
  \dop(t)
  &= \sum_{ij} \mel{\phi_i}{\opr{U}(t)\qty(\dop(0) \tp
  \lambda_j\op*{\phi_j}{\phi_j})\opr{U}^\dag(t)}{\phi_i}
  \label{eq:krauspre} \\
  &= \sum_{ij} \lambda_j\mel{\phi_i}{\opr{U}(t)}{\phi_j}\dop(0)
  \mel{\phi_j}{\opr{U}^\dag(t)}{\phi_i}
  \label{eq:krauspost} \\
  &= \sum_{ij} \opr{M}_{ij}(t)\dop(0)\opr{M}_{ij}^\dag(t),
  \label{eq:krausrep}
\end{align}
where $\opr{M}_{ij}(t) \equiv \sqrt{\lambda_j} \mel*{\phi_i}{\opr{U}(t)}{\phi_j}$.
This decomposition in terms of the $\opr{M}_{ij}$ is an instance of the
Choi-Kraus representation theorem (\cref{thm:kraus}). We can express the
$\opr{M}_{ij}$ in terms of an orthonormal complete basis $\{\opr{F}_n\}$ for
$\liou(\hilb_S)$ as $\opr{M}_{ij} = \sum_k \opr{F}_k
\ip*{\opr{F}_k}{\opr{M}_{ij}}$. Then \cref{eq:krausrep} becomes
\begin{equation}
  \dop(t)
  = \sum_{mn} c_{mn}(t) \opr{F}_m\dop(0)\opr{F}_n^\dag,
  \label{eq:Fintro}
\end{equation}
where
\begin{equation}
  c_{mn}(t)
  \equiv \sum_{ij} \ip{\opr{F}_m}{\opr{M}_{ij}(t)}\ip{\opr{M}_{ij}(t)}{\opr{F}_n}.
  \label{eq:coeffs}
\end{equation}
For convenience, we may choose $\opr{F}_{d^2} = \idopr / \sqrt{d}$, where $d =
\dim(\hilb_S)$. With an eye towards simplifying \cref{eq:Vderiv}, we eliminate
the explicit time dependence of \cref{eq:coeffs} by defining
\begin{equation}
  a_{mn}
  \equiv \lim_{t \to 0^+} \frac{c_{mn}(t) - d\delta_{d^2 d^2}}{t}
  \label{eq:limcoeffs}
\end{equation}
and introduce the sum of Kraus operators
\begin{align}
  \opr{F}
  &= \frac{1}{\sqrt{d}}\sum_{n=1}^{d^2 - 1} a_{n d^2} \opr{F}_n
  \label{eq:kraussum} \\
  &= \frac{\opr{F} + \opr{F}^\dag}{2}
  + \im \frac{\opr{F} - \opr{F}^\dag}{2\im}
  \equiv \opr{G} + \flatfrac{\ham}{\im\hbar}\!,
  \label{eq:kraussherm}
\end{align}
where we have decomposed the sum $\opr{F}$ into Hermitian and anti-Hermitian
parts and included $\hbar$ so that $\ham$ will have dimensions of energy. Now we
may write the master equation $\liou\dop = \dot{\dop}$ as
\begin{align}
  \dot{\dop}
  &= \lim_{\Delta t \to 0^+} \frac{\sopr{V}(\Delta t)\dop - \dop}{\Delta t}
  \nonumber \\
  %
  &= \lim_{\Delta t \to 0^+} \Bigg(
  \frac{c_{d^2 d^2} - d}{d\Delta t} \dop
  + \sum_{m,n=1}^{d^2 - 1}
  \frac{c_{mn}(\Delta t)}{\Delta t}\opr{F}_m \dop \opr{F}_n^\dag
  \nonumber \\
  &+ \frac{1}{\sqrt{d}} \sum_{n=1}^{d^2 - 1} \qty(%
  \frac{c_{nd^2}(\Delta t)}{\Delta t} \opr{F}_n \dop
  + \frac{c_{d^2 n}(\Delta t)}{\Delta t} \dop \opr{F}_n^\dag)
  \Bigg) \label{eq:Vderiv} \\
  %
  &= \frac{a_{d^2 d^2}}{d} \dop 
  + \opr{F}\dop + \dop\opr{F}^\dag
  + \sum_{m,n=1}^{d^2 - 1} a_{mn} \opr{F}_m \dop \opr{F}_n^\dag
  \nonumber \\
  %
  &= \frac{a_{d^2 d^2}}{d} \dop 
  + \acomm{\opr{G}}{\dop}
  + \frac{\comm{\ham}{\dop}}{\im\hbar}
  + \sum_{m,n=1}^{d^2 - 1} a_{mn} \opr{F}_m \dop \opr{F}_n^\dag
  \nonumber \\
  %
  &= \acomm{\opr{G}'}{\dop}
  + \frac{\comm{\ham}{\dop}}{\im\hbar}
  + \sum_{m,n=1}^{d^2 - 1} a_{mn} \opr{F}_m \dop \opr{F}_n^\dag,
  \label{eq:GHform}
\end{align}
where $\opr{G}' = \opr{G} + \flatfrac{a_{d^2 d^2} \idopr}{d}$. Since
$\sopr{V}(t)$ is trace-preserving, $\tr\dot{\dop} = 0$. Applying this
condition to \cref{eq:GHform} and cycling the trace gives
\[
  0
  = \tr\qty(2\opr{G}'\dop
  + \sum_{m,n=1}^{d^2 - 1} a_{mn} \opr{F}_n^\dag \opr{F}_m \dop)\mkern1mu,
\]
so $\opr{G}' = -\sum_{m,n=1}^{d^2 - 1} a_{mn} \opr{F}_n^\dag\opr{F}_m / 2$. This
allows us to write \cref{eq:GHform} as
\begin{equation}
  \dot{\dop}
  = \frac{\comm{\ham}{\dop}}{\im\hbar}
  + \sum_{m,n=1}^{d^2 - 1} a_{mn}\qty(\opr{F}_m\dop\opr{F}_n^\dag -
  \frac{1}{2}\acomm{\opr{F}_n^\dag \opr{F}_m}{\dop}),
  \label{eq:markovian}
\end{equation}
which is the first form of the \emph{Lindblad equation}. This may be simplified
further if we diagonalize the coefficient matrix $\opr{a}$ by applying a unitary
transformation $\opr{u}$ to give $\opr{a} = \opr{u}\opr{γ}\opr{u}^\dag$, where
the $\qty{\gamma_k}_{k=1}^{d^2 - 1}$ are the non-negative eigenvalues of
$\opr{a}$. This is possible since the coefficient matrix $\opr{c}$ is seen
from \cref{eq:coeffs} to be Hermitian, and \cref{eq:limcoeffs} then gives that
$\opr{a}$ is Hermitian. We may then express $\opr{F}_{n \neq d^2} =
\sum_{k=1}^{d^2 - 1} \opr{L}_n u_{nk}$ in terms of the \term{Lindblad operators}
$\opr{L}_n$ to find
\begin{equation}
  \dot{\dop}
  = \frac{\comm{\ham}{\dop}}{\im\hbar}
  + \sum_{k=1}^{d^2 - 1} \gamma_k \qty(\opr{L}_k\dop\opr{L}_k^\dag -
  \frac{1}{2}\acomm{\opr{L}_k^\dag \opr{L}_k}{\dop})
  \equiv
  \sopr{L}\dop,
  \label{eq:lindblad}
\end{equation}
which is the \emph{diagonal form} of the Lindblad equation. The eigenvalues
$\gamma_k$ have dimensions of inverse time and the Lindblad operators may be
taken to be traceless. The second term is often called the \term{dissipator}
$\sopr{D}$ (see \cref{sec:interaction}), so the Lindbladian may be separated
into unitary and non-unitary parts.

\section{The weak-coupling limit\label{sec:interaction}}

Now that we have found the general form for a stochastic \textsc{ctcp}
generator, we must now determine the conditions for interaction Hamiltonian
in \cref{eq:fullham} to give rise to Markovian dynamics. While there are
several different regimes where this is true, we will consider the
\term{weak-coupling} limit which we justify by supposing that the environment is
similar to a \term{harmonic bath} of many harmonic oscillators.

We start by expressing the interaction Hamiltonian in terms of Hermitian
operators as
\[
  \ham_I
  = \sum_\alpha \opr{A}_\alpha \tp \opr{B}_\alpha.
\]
We suppose that the system in isolation would have \emph{discrete} energy
levels, so the eigenoperators of the superoperator $\sopr{S} = \comm{\ham_S}{-}$
form a complete basis for $\liou(\hilb_S)$. We then may write $\opr{A}_\alpha =
\sum_\omega \opr{A}_{\alpha\omega}$, where
\begin{equation}
  \comm{\ham_S}{\opr{A}_{\alpha\omega}}
  = -\omega\opr{A}_{\alpha\omega}.
  \label{eq:Aomega}
\end{equation}
Using \cref{eq:Aomega} to commute past the exponential in \cref{eq:intobs}
gives $\opr{A}_{\alpha\omega}' = e^{-\im\omega t} \opr{A}_{\alpha\omega}$ in the
interaction picture. Thus the interaction Hamiltonian in the interaction picture
is
\begin{equation}
  \ham_I'
  = \sum_{\alpha\omega} e^{-\im\omega t} \opr{A}_{\alpha\omega}
  \tp \opr{B}_\alpha',
  \label{eq:inthamdecomp}
\end{equation}
where $\opr{B}_\alpha'(t) = e^{-\ham_B t/\im\hbar} \opr{B}_\alpha e^{\ham_B t/
\im\hbar}$ per \cref{eq:intobs}.

Since we are interested in how fluctuations in different environment modes are
related, we will consider the \term{reservoir correlation functions}
\begin{equation}
  \sensavg[\dop_B]{\opr{B}_\alpha^\dag(t)\opr{B}_\beta(t - s)}
  \label{eq:corr}
\end{equation}
and their one-sided Fourier transform
\begin{align}
  \Gamma_{\alpha\beta}(\omega)
  &\equiv \int_0^\infty \dd{s} e^{\im\omega s}
  \ensavg[\dop_B]{\opr{B}_\alpha^\dag(t)\opr{B}_\beta(t - s)}
  \label{eq:onegammas} \\
  &\equiv \im S_{\alpha\beta}(\omega) +
  \flatfrac{\gamma_{\alpha\beta}(\omega)}{2}\!,
  \label{eq:gammadecomp}
\end{align}
where the corresponding matrix $\opr{S} = (\opr{Γ} - \opr{Γ}^\dag) / 2\im$ is
Hermitian and the matrix corresponding to the full Fourier transform
\begin{equation}
  \gamma_{\alpha\beta}(\omega)
  \equiv \int_{-\infty}^\infty \dd{s} e^{\im\omega s}
  \ensavg[\dop_B]{\opr{B}_\alpha^\dag(t)\opr{B}_\beta(t - s)}
  \label{eq:gammas}
\end{equation}
is positive.

With this setup, we may now move to the main derivation. It is helpful to
consider the interaction picture time evolution \cref{eq:intliou} in the
integral form
\[
  \dop(t)
  = \dop(0) - \im\int_0^t \dd{s} \comm{\ham_I(s)}{\dop(s)}.
\]
Applying \cref{eq:intliou} again and tracing out the environment gives the
closed equation
\[
  \dot{\dop}_S(t)
  = -\int_0^t \dd{s}
  \tr_B\comm{\ham_I(t)}{\comm{\ham_I(s)}{\dop_S(s) \tp \dop_B}}
\]
for the system density operator. In doing so we have made two assumptions: that
\[
  \tr_B\comm{\ham_I(t)}{\dop(0)}
  = 0,
\]
which is the \term{weak-coupling approximation}, and that
\[
  \dop(t)
  = \dop_S(t) \tp \dop_B,
\]
which is the \term{Born approximation}. It should be noted that weak-coupling
follows if the reservoir averages of the interactions vanish:
$\ensavg[\dop_B]{\opr{B}_\alpha(t)} = 0$.

We now make the \term{Markov approximation} that $\dop_S(s) = \dop_S(t)$, so
that the time-evolution only depends on the present time, to obtain the
\term{Redfield equation}. To simplify further, we make the substitution $s
\mapsto t - s$ and set the upper limit of the integral to infinity:
\begin{equation}
  \dot{\dop}_S
  = -\int_0^\infty \dd{s}
  \tr_B\comm{\ham_I(t)}{\comm{\ham_I(t - s)}{\dop_S(t) \tp \dop_B}}.
  \label{eq:infbornmarkov}
\end{equation}
This is justified when the reservoir correlation functions
in \cref{eq:onegammas} vanish quickly over a time $\tau_B$ that is smaller than
the relaxation time $\tau_R$ (see \cref{sec:limitations}).
Substituting \cref{eq:inthamdecomp} into \cref{eq:infbornmarkov} and
using \cref{eq:onegammas} gives
\begin{equation}
  \dot{\dop}_S
  = 2\herm \sum_{\alpha\beta\omega\omega'}
  e^{\im(\omega' - \omega)t}
  \Gamma_{\alpha\beta}(\omega) \qty(%
  \opr{A}_{\beta\omega} \dop_S \opr{A}_{\alpha\omega'}^\dag
  - \opr{A}_{\alpha\omega'}^\dag \opr{A}_{\beta\omega} \dop_S),
  \label{eq:decompsub}
\end{equation}
where $\herm\opr{Γ} \equiv (\opr{Γ} + \opr{Γ}^\dag) / 2$. If the typical times
\[
  \tau_S = \abs{\omega' - \omega}^{-1}
  \qfor \omega' \ne \omega
\]
for system evolution are large compared to the relaxation time $\tau_R$, then
the contribution from the fast-oscillating terms of \cref{eq:decompsub} where
$\omega' \ne \omega$ may be neglected. This \term{rotating wave} or
\term{secular approximation} is analogous to how we consider the high-energy
position distribution in the infinite square well to be uniform, even though it
is actually a fast-oscillating function. By coarse-graining in this sense, we
obtain
\begin{equation}
  \dot{\dop}_S
  = 2\herm \sum_{\alpha\beta\omega}
  \Gamma_{\alpha\beta}(\omega) \qty(%
  \opr{A}_{\beta\omega} \dop_S \opr{A}_{\alpha\omega}^\dag
  - \opr{A}_{\alpha\omega}^\dag \opr{A}_{\beta\omega} \dop_S).
  \label{eq:rotwave}
\end{equation}
Now applying the decomposition \cref{eq:gammadecomp} gives the interaction
picture Lindblad equation
\begin{equation}
  \dot{\dop}_S
  = \im\comm{\ham_{LS}}{\dop_S} + \sopr{D}\dop_S,
  \label{eq:microlindblad}
\end{equation}
where the \term{Lamb shift Hamiltonian} is
\begin{equation}
  \ham_{LS}
  = \sum_{\alpha\beta\omega}
  S_{\alpha\beta}(\omega) \opr{A}_{\alpha\omega}^\dag \opr{A}_{\beta\omega},
  \label{eq:lambham}
\end{equation}
and the \emph{dissipator} is
\begin{equation}
  \sopr{D}\dop_S
  = \sum_{\alpha\beta\omega} \gamma_{\alpha\beta} \qty(%
  \opr{A}_{\beta\omega} \dop_S \opr{A}_{\alpha\omega}^\dag
  - \frac{1}{2} \acomm{\opr{A}_{\alpha\omega}^\dag
  \opr{A}_{\beta\omega}}{\dop_S}).
  \label{eq:dissipator}
\end{equation}
The Lamb shift (or environment renormalization) Hamiltonian commutes with the
system Hamiltonian since \cref{eq:Aomega} implies that
$\comm*{\ham_S}{\opr{A}_{\alpha\omega}^\dag \opr{A}_{\beta\omega}} = 0$. Adding
the system's Hamiltonian $\ham_S$ to $\ham_{LS}$ and diagonalizing gives the
Schr\"odinger picture Lindblad equation \cref{eq:lindblad}.

\section{Relaxation to thermal equilibrium\label{sec:thermo}}

The system will generally relax from its initial configuration to a stationary
solution of \cref{eq:lindblad} (see \cref{sec:limitations}). We expect that
the thermal state
\[
  \dop_S
  = \frac{e^{-\beta\ham_S}}{Z}
  \qq{where}
  Z
  = \tr(e^{-\beta\ham_S})
\]
would be the equilibrium state. This is true when the reservoir correlation
functions obey the \textsc{kms} condition~\cite{kubo,martinschwinger}
\begin{equation}
  \ensavg[\dop_B]{\opr{B}_\alpha^\dag(t) \opr{B}_\beta(0)}
  = \ensavg[\dop_B]{\opr{B}_\beta(0) \opr{B}_\alpha^\dag(t + \im\beta)},
  \label{eq:kms}
\end{equation}
which is true when the environment is in the thermal state $\dop_B =
{e^{-\beta\ham_B}} / {\tr(e^{-\beta\ham_B})}$.

\section{A two-level atom\label{sec:example}}

To demonstrate the use of the Lindblad equation, we will study a model for the
decay of a two-level atom. Our aim is to glimpse why electrons in atoms undergo
optical decay, even though excited states are stable atomic states. Suppose that
the atom has Hamiltonian $\ham_S = \hbar\omega\opr{σ}_3 / 2$, where $\opr{σ}_3 =
\op{1} - \op{0}$. The operators $\opr{σ}_- = \op{0}{1}$ and $\opr{σ}_+ =
\op{1}{0}$ are Lindblad operators, since they are eigenoperators of the
superoperator $\comm{\ham_S}{-}$, like in \cref{eq:Aomega}. These correspond to
lowering and raising the energy by $\hbar\omega$, and will be our analogues of
the emission and absorption processes. The derivation of \cref{sec:lindblad}
is similar for a bath of photons in equilibrium, and our assumptions are
justified because typical atomic relaxation times of about \SI{20}{\ns} are much
slower than the periods of electromagnetic waves~\cite{steck}. Ignoring the Lamb
shift (which only offsets) and considering only the effects at
$\omega$, \cref{eq:microlindblad} becomes
\begin{align}
  \dot{\dop}
  &= \gamma_0(N+1)\qty(\opr{σ}_- \dop \opr{σ}_+
  - \frac{1}{2}\acomm{\opr{σ}_+\opr{σ}_-}{\dop}) \nonumber \\
  &+ \gamma_0 N\qty(\opr{σ}_+ \dop \opr{σ}_-
  - \frac{1}{2}\acomm{\opr{σ}_-\opr{σ}_+}{\dop})
  \equiv \sopr{D}\dop,
  \label{eq:twomaster}
\end{align}
where $N = 1 / (e^{\beta\hbar\omega} - 1)$. This is straightforward to solve
given the properties of the Pauli matrices. From the initial density operator
$\dop(0) = \op{1}$, we find that the population of upper level is
\[
  \dop_{11}(t)
  = \frac{N}{2N + 1}\qty(1 - e^{-\gamma t}), \qq{where}
  \gamma = \gamma_0(2N + 1).
\]
This is consistent with what we observe in atomic spectra: an exponential decay
to an equilibrium level which gives Lorentzian peaks. At low temperatures ($N
\to 0$) the system approaches the ground state in accordance with the third law
of thermodynamics, and at high temperatures ($N \gg 1$), the level is
half-occupied and the absorption is saturated~\cite{rubidium}
(\textit{Cf}.~\footnote{It's cool how this provides a fundamental explanation of
  the \textit{Doppler-free saturated absorption spectroscopy of Rubidium vapor}
\textsc{JLab} experiment that I did.}).

\section{Limitations\label{sec:limitations}}

Though the Lindblad equation is widely applicable, there are some situations in
which key assumptions in its derivation break down. For one, we have glossed
over the issue of \emph{ergodicity} in considering a harmonic bath. Since we
needed to assume a discrete spectrum, the correlation functions \cref{eq:corr}
will be quasi-periodic and will not decay as we required. It is only in the
limit of a reservoir with infinitely many degrees of freedom that we expect
non-periodic behavior like decay to emerge, but then there may be issues with
unbounded operators. This can lead to the interesting behavior of spontaneous
symmetry breaking and phase transitions.

There are also many systems with dynamics that occur on time scales comparable
to the relaxation time. For example, a paper published less than a month ago (!)
demonstrates how fast pulsed laser experiments can probe the relaxation of
temporarily polarized gas molecules due to
collisions~\cite{collisions1,collisions2}.

\section{Conclusion}

We have seen how the general consideration of Markovian \textsc{ctcp} maps on
density operators leads to the Lindblad equation, and considered the
weak-coupling limit as an example of a physical regime where the assumption of
stochastic dynamics is valid. However, we have only scratched the surface of
what can be done with the Lindblad equation, especially with respect to solving
it. Since the two-level atom is a small system, it is simple to diagonalize, but
larger systems provide more difficulty as the dimension of the Hilbert space
grows. Numerical solutions are complicated by the additional requirement of
trace preservation, but they can still be done in many
situations~\cite{intro,collisions1}. The theory of open quantum systems gives
some fundamental justifications for the assumptions of equilibrium statistical
mechanics, as was briefly noted in \cref{sec:thermo}, and has made the
picture of decoherence a bit more clear.

\section{Mathematical details\label{sec:math}}

\begin{defn}[Tensor product]\label{def:tensors}
  Consider vector spaces $V(k)$, $W(k)$, and $Z$. For any bilinear map $h:V
  \times W \to Z$, the \term{tensor product} $V \tp W$ and associated
  bilinear map $\phi:V \times W \to V \tp W$ map have the property that there
  is a unique linear map $g:V \tp W \to Z$ such that $h = g \circ \phi$. For
  tensor products of Hilbert spaces, the inner product is defined on each
  element of a product and then the space is completed. There is then a natural
  correspondence between the element $v \tp f$ of the tensor product $V \tp V^*$
  and the linear map $T:V \to V$ defined by $Tx = f(x)v$.

  This induces an extension of Dirac notation where all pairs $f \tp x$ of
  dual and usual vectors from the same space are evaluated as $\ip{f}{x} = f(x)$
  and extended linearly. For example, given a linear operator $\opr{U}:V \tp
  W \to V \tp W$ and a basis $\ket{\phi_i}$ for $W$, the partial trace over
  $W$ may be expressed as $\tr_W\opr{U} = \ev{\opr{U}}{\phi_i}$. This forms the
  justification of the step from \cref{eq:krauspre} to \cref{eq:krauspost} and
  of the manipulations in \cref{thm:kraus}.
\end{defn}

\begin{thm}[Klein inequality]\label{thm:klein}
  For density operators $\dop$ and $\dop'$, $S(\dop \| \dop') \geq 0$, with
  equality if and only if $\dop = \dop'$.
  \begin{proof}
    The case for equality is trivial, so we will consider $\dop \neq \dop'$. Let
    $\sopr{F}\dop = \dop\ln\dop$, so that we may express the relative entropy as
    \[
      S(\dop \| \dop')
      = \tr(\sopr{F}\dop - \sopr{F}\dop' - \opr{δ}\sopr{F}'\dop'),
    \]
    where $\opr{δ} = \dop - \dop'$. We then have for $0 < t < 1$ that
    \[
      \dop' + t\opr{δ}
      = t\dop + (1 - t)\dop'.
    \]
    Now let $f(t) = \tr\qty(\sopr{F}(\dop' + t\opr{δ}))$. Since the trace is
    monotonic and convex, $f$ is convex and $f(t) \leq f(0) + t\qty(f(1) -
    f(0))$. Rearranging and taking the limit as $t \to 0^+$ gives
    \[
      f'(0)
      \leq f(1) - f(0),
    \]
    which evaluates to
    \[
      \tr\qty(\opr{δ}\sopr{F}'\dop')
      \leq \tr\sopr{F}\dop - \tr\sopr{F}\dop'.
      \qedhere
    \]
  \end{proof}
\end{thm}

\begin{thm}\label{thm:relSunitary}
  For a unitary operator $\opr{U}$ and density operators $\dop$ and $\dop'$,
  \[
    S(\opr{U}\dop\opr{U}^\dag \| \opr{U}\dop'\opr{U}^\dag)
    = S(\dop \| \dop').
  \]
  \begin{proof}
    Since we may cycle the traces, it suffices to show that
    \[
      \ln\qty(\opr{U}\dop\opr{U}^\dag)
      = \ln\dop.
    \]
    This follows from Jacobi's formula for invertible matrices when
    applied to the logarithm that takes us from a Lie group to its corresponding
    Lie algebra, giving $\tr \circ \det = \tr \circ \log$.
  \end{proof}
\end{thm}

\begin{thm}\label{thm:relSsubsys}
  For density operators $\dop$ and $\dop'$,
  \[
    S(\tr_B\dop \| \tr_B\dop')
    \leq S(\dop \| \dop'),
  \]
  with equality if and only if $\dop$ or $\dop'$ is uncorrelated.
\end{thm}

\begin{thm}[Choi-Kraus representation~\cite{intro}]\label{thm:kraus}
  A superoperator $\sopr{S}$ on a density operator $\dop$ is completely positive
  and trace-preserving if and only if it may be represented as
  \[
    \sopr\dop
    = \sum_{k=1}^K \opr{M}_k \dop \opr{M}_k^\dag,
    \qq{where}
    \sum_{k=1}^K \opr{M}_k \opr{M}_k^\dag
    = \idopr.
  \]
\end{thm}

\section{The Born-Markov approximation for the Ising chain in a
bath\label{sec:bm-ising}}

The bath Hamiltonian is
\begin{equation}
  \ham_B
  = \sum_{\vq{k},\, \lambda} \hbar\omega_{\vq{k}}
  \opr{a}^\dag_{\vq{k},\lambda}\opr{a}_{\vq{k},\lambda}.
  \label{eq:bath-hamiltonian}
\end{equation}
The vacuum energy $\sum_{\vq{k},\, \lambda} \hbar\omega_{\vq{k}} / 2$ is
dropped, since it diverges in the continuum limit.

The interaction Hamiltonian for spin-$s$ objects in a magnetic field
is\footnote{The time dependence of the field is absorbed into the operators
$\opr{a}_{\vq{k},\lambda}$, and the prefactor is chosen so that these operators
are dimensionless, but $\vecopr{B}$ is not.}
\begin{align}
  \ham_I
  &= -\int\dd{\vq{r}} \vecopr{\mu} \vdot \vecopr{B}
  \\
  \begin{split}
  &= -\int\dd{\vq{r}} \sum_i m_s g_s \mu_B \delta(\vq{r}_i) \vecopr{\pauli}_i \\
  &\vdot
  \im\sum_{\vq{k},\, \lambda} \sqrt{\frac{\hbar c^2 \mu_0}{2V\omega_{\vq{k}}}}
  \qty(%
  \qty(\vq{k} \cross \vq{e}_{\vq{k},\lambda})
  e^{\im\vq{k}\vdot\vq{r}} \opr{a}_{\vq{k},\lambda}
  - \qty(\vq{k} \cross \vq{e}^*_{\vq{k},\lambda})
  e^{-\im\vq{k}\vdot\vq{r}} \opr{a}^\dag_{\vq{k},\lambda}
  )
  \end{split}
  \\
  &= -\sum_{i,\,\mu} m_s g_s \mu_B \pauli_i^\mu \opr{B}_i^\mu,
\end{align}
where we have defined the Hermitian operator
\begin{equation}
  \opr{B}_i^\mu
  = \im\sum_{\vq{k},\, \lambda} \sqrt{\frac{\hbar c^2 \mu_0}{2V\omega_{\vq{k}}}}
  \qty(%
  {\qty(\vq{k} \cross \vq{e}_{\vq{k},\lambda})}_\mu
  e^{\im\vq{k}\vdot\vq{r}_i} \opr{a}_{\vq{k},\lambda}
  - {\qty(\vq{k} \cross \vq{e}^*_{\vq{k},\lambda})}_\mu
  e^{-\im\vq{k}\vdot\vq{r}_i} \opr{a}^\dag_{\vq{k},\lambda}
  ).
  \label{eq:Bimu-schrodinger}
\end{equation}
In the interaction picture:
\begin{align}
  \opr{B}_i^\mu(t)
  &= e^{\im\ham_B t/\hbar} \opr{B}_i^\mu e^{-\im\ham_B t/\hbar} \\
  &= \im\sum_{\vq{k},\, \lambda} \sqrt{\frac{\hbar c^2 \mu_0}{2V\omega_{\vq{k}}}}
  \qty(%
  {\qty(\vq{k} \cross \vq{e}_{\vq{k},\lambda})}_\mu
  e^{\im(\vq{k}\vdot\vq{r}_i - \omega_{\vq{k}} t)} \opr{a}_{\vq{k},\lambda}
  - {\qty(\vq{k} \cross \vq{e}^*_{\vq{k},\lambda})}_\mu
  e^{-\im(\vq{k}\vdot\vq{r}_i - \omega_{\vq{k}} t)} \opr{a}^\dag_{\vq{k},\lambda}
  ).
  \label{eq:Bimu-interaction}
\end{align}

The spectral correlation tensor is then
\begin{align}
  \Gamma_{i\mu,j\nu}(\omega)
  &= \frac{1}{\hbar^2} \int_0^\infty \dd{s} e^{\im\omega s}
  \ev{\opr{B}_i^\mu{(t)}^\dag \opr{B}_j^\nu(t - s)}
  \\
  \begin{split}
  &= -\frac{1}{\hbar^2} \frac{\hbar c^2 \mu_0}{2V} \int_0^\infty \dd{s}
  \sum_{\vq{k},\, \vq{k}',\, \lambda,\, \lambda'}
  \sqrt{\frac{1}{\omega_{\vq{k}} \omega_{\vq{k'}}}}
  :
  \\
  &{\qty(\vq{k} \cross \vq{e}_{\vq{k},\lambda})}_\mu
    {\qty(\vq{k}' \cross \vq{e}_{\vq{k}',\lambda'})}_\nu
    e^{\im(\vq{k}\vdot\vq{r}_i - \omega_{\vq{k}} t + \vq{k}'\vdot\vq{r}_j -
    \omega_{\vq{k}'} (t-s) + \omega s)}
  \ev{%
    \opr{a}_{\vq{k},\lambda}\opr{a}_{\vq{k}',\lambda'}
  } \\
  &-
    {\qty(\vq{k} \cross \vq{e}_{\vq{k},\lambda})}_\mu
    {\qty(\vq{k}' \cross \vq{e}^*_{\vq{k}',\lambda'})}_\nu
    e^{\im(\vq{k}\vdot\vq{r}_i - \omega_{\vq{k}} t - \vq{k}'\vdot\vq{r}_j +
    \omega_{\vq{k}'} (t-s) + \omega s)}
  \ev{%
    \opr{a}_{\vq{k},\lambda}\opr{a}^\dag_{\vq{k}',\lambda'}
  } \\
  &-
    {\qty(\vq{k} \cross \vq{e}^*_{\vq{k},\lambda})}_\mu
    {\qty(\vq{k}' \cross \vq{e}_{\vq{k}',\lambda'})}_\nu
    e^{-\im(\vq{k}\vdot\vq{r}_i - \omega_{\vq{k}} t - \vq{k}'\vdot\vq{r}_j +
    \omega_{\vq{k}'} (t-s) - \omega s)}
  \ev{%
    \opr{a}^\dag_{\vq{k},\lambda}\opr{a}_{\vq{k}',\lambda'}
  } \\
  &+
    {\qty(\vq{k} \cross \vq{e}^*_{\vq{k},\lambda})}_\mu
    {\qty(\vq{k} \cross \vq{e}^*_{\vq{k}',\lambda'})}_\nu
    e^{-\im(\vq{k}\vdot\vq{r}_i - \omega_{\vq{k}} t + \vq{k}'\vdot\vq{r}_j -
    \omega_{\vq{k}'} (t-s) - \omega s)}
  \ev{%
    \opr{a}^\dag_{\vq{k},\lambda}\opr{a}^\dag_{\vq{k}',\lambda'}
  }.
\end{split}
\end{align}

In the thermal state
\begin{equation}
  \dop_B
  = \frac{e^{-\beta\ham_B}}{\tr e^{-\beta\ham_B}}
  = \prod_{\vq{k},\,\lambda} \qty(1 - e^{-\beta\hbar\omega_{\vq{k}}})
  e^{-\beta\hbar\omega_{\vq{k}} \opr{a}^\dag_{\vq{k},\lambda} \opr{a}_{\vq{k},\lambda}}
\end{equation}
Since $\comm{\opr{a}_{\vq{k},\lambda}}{\opr{a}^\dag_{\vq{k}',\lambda'}} =
\delta_{\vq{k},\vq{k}'}\delta_{\lambda\lambda'} \idopr$,
\begin{align}
  \ev{\opr{a}^\dag_{\vq{k},\lambda} \opr{a}_{\vq{k}',\lambda'}}
  &= {\tr(e^{-\beta\ham_B})}^{-1} \tr(e^{-\beta\ham_B}
  \opr{a}^\dag_{\vq{k},\lambda} \opr{a}_{\vq{k}',\lambda'})
  \\
  &= {\tr(e^{-\beta\ham_B})}^{-1} \tr(e^{-\beta\ham_B}
  \opr{a}_{\vq{k}',\lambda'} \opr{a}^\dag_{\vq{k},\lambda})
  - \delta_{\vq{k},\vq{k}'}\delta_{\lambda\lambda'}
  \\
  &= {\tr(e^{-\beta\ham_B})}^{-1} \tr(
  e^{\beta\hbar \omega_{\vq{k}}}
  \opr{a}_{\vq{k}',\lambda'} e^{-\beta\ham_B} \opr{a}^\dag_{\vq{k},\lambda})
  - \delta_{\vq{k},\vq{k}'}\delta_{\lambda\lambda'}
  \\
  &= e^{\beta\hbar \omega_{\vq{k}}}
  \ev{\opr{a}^\dag_{\vq{k},\lambda} \opr{a}_{\vq{k}',\lambda'}}
  - \delta_{\vq{k},\vq{k}'}\delta_{\lambda\lambda'}
  \\
  &= \delta_{\vq{k},\vq{k}'}\delta_{\lambda\lambda'} n_B(\omega_{\vq{k}}).
\end{align}
Similarly,
\begin{align}
  \ev{\opr{a}_{\vq{k},\lambda} \opr{a}_{\vq{k}',\lambda'}}
  &= \ev{\opr{a}^\dag_{\vq{k},\lambda} \opr{a}^\dag_{\vq{k}',\lambda'}}
  = 0
  \\
  \ev{\opr{a}_{\vq{k},\lambda} \opr{a}^\dag_{\vq{k}',\lambda'}}
  &= \delta_{\vq{k}\vq{k}'} \delta_{\lambda\lambda'}
  \qty(1 + n_B(\omega_{\vq{k}})).
\end{align}
Then for a thermal bath, the spectral correlation tensor becomes
\begin{align}
  \begin{split}
  \Gamma_{i\mu,j\nu}(\omega)
  &= \frac{c^2 \mu_0}{2\hbar V} \int_0^\infty \dd{s}
  \sum_{\vq{k},\,\lambda}
  \frac{1}{\omega_{\vq{k}}}
  :
  \\
  &{\qty(\vq{k} \cross \vq{e}_{\vq{k},\lambda})}_\mu
    {\qty(\vq{k} \cross \vq{e}^*_{\vq{k},\lambda})}_\nu
    e^{\im(\vq{k}\vdot(\vq{r}_i - \vq{r}_j) + s(\omega - \omega_{\vq{k}}))}
    \qty(1 + n_B(\omega_{\vq{k}}))
  \\
  &+
    {\qty(\vq{k} \cross \vq{e}^*_{\vq{k},\lambda})}_\mu
    {\qty(\vq{k} \cross \vq{e}_{\vq{k},\lambda})}_\nu
    e^{-\im(\vq{k}\vdot(\vq{r}_i - \vq{r}_j) - s(\omega + \omega_{\vq{k}}))}
    n_B(\omega_{\vq{k}}).
\end{split}\label{eq:Gamma-precontinuum}
\end{align}
\todo[inline]{%
  $\vq{k}$ in exponent of original integral. Is there a reason why the
  external field $\vq{k}$ would be the same as a reciprocal lattice vector? Like
  optical phonon branch. Magnons? Just no absorption for off frequencies? Or do
  typical scales enforce limits?
}

To evaluate \cref{eq:Gamma-precontinuum}, we now consider a chain of $N$ spins
along the $z$-axis, so that $\vq{r}_i = r_i \vu{z}$.\footnote{We could consider
any axis given the spherical symmetry, but the $z$-axis is the simplest to
evaluate.} Then $\vq{k} \vdot (\vq{r}_i - \vq{r}_j) = k_z \Delta r_{ij}$.

In the continuum limit,
\begin{equation}
  \frac{1}{V}\sum_{\vq{k}}
  \mapsto
  \int \frac{\dd{\vq{k}}}{{(2\pi)}^3}
  = \frac{1}{{(2\pi c)}^3} \int_0^\infty \dd{\omega_k} \omega_k^2
  \int\dd{\Omega},
\end{equation}
where the integral over solid angle is
\begin{equation}
  \int\dd{\Omega}
  = \int\dd{\phi}\int\dd{\theta}\sin\theta.
  \label{eq:solid-angle}
\end{equation}
To apply this limit to \cref{eq:Gamma-precontinuum}, we first note that
\begin{align}
  \sum_\lambda {\qty(\vq{k} \cross \vq{e}_{\vq{k},\lambda})}_\mu
  {\qty(\vq{k} \cross \vq{e}^*_{\vq{k},\lambda})}_\nu
  &= \sum_{abcd} \varepsilon_{\mu ab}\varepsilon_{\nu cd} k^a k^c 
  \sum_\lambda e_{\vq{k},\lambda}^b {\qty(e_{\vq{k},\lambda}^d)}^*
  \\
  &= \sum_{abcd} \varepsilon_{\mu ab}\varepsilon_{\nu cd} k^a k^c 
  \qty(\delta_{bd} - \frac{k^b k^d}{k^2})
  \\
  &= \sum_{abc} \varepsilon_{\mu ab}\varepsilon_{\nu cb} k^a k^c 
  \\
  &= \sum_{ac} \qty(\delta_{\mu\nu}\delta_{ac} - \delta_{\mu c}\delta_{a\nu}) k^a k^c 
  \\
  &= k^2 \delta_{\mu\nu} - k^\mu k^\nu.
\end{align}
Thus
\begin{equation}
  \int\dd{\Omega} e^{\pm\im k_z \Delta r_{ij}}
  \sum_\lambda {\qty(\vq{k} \cross \vq{e}_{\vq{k},\lambda})}_\mu
  {\qty(\vq{k} \cross \vq{e}^*_{\vq{k},\lambda})}_\nu
  = \frac{8\pi\omega_k^2}{3c^2} \delta_{\mu\nu}
  \Omega_{\mu\nu}\qty(\frac{\omega_k\Delta r_{ij}}{c}),
  \label{eq:angular-integral}
\end{equation}
where
\begin{equation}
  \Omega_{\mu\nu}(u)
  = \qty(\delta_{\nu z} - \frac{\delta_{\nu x} + \delta_{\nu y}}{2})
  \frac{\sinc u - \cos u}{u^2}
  + \frac{\delta_{\nu x} + \delta_{\nu y}}{2} \sinc u.
\end{equation}
Now \cref{eq:angular-integral} gives that the continuum limit of the spectral
correlation tensor for the spin chain is
\begin{align}
  \begin{split}
  \Gamma_{i\mu,j\nu}(\omega)
  &= \delta_{\mu\nu} \frac{\mu_0}{6\pi^2 \hbar c^3}
  \int_0^\infty \dd{\omega_k} \omega_k^3
  \Omega_{\mu\nu}\qty(\frac{\omega_k \Delta r_{ij}}{c})
  :
  \\
  &\qty(1 + n_B(\omega_k)) \int_0^\infty \dd{s} e^{\im s(\omega - \omega_k)}
  + n_B(\omega_k) \int_0^\infty \dd{s} e^{\im s(\omega + \omega_k)}.
  \end{split}
\end{align}
We now use that
\begin{equation}
  n_B(-\omega)
  = -\qty(1 + n_B(\omega))
\end{equation}
and 
\begin{equation}
  \int_0^\infty \dd{s} e^{-\im\omega s}
  = \pi\delta(\omega) - \im\pv\frac{1}{\omega},
\end{equation}
where $\pv$ denotes the Cauchy principal value, to find
\begin{align}
  \Gamma_{i\mu,j\nu}(\omega)
  &= \frac{1}{2}\gamma_{i\mu,j\nu}(\omega) + \im S_{i\mu,j\nu}(\omega),
  \\
  \shortintertext{where}
  \gamma_{i\mu,j\nu}(\omega)
  &= \delta_{\mu\nu} \frac{\mu_0 \omega^3}{3\pi \hbar c^3}
  \Omega_{\mu\nu}\qty(\frac{\abs{\omega} \Delta r_{ij}}{c})
  (1 + n_B(\omega))
  \\
  S_{i\mu,j\nu}(\omega)
  &= \delta_{\mu\nu} \frac{\mu_0}{6\pi^2 \hbar c^3}
  \pv\int_0^\infty \dd{\omega_k} \omega_k^3
  \Omega_{\mu\nu}\qty(\frac{\omega_k \Delta r_{ij}}{c})
  \qty(\frac{1 + n_B(\omega_k)}{\omega - \omega_k}
  + \frac{n_B(\omega_k)}{\omega + \omega_k}).
\end{align}
\todo[inline]{%
  Should the $n_B$ and $1 + n_B$ above be flipped? Otherwise I don't think the
  principal value integral is finite as $\omega_k \to \infty$. However, the
  divergent form is also in~\cite[p.~145]{opensys}.
}

\subsection{Old}

We consider the Hamiltonian (in natural units)
\begin{align}
  \ham
  &= \qty(-J\sum_i \pauli_{zi}\pauli_{z(i+1)} - h\sum_i \pauli_{xi})
  + \sum_{ik} \omega_{ik} \opr{a}_{ik}^\dag \opr{a}_{ik}
  + \sum_{ik} C_k \qty(\opr{a}_{ik}^\dag + \opr{a}_{ik})\pauli_{zi} \\
  &\equiv \ham_S \tp \idopr + \idopr \tp \ham_B + \ham_I.
\end{align}

The Schrodinger picture operators of the interaction Hamiltonian
\begin{equation}
  \ham_I
  = \sum_i \opr{A}_i \tp \opr{B}_i
\end{equation}
are
\begin{align}
  \opr{A}_i &= \pauli_{zi} \\
  \opr{B}_i &= \sum_k C_k\qty(\opr{a}_{ik}^\dag + \opr{a}_{ik})
  \equiv \sum_k \opr{B}_{ik}.
\end{align}
In the interaction picture,
\begin{align}
  \opr{A}_i(t)
  &= \pauli_{zi} \\
  \shortintertext{and}
  \opr{B}_{ik}(t)
  &= C_k \qty(e^{\im \omega_{ik} t} \opr{a}_{ik}^\dag
  + e^{-\im \omega_{ik} t} \opr{a}_{ik}).
\end{align}
\begin{proof}
  Consider an observable $\opr{A}$ which satisfies
  \begin{align}
    \comm{\ham}{\opr{A}} &= \omega\opr{A}
  \end{align}
  for a Hamiltonian $\ham$. Such an \term{eigenoperator} of $\ham$ has
  \begin{align}
    \ham^n\opr{A}
    &= \ham^{n-1}\opr{A}\ham + \ham^{n-1}\comm{\ham}{\opr{A}} \\
    &= \ham^{n-1}\opr{A}\qty(\ham + \omega\idopr) \\
    &= \opr{A}{\qty(\ham + \omega\idopr)}^n.
  \end{align}
  Then
  \begin{align}
    e^{\ham} \opr{A}
    &= \sum_{n \ge 0} \frac{\ham^n \opr{A}}{n!} \\
    &= \opr{A} \sum_{n \ge 0} \frac{{\qty(\ham + \omega\idopr)}^n}{n!} \\
    &= \opr{A} e^{\ham + \omega\idopr} \\
    &= \opr{A} e^{\omega} e^{\ham}, \by{\textsc{bch}}
  \end{align}
  so the interaction picture operator is
  \begin{align}
    \opr{A}(t)
  &= e^{\im\ham t} \opr{A} e^{-\im\ham t} \\
  &= e^{\im\omega t} \opr{A} e^{\im\ham t}e^{-\im\ham t} \\
  &= e^{\im\omega t} \opr{A}.
  \end{align}
  In our case, $\ham = \ham_B$ and we have the eigenoperators
  \begin{alignat}{2}
    \comm{\ham_B}{\opr{a}_{jl}}
  &=
  \sum_{ik} \comm{\opr{a}_{ik}^\dag\opr{a}_{ik}}{\opr{a}_{jl}}
  && \by{definition of $\ham_B$}
  \\
  &= -\sum_{ik} \delta_{ij}\delta_{kl} \opr{a}_{jl}
  && \by{commutation relations}
  \\
  &= -\opr{a}_{jl},
  \end{alignat}
  which follows from the commutation relations
  \begin{align}
    \comm{\opr{a}^\dag\opr{a}}{\opr{a}}
  &= \opr{a}^\dag\opr{a}\opr{a} - \opr{a}\opr{a}^\dag \opr{a} \\
  &=
  \opr{a}^\dag\opr{a}\opr{a} - \opr{a}^\dag\opr{a} \opr{a} -
  \opr{a}\comm{\opr{a}}{\opr{a}^\dag} \\
  &= -\opr{a}
  \\
  \shortintertext{and}
  \comm{\opr{a}^\dag\opr{a}}{\opr{a}^\dag}
  &= \opr{a}^\dag\opr{a}\opr{a}^\dag - \opr{a}^\dag\opr{a}^\dag \opr{a} \\
  &= \opr{a}^\dag\opr{a}\opr{a}^\dag - \opr{a}^\dag\opr{a}\opr{a}^\dag
  - \opr{a}^\dag\comm{\opr{a}^\dag}{\opr{a}} \\
  &= \opr{a}^\dag.
  \qedhere
  \end{align}
\end{proof}

For the Born-Markov approximation to hold, we must verify that
\begin{enumerate}
  \item $\tr_B\comm{\ham_I(t)}{\dop_0} = 0$, or also that
    $0 = \ensavg[\dop_B^{(I)}(t)]{\opr{B}_{ik}(t)} =
    \ensavg[\dop_B^{(S)}(t)]{\opr{B}_{ik}}$.
  \item $\dop(t) \approx \dop_S(t) \tp \dop_B$ (see~\cite[p.~131]{opensys}).
  \item The reservoir correlation functions
    $\ensavg[\dop_B^{(I)}(t)]{\opr{B}_{ik}^\dag(t)\opr{B}_{jl}(t-s)}$ decay
    quickly over a time $\tau_B$ much less than the relaxation time $\tau_R$.
\end{enumerate}
Condition \textbf{1} is satisfied if the bath is in a thermal state, which we
will also assume at $t = 0$ for condition \textbf{2}. The validity of \textbf{2}
rests on \textbf{3}: we do not require that the bath is truly
stationary, but only that it is approximately so on the coarser timescale of
system evolution.

In the thermal state
\begin{equation}
  \dop_\text{th}
  = \frac{e^{-\beta\ham_B}}{\tr e^{-\beta\ham_B}},
\end{equation}
the reservoir correlation functions are
\begin{align}
  \ev{\opr{B}_{ik}^\dag(t)\opr{B}_{jl}(t-s)} 
  &= \ev{\opr{B}_{ik}^\dag(s)\opr{B}_{jl}(0)} \\
  &= \delta_{ij}\delta_{kl}C_k C_l \qty(
  e^{-\im \omega_{ik} s}
  \qty(n_B(\omega_{ik}) + 1)
  +
  e^{\im \omega_{ik} s}
  n_B(\omega_{ik})
  ).
\end{align}
\begin{proof}
  We have that~\cite[p.~144]{opensys}
  \begin{align}
    \ev{\opr{a}_{ik}\opr{a}_{jl}}
  &= 0 \\
  \ev{\opr{a}_{ik}^\dag\opr{a}_{jl}^\dag}
  &= 0 \\
  \ev{\opr{a}_{ik}^\dag\opr{a}_{jl}}
  &= \delta_{ij}\delta_{kl} \, n_B(\omega_{ik}) \\
  \ev{\opr{a}_{ik}\opr{a}_{jl}^\dag}
  &= \delta_{ij}\delta_{kl} \qty(n_B(\omega_{ik}) + 1),
  \end{align}
  where
  \begin{equation}
    n_B(\omega)
    = \frac{1}{e^{\beta\omega} - 1}
  \end{equation}
  is the \term{Planck distribution}. We may then compute that
  \begin{align}
    &\ev{\opr{B}_{ik}^\dag(t)\opr{B}_{jl}(t-s)} \\
    &= C_k C_l\ev{\qty(e^{-\im \omega_{ik} t} \opr{a}_{ik}
      + e^{\im \omega_{ik} t} \opr{a}_{ik}^\dag)
      \qty(e^{\im \omega_{jl} (t-s)} \opr{a}_{jl}^\dag + e^{-\im \omega_{jl}
    (t-s)} \opr{a}_{jl})} \\
    &= C_k C_l
    \left(
      e^{-\im \omega_{ik} t + \im \omega_{jl} (t-s)}
      \ev{\opr{a}_{ik} \opr{a}_{jl}^\dag}
      +
      e^{-\im \omega_{ik} t - \im \omega_{jl} (t-s)}
      \ev{\opr{a}_{ik} \opr{a}_{jl}}
    \right. \\
    &\mathrel{\phantom{=}}
    + \left. e^{\im \omega_{ik} t + \im \omega_{jl} (t-s)}
      \ev{\opr{a}_{ik}^\dag \opr{a}_{jl}^\dag}
      +
      e^{\im \omega_{ik} t - \im \omega_{jl} (t-s)}
      \ev{\opr{a}_{ik}^\dag \opr{a}_{jl}}
    \right). \\
    &= \delta_{ij}\delta_{kl}C_k C_l \qty(
    e^{-\im \omega_{ik} s}
    \qty(n_B(\omega_{ik}) + 1)
    +
    e^{\im \omega_{ik} s}
    n_B(\omega_{ik})
    ).
    \qedhere
  \end{align}
\end{proof}
Thus the \term{spectral correlation tensor} is
\begin{align}
  \Gamma_{ij}(\omega)
  &\equiv \sum_{kl} \int_0^\infty \dd{s} e^{\im\omega s}
  \ensavg[\dop_B]{\opr{B}_{ik}^\dag(t)\opr{B}_{jl}(t-s)}
  \\
  &= \delta_{ij}\sum_k\int_0^\infty \dd{s} C_k^2 \qty(
  e^{\im (\omega - \omega_{ik}) s}
  \qty(n_B(\omega_{ik}) + 1)
  +
  e^{\im (\omega + \omega_{ik}) s}
  n_B(\omega_{ik})
  ) \\
  &= \delta_{ij} \sum_k C_k^2
  \qty(\pi\delta(\omega_{ik} - \omega) - \im\pv\frac{1}{\omega_{ik} - \omega})
  \qty(n_B(\omega_{ik}) + 1) \\
  &+
  \qty(\pi\delta(-\omega_{ik} - \omega) + \im\pv\frac{1}{\omega_{ik} + \omega})
  n_B(\omega_{ik}) \\
  &\equiv \delta_{ij}\qty(\frac{\gamma(\omega)}{2} + \im S(\omega)),
\end{align}
where we have used that
\begin{equation}
  \int_0^\infty \dd{s} e^{-\im\omega s}
  = \pi\delta(\omega) - \im\pv\frac{1}{\omega},
\end{equation}
and $\pv$ denotes the Cauchy principal value.
We now take the continuum limit of a large 1\textsc{d} cavity (with $\omega =
ck$)\footnote{We are assuming that there is only one mode per frequency per site
in 1\textsc{d}, rather than that there may be many modes for a given frequency,
as usual in 3\textsc{d}.}
\begin{equation}
  \sum_k
  \mapsto \frac{L}{\pi}\int_0^\infty \dd{k}
  = \frac{L}{\pi c}\int_0^\infty \dd{\omega}
\end{equation}
so that we have\footnote{Note that the Planck distribution satisfies
$n_B(\omega) + 1 = -n_B(-\omega)$.}
\begin{align}
  \gamma(\omega)
  &= \frac{2L}{c} {C(\omega)}^2 \qty(n_B(\omega) + 1)
  \label{eq:gamma-corr} \\
  \shortintertext{and}
  S(\omega)
  &= \pv\int_0^\infty \dd{\omega_k} \frac{L}{c} {C(\omega_k)}^2
  \qty(\frac{n_B(\omega_{ik}) + 1}{\omega - \omega_{ik}}
  + \frac{n_B(\omega_{ik})}{\omega + \omega_{ik}}).
\end{align}
Since the only eigenvalue of $\ham_S\angle$ is $\omega = 0$, we must require
that
\begin{equation}
  \gamma(0)
  = \lim_{\omega \to 0} \frac{2L}{c} {C(\omega)}^2 \qty(n_B(\omega) + 1)
\end{equation}
is finite. There are two relevant cases.

If $C(\omega)$ initially grows faster than $\sqrt{\omega}$, then
$\gamma(0) = 0$ and the jump operators vanish, leaving just the Lamb shift.
We then evaluate
\begin{align}
  S(0)
  &= \pv\int_0^\infty \dd{\omega_k} \frac{L}{c} {C(\omega_k)}^2
  \qty(\frac{n_B(\omega_{ik}) + 1}{- \omega_{ik}}
  + \frac{n_B(\omega_{ik})}{\omega_{ik}}) \\
  &= -\frac{L}{c} \pv\int_0^\infty \dd{\omega_k} \frac{{C(\omega_k)}^2}{\omega_k}
  \by{let $\omega_{ik} \coloneqq\omega_{k}$}
\end{align}
In this case, the exact form of $C(\omega_k)$ does not matter much. So long as it also
goes to zero, $S(0)$ will take some constant negative value.

If instead, $C(\omega) \propto \sqrt{\omega}$, then $\gamma(0)$ is finite. A
common choice makes the \term{spectral density}
\begin{align}
  J(\omega)
  &\equiv \frac{2\alpha}{\pi} \int_0^\infty \dd{\omega_k} {C(\omega_k)}^2
  \delta(\omega - \omega_k) \\
  &= \frac{2\alpha}{\pi} {C(\omega)}^2 \\
  \shortintertext{be}
  J(\omega)
  &= \frac{2\alpha}{\pi}\frac{\omega}{1 + {(\omega/\Omega)}^2}.
  \label{eq:ohmicJ}
\end{align}
This is known as the \term{Ohmic spectral density} with cutoff frequency
$\Omega$, which gives rise to frequency-independent damping with a rate
proportional to $\alpha$~\cite[p.~175]{opensys}. Then
\begin{align}
  S(0)
  &= -\frac{2\alpha}{\pi}\pv\int_0^\infty \frac{\dd{\omega_k}}{1 +
  {(\omega_k / \Omega)}^2} \\
  &= -\frac{\Omega}{\alpha}.
\end{align}
Thus the Lamb shift Hamiltonian
\[
  \ham_{LS}
  = \sum_{ij} \delta_{ij} S(0) \pauli_{zi} \pauli_{zj}
  = -\frac{N\Omega}{\alpha}\idopr
\]
only shifts the energy of the chain. With an Ohmic bath, we find
\begin{align}
  \gamma(0)
  &= \lim_{\omega \to 0} \frac{4\alpha L}{\pi c} \frac{\omega}{1 + {(\omega/\Omega)}^2}
  \qty(n_B(\omega) + 1) \\
  &= \frac{2L}{c} \frac{1}{\beta},
\end{align}
so the dissipator is
\begin{equation}
  \sopr{D} \dop_S
  = \frac{4L}{\pi c}\frac{\alpha}{\beta}
  \sum_i \qty(\pauli_{zi}\dop_S\pauli_{zi} - \dop_S).
\end{equation}
Thus neglecting the Lamb shift and $4L / \pi c$, we have that the system density
matrix in the interaction picture obeys
\begin{align}
  \dot{\dop}_S(t)
  &= \frac{\alpha}{\beta}
  \sum_i \qty(\pauli_{zi}\dop_S(t)\pauli_{zi} - \dop_S(t)) \\
  &= \frac{\alpha}{\beta}
  \sum_i \comm{\pauli_{zi}}{\dop_S(t)}\pauli_{zi}.
\end{align}
The reduced density matrix entries for each site are then determined by the equations
\begin{align}
  \dot{\dop}_{00}
  &= 0
  \\
  \dot{\dop}_{11}
  &= 0
  \\
  \dot{\dop}_{01}
  &= -\frac{2\alpha}{\beta} \dop_{01}
  \\
  \dot{\dop}_{01}
  &= -\frac{2\alpha}{\beta} \dop_{10}.
\end{align}
\begin{proof}
  If $\opr{A}\ket{a} = a\ket{a}$, then
  \[
    \opr{A} \angle \op{a}{b}
    = (a - b)\op{a}{b},
  \]
  where the superoperator $\opr{A}\angle$ is defined by
  $\opr{A}\mathbin{\angle}\opr{B} \equiv \comm{\opr{A}}{\opr{B}}$.
\end{proof}

\section{Solving eigenoperator problems in coordinates}

Given a complete basis $\opr{A}_i$ for $\liou(\hilb_A)$ and the Hamiltonian
\begin{equation}
  \ham
  = \sum_i h_i \opr{A}_i,
\end{equation}
we want to find eigenoperators
\begin{equation}
  \opr{A}
  = \sum_j a_j \opr{A}_j,
\end{equation}
of $\ham\angle$. If 
\begin{align}
  \comm{\opr{A}_i}{\opr{A}_j}
  &= \sum_k s_{ijk} \opr{A}_k \\
  s_{ijk}
  &= \ip{\comm{\opr{A}_i}{\opr{A}_j}}{\opr{A}_k},
\end{align}
then the eigenvalue equation is
\begin{align}
  \comm{\ham}{\opr{A}}
  &= \omega\opr{A} \\
  \sum_{ijk} h_i a_j s_{ijk} \opr{A}_k
  &= \sum_{j} \omega a_j \opr{A}_j \\
  \sum_{jk} S_{jk} a_j  \opr{A}_k
  &= \sum_{j} \omega a_j \opr{A}_j,
\end{align}
where the matrix $\mq{S}$ has coefficients
\begin{equation}
  S_{jk}
  = \sum_i h_i s_{ijk}
  = \mel{\opr{A}_j}{\ham\angle}{\opr{A}_k}.
\end{equation}
Thus eigenoperators may be found by solving the ordinary eigenvalue problem
\begin{equation}
  \mq{S}^\trans \vq{a} = \omega \vq{a}.
\end{equation}

\section{Eigenoperators for the transverse Ising Hamiltonian}

Pfeuty algebra time.


% \notebook{closed-simulations}

% \notebook{ising-diss}

\end{document}

