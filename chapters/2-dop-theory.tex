\documentclass[../thesis.tex]{subfiles}
\begin{document}

\chapter{Density operator theory}

\section{Mathematical details\label{sec:math}}

\begin{defn}[Tensor product]\label{def:tensors}
  Consider vector spaces $V(k)$, $W(k)$, and $Z$. For any bilinear map $h:V
  \times W \to Z$, the \term{tensor product} $V \tp W$ and associated
  bilinear map $\phi:V \times W \to V \tp W$ map have the property that there
  is a unique linear map $g:V \tp W \to Z$ such that $h = g \circ \phi$. For
  tensor products of Hilbert spaces, the inner product is defined on each
  element of a product and then the space is completed. There is then a natural
  correspondence between the element $v \tp f$ of the tensor product $V \tp V^*$
  and the linear map $T:V \to V$ defined by $Tx = f(x)v$.

  This induces an extension of Dirac notation where all pairs $f \tp x$ of
  dual and usual vectors from the same space are evaluated as $\ip{f}{x} = f(x)$
  and extended linearly. For example, given a linear operator $\opr{U}:V \tp
  W \to V \tp W$ and a basis $\ket{\phi_i}$ for $W$, the partial trace over
  $W$ may be expressed as $\tr_W\opr{U} = \ev{\opr{U}}{\phi_i}$. This forms the
  justification of the step from \cref{eq:krauspre} to \cref{eq:krauspost} and
  of the manipulations in \cref{thm:kraus}.
\end{defn}

\begin{thm}[Klein inequality]\label{thm:klein}
  For density operators $\dop$ and $\dop'$, $S(\dop \| \dop') \geq 0$, with
  equality if and only if $\dop = \dop'$.
  \begin{proof}
    The case for equality is trivial, so we will consider $\dop \neq \dop'$. Let
    $\sopr{F}\dop = \dop\ln\dop$, so that we may express the relative entropy as
    \[
      S(\dop \| \dop')
      = \tr(\sopr{F}\dop - \sopr{F}\dop' - \opr{δ}\sopr{F}'\dop'),
    \]
    where $\opr{δ} = \dop - \dop'$. We then have for $0 < t < 1$ that
    \[
      \dop' + t\opr{δ}
      = t\dop + (1 - t)\dop'.
    \]
    Now let $f(t) = \tr\qty(\sopr{F}(\dop' + t\opr{δ}))$. Since the trace is
    monotonic and convex, $f$ is convex and $f(t) \leq f(0) + t\qty(f(1) -
    f(0))$. Rearranging and taking the limit as $t \to 0^+$ gives
    \[
      f'(0)
      \leq f(1) - f(0),
    \]
    which evaluates to
    \[
      \tr\qty(\opr{δ}\sopr{F}'\dop')
      \leq \tr\sopr{F}\dop - \tr\sopr{F}\dop'.
      \qedhere
    \]
  \end{proof}
\end{thm}

\begin{thm}\label{thm:relSunitary}
  For a unitary operator $\opr{U}$ and density operators $\dop$ and $\dop'$,
  \[
    S(\opr{U}\dop\opr{U}^\dag \| \opr{U}\dop'\opr{U}^\dag)
    = S(\dop \| \dop').
  \]
  \begin{proof}
    Since we may cycle the traces, it suffices to show that
    \[
      \ln\qty(\opr{U}\dop\opr{U}^\dag)
      = \ln\dop.
    \]
    This follows from Jacobi's formula for invertible matrices when
    applied to the logarithm that takes us from a Lie group to its corresponding
    Lie algebra, giving $\tr \circ \det = \tr \circ \log$.
  \end{proof}
\end{thm}

\begin{thm}\label{thm:relSsubsys}
  For density operators $\dop$ and $\dop'$,
  \[
    S(\tr_B\dop \| \tr_B\dop')
    \leq S(\dop \| \dop'),
  \]
  with equality if and only if $\dop$ or $\dop'$ is uncorrelated.
\end{thm}

\begin{thm}[Choi-Kraus representation~\cite{intro}]\label{thm:kraus}
  A superoperator $\sopr{S}$ on a density operator $\dop$ is completely positive
  and trace-preserving if and only if it may be represented as
  \[
    \sopr\dop
    = \sum_{k=1}^K \opr{M}_k \dop \opr{M}_k^\dag,
    \qq{where}
    \sum_{k=1}^K \opr{M}_k \opr{M}_k^\dag
    = \idopr.
  \]
\end{thm}

\section{The Born-Markov approximation for the Ising chain in a
bath\label{sec:bm-ising}}

The bath Hamiltonian is
\begin{equation}
  \ham_B
  = \sum_{\vq{k},\, \lambda} \hbar\omega_{\vq{k}}
  \opr{a}^\dag_{\vq{k},\lambda}\opr{a}_{\vq{k},\lambda}.
  \label{eq:bath-hamiltonian}
\end{equation}
The vacuum energy $\sum_{\vq{k},\, \lambda} \hbar\omega_{\vq{k}} / 2$ is
dropped, since it diverges in the continuum limit.

The interaction Hamiltonian for spin-$s$ objects in a magnetic field
is\footnote{The time dependence of the field is absorbed into the operators
$\opr{a}_{\vq{k},\lambda}$, and the prefactor is chosen so that these operators
are dimensionless, but $\vecopr{B}$ is not.}
\begin{align}
  \ham_I
  &= -\int\dd{\vq{r}} \vecopr{\mu} \vdot \vecopr{B}
  \\
  \begin{split}
  &= -\int\dd{\vq{r}} \sum_i m_s g_s \mu_B \delta(\vq{r}_i) \vecopr{\pauli}_i \\
  &\vdot
  \im\sum_{\vq{k},\, \lambda} \sqrt{\frac{\hbar c^2 \mu_0}{2V\omega_{\vq{k}}}}
  \qty(%
  \qty(\vq{k} \cross \vq{e}_{\vq{k},\lambda})
  e^{\im\vq{k}\vdot\vq{r}} \opr{a}_{\vq{k},\lambda}
  - \qty(\vq{k} \cross \vq{e}^*_{\vq{k},\lambda})
  e^{-\im\vq{k}\vdot\vq{r}} \opr{a}^\dag_{\vq{k},\lambda}
  )
  \end{split}
  \\
  &= -\sum_{i,\,\mu} m_s g_s \mu_B \pauli_i^\mu \opr{B}_i^\mu,
\end{align}
where we have defined the Hermitian operator
\begin{equation}
  \opr{B}_i^\mu
  = \im\sum_{\vq{k},\, \lambda} \sqrt{\frac{\hbar c^2 \mu_0}{2V\omega_{\vq{k}}}}
  \qty(%
  {\qty(\vq{k} \cross \vq{e}_{\vq{k},\lambda})}_\mu
  e^{\im\vq{k}\vdot\vq{r}_i} \opr{a}_{\vq{k},\lambda}
  - {\qty(\vq{k} \cross \vq{e}^*_{\vq{k},\lambda})}_\mu
  e^{-\im\vq{k}\vdot\vq{r}_i} \opr{a}^\dag_{\vq{k},\lambda}
  ).
  \label{eq:Bimu-schrodinger}
\end{equation}
In the interaction picture:
\begin{align}
  \opr{B}_i^\mu(t)
  &= e^{\im\ham_B t/\hbar} \opr{B}_i^\mu e^{-\im\ham_B t/\hbar} \\
  &= \im\sum_{\vq{k},\, \lambda} \sqrt{\frac{\hbar c^2 \mu_0}{2V\omega_{\vq{k}}}}
  \qty(%
  {\qty(\vq{k} \cross \vq{e}_{\vq{k},\lambda})}_\mu
  e^{\im(\vq{k}\vdot\vq{r}_i - \omega_{\vq{k}} t)} \opr{a}_{\vq{k},\lambda}
  - {\qty(\vq{k} \cross \vq{e}^*_{\vq{k},\lambda})}_\mu
  e^{-\im(\vq{k}\vdot\vq{r}_i - \omega_{\vq{k}} t)} \opr{a}^\dag_{\vq{k},\lambda}
  ).
  \label{eq:Bimu-interaction}
\end{align}

The spectral correlation tensor is then
\begin{align}
  \Gamma_{i\mu,j\nu}(\omega)
  &= \frac{1}{\hbar^2} \int_0^\infty \dd{s} e^{\im\omega s}
  \ev{\opr{B}_i^\mu{(t)}^\dag \opr{B}_j^\nu(t - s)}
  \\
  \begin{split}
  &= -\frac{1}{\hbar^2} \frac{\hbar c^2 \mu_0}{2V} \int_0^\infty \dd{s}
  \sum_{\vq{k},\, \vq{k}',\, \lambda,\, \lambda'}
  \sqrt{\frac{1}{\omega_{\vq{k}} \omega_{\vq{k'}}}}
  :
  \\
  &{\qty(\vq{k} \cross \vq{e}_{\vq{k},\lambda})}_\mu
    {\qty(\vq{k}' \cross \vq{e}_{\vq{k}',\lambda'})}_\nu
    e^{\im(\vq{k}\vdot\vq{r}_i - \omega_{\vq{k}} t + \vq{k}'\vdot\vq{r}_j -
    \omega_{\vq{k}'} (t-s) + \omega s)}
  \ev{%
    \opr{a}_{\vq{k},\lambda}\opr{a}_{\vq{k}',\lambda'}
  } \\
  &-
    {\qty(\vq{k} \cross \vq{e}_{\vq{k},\lambda})}_\mu
    {\qty(\vq{k}' \cross \vq{e}^*_{\vq{k}',\lambda'})}_\nu
    e^{\im(\vq{k}\vdot\vq{r}_i - \omega_{\vq{k}} t - \vq{k}'\vdot\vq{r}_j +
    \omega_{\vq{k}'} (t-s) + \omega s)}
  \ev{%
    \opr{a}_{\vq{k},\lambda}\opr{a}^\dag_{\vq{k}',\lambda'}
  } \\
  &-
    {\qty(\vq{k} \cross \vq{e}^*_{\vq{k},\lambda})}_\mu
    {\qty(\vq{k}' \cross \vq{e}_{\vq{k}',\lambda'})}_\nu
    e^{-\im(\vq{k}\vdot\vq{r}_i - \omega_{\vq{k}} t - \vq{k}'\vdot\vq{r}_j +
    \omega_{\vq{k}'} (t-s) - \omega s)}
  \ev{%
    \opr{a}^\dag_{\vq{k},\lambda}\opr{a}_{\vq{k}',\lambda'}
  } \\
  &+
    {\qty(\vq{k} \cross \vq{e}^*_{\vq{k},\lambda})}_\mu
    {\qty(\vq{k} \cross \vq{e}^*_{\vq{k}',\lambda'})}_\nu
    e^{-\im(\vq{k}\vdot\vq{r}_i - \omega_{\vq{k}} t + \vq{k}'\vdot\vq{r}_j -
    \omega_{\vq{k}'} (t-s) - \omega s)}
  \ev{%
    \opr{a}^\dag_{\vq{k},\lambda}\opr{a}^\dag_{\vq{k}',\lambda'}
  }.
\end{split}
\end{align}

In the thermal state
\begin{equation}
  \dop_B
  = \frac{e^{-\beta\ham_B}}{\tr e^{-\beta\ham_B}}
  = \prod_{\vq{k},\,\lambda} \qty(1 - e^{-\beta\hbar\omega_{\vq{k}}})
  e^{-\beta\hbar\omega_{\vq{k}} \opr{a}^\dag_{\vq{k},\lambda} \opr{a}_{\vq{k},\lambda}}
\end{equation}
Since $\comm{\opr{a}_{\vq{k},\lambda}}{\opr{a}^\dag_{\vq{k}',\lambda'}} =
\delta_{\vq{k},\vq{k}'}\delta_{\lambda\lambda'} \idopr$,
\begin{align}
  \ev{\opr{a}^\dag_{\vq{k},\lambda} \opr{a}_{\vq{k}',\lambda'}}
  &= {\tr(e^{-\beta\ham_B})}^{-1} \tr(e^{-\beta\ham_B}
  \opr{a}^\dag_{\vq{k},\lambda} \opr{a}_{\vq{k}',\lambda'})
  \\
  &= {\tr(e^{-\beta\ham_B})}^{-1} \tr(e^{-\beta\ham_B}
  \opr{a}_{\vq{k}',\lambda'} \opr{a}^\dag_{\vq{k},\lambda})
  - \delta_{\vq{k},\vq{k}'}\delta_{\lambda\lambda'}
  \\
  &= {\tr(e^{-\beta\ham_B})}^{-1} \tr(
  e^{\beta\hbar \omega_{\vq{k}}}
  \opr{a}_{\vq{k}',\lambda'} e^{-\beta\ham_B} \opr{a}^\dag_{\vq{k},\lambda})
  - \delta_{\vq{k},\vq{k}'}\delta_{\lambda\lambda'}
  \\
  &= e^{\beta\hbar \omega_{\vq{k}}}
  \ev{\opr{a}^\dag_{\vq{k},\lambda} \opr{a}_{\vq{k}',\lambda'}}
  - \delta_{\vq{k},\vq{k}'}\delta_{\lambda\lambda'}
  \\
  &= \delta_{\vq{k},\vq{k}'}\delta_{\lambda\lambda'} n_B(\omega_{\vq{k}}).
\end{align}
Similarly,
\begin{align}
  \ev{\opr{a}_{\vq{k},\lambda} \opr{a}_{\vq{k}',\lambda'}}
  &= \ev{\opr{a}^\dag_{\vq{k},\lambda} \opr{a}^\dag_{\vq{k}',\lambda'}}
  = 0
  \\
  \ev{\opr{a}_{\vq{k},\lambda} \opr{a}^\dag_{\vq{k}',\lambda'}}
  &= \delta_{\vq{k}\vq{k}'} \delta_{\lambda\lambda'}
  \qty(1 + n_B(\omega_{\vq{k}})).
\end{align}
Then for a thermal bath, the spectral correlation tensor becomes
\begin{align}
  \begin{split}
  \Gamma_{i\mu,j\nu}(\omega)
  &= \frac{c^2 \mu_0}{2\hbar V} \int_0^\infty \dd{s}
  \sum_{\vq{k},\,\lambda}
  \frac{1}{\omega_{\vq{k}}}
  :
  \\
  &{\qty(\vq{k} \cross \vq{e}_{\vq{k},\lambda})}_\mu
    {\qty(\vq{k} \cross \vq{e}^*_{\vq{k},\lambda})}_\nu
    e^{\im(\vq{k}\vdot(\vq{r}_i - \vq{r}_j) + s(\omega - \omega_{\vq{k}}))}
    \qty(1 + n_B(\omega_{\vq{k}}))
  \\
  &+
    {\qty(\vq{k} \cross \vq{e}^*_{\vq{k},\lambda})}_\mu
    {\qty(\vq{k} \cross \vq{e}_{\vq{k},\lambda})}_\nu
    e^{-\im(\vq{k}\vdot(\vq{r}_i - \vq{r}_j) - s(\omega + \omega_{\vq{k}}))}
    n_B(\omega_{\vq{k}}).
\end{split}\label{eq:Gamma-precontinuum}
\end{align}
\todo[inline]{%
  $\vq{k}$ in exponent of original integral. Is there a reason why the
  external field $\vq{k}$ would be the same as a reciprocal lattice vector? Like
  optical phonon branch. Magnons? Just no absorption for off frequencies? Or do
  typical scales enforce limits?
}

To evaluate \cref{eq:Gamma-precontinuum}, we now consider a chain of $N$ spins
along the $z$-axis, so that $\vq{r}_i = r_i \vu{z}$.\footnote{We could consider
any axis given the spherical symmetry, but the $z$-axis is the simplest to
evaluate.} Then $\vq{k} \vdot (\vq{r}_i - \vq{r}_j) = k_z \Delta r_{ij}$.

In the continuum limit,
\begin{equation}
  \frac{1}{V}\sum_{\vq{k}}
  \mapsto
  \int \frac{\dd{\vq{k}}}{{(2\pi)}^3}
  = \frac{1}{{(2\pi c)}^3} \int_0^\infty \dd{\omega_k} \omega_k^2
  \int\dd{\Omega},
\end{equation}
where the integral over solid angle is
\begin{equation}
  \int\dd{\Omega}
  = \int\dd{\phi}\int\dd{\theta}\sin\theta.
  \label{eq:solid-angle}
\end{equation}
To apply this limit to \cref{eq:Gamma-precontinuum}, we first note that
\begin{align}
  \sum_\lambda {\qty(\vq{k} \cross \vq{e}_{\vq{k},\lambda})}_\mu
  {\qty(\vq{k} \cross \vq{e}^*_{\vq{k},\lambda})}_\nu
  &= \sum_{abcd} \varepsilon_{\mu ab}\varepsilon_{\nu cd} k^a k^c 
  \sum_\lambda e_{\vq{k},\lambda}^b {\qty(e_{\vq{k},\lambda}^d)}^*
  \\
  &= \sum_{abcd} \varepsilon_{\mu ab}\varepsilon_{\nu cd} k^a k^c 
  \qty(\delta_{bd} - \frac{k^b k^d}{k^2})
  \\
  &= \sum_{abc} \varepsilon_{\mu ab}\varepsilon_{\nu cb} k^a k^c 
  \\
  &= \sum_{ac} \qty(\delta_{\mu\nu}\delta_{ac} - \delta_{\mu c}\delta_{a\nu}) k^a k^c 
  \\
  &= k^2 \delta_{\mu\nu} - k^\mu k^\nu.
\end{align}
Thus
\begin{equation}
  \int\dd{\Omega} e^{\pm\im k_z \Delta r_{ij}}
  \sum_\lambda {\qty(\vq{k} \cross \vq{e}_{\vq{k},\lambda})}_\mu
  {\qty(\vq{k} \cross \vq{e}^*_{\vq{k},\lambda})}_\nu
  = \frac{8\pi\omega_k^2}{3c^2} \delta_{\mu\nu}
  \Omega_{\mu\nu}\qty(\frac{\omega_k\Delta r_{ij}}{c}),
  \label{eq:angular-integral}
\end{equation}
where
\begin{equation}
  \Omega_{\mu\nu}(u)
  = \qty(\delta_{\nu z} - \frac{\delta_{\nu x} + \delta_{\nu y}}{2})
  \frac{\sinc u - \cos u}{u^2}
  + \frac{\delta_{\nu x} + \delta_{\nu y}}{2} \sinc u.
\end{equation}
Now \cref{eq:angular-integral} gives that the continuum limit of the spectral
correlation tensor for the spin chain is
\begin{align}
  \begin{split}
  \Gamma_{i\mu,j\nu}(\omega)
  &= \delta_{\mu\nu} \frac{\mu_0}{6\pi^2 \hbar c^3}
  \int_0^\infty \dd{\omega_k} \omega_k^3
  \Omega_{\mu\nu}\qty(\frac{\omega_k \Delta r_{ij}}{c})
  :
  \\
  &\qty(1 + n_B(\omega_k)) \int_0^\infty \dd{s} e^{\im s(\omega - \omega_k)}
  + n_B(\omega_k) \int_0^\infty \dd{s} e^{\im s(\omega + \omega_k)}.
  \end{split}
\end{align}
We now use that
\begin{equation}
  n_B(-\omega)
  = -\qty(1 + n_B(\omega))
\end{equation}
and 
\begin{equation}
  \int_0^\infty \dd{s} e^{-\im\omega s}
  = \pi\delta(\omega) - \im\pv\frac{1}{\omega},
\end{equation}
where $\pv$ denotes the Cauchy principal value, to find
\begin{align}
  \Gamma_{i\mu,j\nu}(\omega)
  &= \frac{1}{2}\gamma_{i\mu,j\nu}(\omega) + \im S_{i\mu,j\nu}(\omega),
  \\
  \shortintertext{where}
  \gamma_{i\mu,j\nu}(\omega)
  &= \delta_{\mu\nu} \frac{\mu_0 \omega^3}{3\pi \hbar c^3}
  \Omega_{\mu\nu}\qty(\frac{\abs{\omega} \Delta r_{ij}}{c})
  (1 + n_B(\omega))
  \\
  S_{i\mu,j\nu}(\omega)
  &= \delta_{\mu\nu} \frac{\mu_0}{6\pi^2 \hbar c^3}
  \pv\int_0^\infty \dd{\omega_k} \omega_k^3
  \Omega_{\mu\nu}\qty(\frac{\omega_k \Delta r_{ij}}{c})
  \qty(\frac{1 + n_B(\omega_k)}{\omega - \omega_k}
  + \frac{n_B(\omega_k)}{\omega + \omega_k}).
\end{align}
\todo[inline]{%
  Should the $n_B$ and $1 + n_B$ above be flipped? Otherwise I don't think the
  principal value integral is finite as $\omega_k \to \infty$. However, the
  divergent form is also in~\cite[p.~145]{opensys}.
}


\section{Superoperators in coordinates}

Consider an orthonormal basis $\ket{i}$ for $\hilb$ and thus $\liou$. This
induces a basis for superoperators by pre and post-multiplication.

The superoperator matrix element for left-multiplication is
\begin{align}
  \mel{ab}{\opr{A}_L}{cd}
  &= \sum_{ef} \mel{ab}{A_{ef}}{ef}\ket{cd} \\
  &= \sum_{ef} \delta_{fc} \mel{ab}{A_{ef}}{ed} \\
  &= \sum_{ef} \delta_{fc} \delta_{ae} \delta_{bd} A_{ef} \\
  &= \delta_{db} A_{ac},
\end{align}
while that for right-multiplication is
\begin{align}
  \mel{ab}{\opr{A}_R}{cd}
  &= \sum_{ef} \mel{ab}{A_{ef}}{cd}\ket{ef} \\
  &= \sum_{ef} \delta_{de} \mel{ab}{A_{ef}}{cf} \\
  &= \sum_{ef} \delta_{de} \delta_{ac} \delta_{bf} A_{ef} \\
  &= \delta_{ac} A_{db}.
\end{align}
Thus for both multiplications, we have that
\begin{align}
  \mel{ab}{\opr{A}_L\opr{B}_R}{cd}
  &= \sum_{ef} \mel{ab}{\opr{A}_L}{ef}\mel{ef}{\opr{B}_R}{cd} \\
  &= \sum_{ef} \delta_{fb} \delta_{ec} A_{ae} B_{df} \\
  &= A_{ac} B_{db}.
\end{align}
In particular, if $\opr{B} = \opr{A}^\dag$, this is $A_{ac} A_{bd}^*$, so that
the trace of the superoperator is $\norm{\opr{A}}^2$.

Given $\opr{A}$, we have the commutator superoperator $[\opr{A}] = \opr{A}_L -
\opr{A}_R$ and the anticommutator superoperator $\{\opr{A}\} = \opr{A}_L +
\opr{A}_R$. We then see that
\begin{equation}
  \frac{1}{2}\mel{ab}{\{\opr{A}^\dag \opr{A}\}}{cd}
  = \frac{1}{2}\sum_k\qty(\delta_{db} A_{ak}A_{ck}^* + \delta_{ac} A_{dk}A_{bk}^*),
\end{equation}
so the elements for the dissipator $\sopr{D} = \sum_i \gamma_i\qty(\opr{A}_L
\opr{A}_R^\dag - \{\opr{A}^\dag \opr{A}\}/2)$ are
\begin{equation}
  \mel{ab}{\sopr{D}}{cd}
  = \sum_i \gamma_i \qty(A_{ac}A_{bd}^* - \frac{1}{2}\sum_k\qty(\delta_{db} A_{ak}A_{ck}^* + \delta_{ac} A_{dk}A_{bk}^*))
\end{equation}


% Notebooks
% \notebook{closed-simulations}
% \notebook{ising-diss}

\end{document}

