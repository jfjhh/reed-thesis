\documentclass[../thesis.tex]{subfiles}
\begin{document}

\chapter{Application to interacting spins}\label{ch:spins}

With the preceeding theory of open quantum systems in place, we are now in a
position to consider the central problem of this thesis: How do interacting
spins relax to thermal equilibrium? We will approach this question in stages,
starting with a single spin.

\section{The closed two-dimensional system}

Our first task is to characterize the closed dynamics of the two-dimensional
quantum system. An arbitrary two-dimensional density operator may be expressed
in terms of the identity operator $\idopr \equiv \pauli^0$ and \term{Pauli
operators} $\pauli^a$ that satisfy the commutation relations
\begin{equation}
  \comm{\pauli^a}{\pauli^b}
  = 2\im \varepsilon_{abc} \pauli^c,
  \label{eq:pauli-comm}
\end{equation}
where $\varepsilon_{abc}$ is the Levi-Civita symbol. These will be abbreviated
as a single vector $\paulivec$. Then a \term{Bloch vector} $\vq{v}$ with
$\abs{\vq{v}} \le 1$ determines the density operator
\begin{equation}
  \dop
  = \frac{1}{2}\qty(\idopr + \vq{v} \vdot \paulivec).
  \label{eq:bloch-dop}
\end{equation}
The expectation values of the Pauli operators are $\ev{\paulivec} = \vq{v}$, so
it suffices to determine the time evolution of $\vq{v}$. A similar vector
$\vq{\omega}$ determines the general Hamiltonian\footnote{%
  If the Hamiltonian $\ham = E\idopr + \ham_0$ includes an energy shift $E$, the
  corresponding unitary operator is $\opr{U}(t) = e^{-\im t E}\opr{U}_0(t)$.
  Then $\opr{U}(t) \dop \opr{U}^\dag(t) = \opr{U}_0(t) \dop \opr{U}_0^\dag(t)$,
  so there is no difference.
}
\begin{equation}
  \ham
  = \frac{1}{2}\vq{\omega} \vdot \paulivec.
\end{equation}
The Liouville equation (\cref{eq:liouville}) is then
\begin{align}
  \dot{\dop}
  &= -\frac{\im}{4}\comm{\vq{\omega} \vdot \paulivec}{\vq{v} \vdot \paulivec} \\
  \frac{1}{2} \dot{\vq{v}} \vdot \paulivec
  &= \frac{1}{2}(\vq{\omega} \cross \vq{v}) \vdot \paulivec,
\end{align}
so the Bloch vector follows the differential equation
\begin{equation}
  \dot{\vq{v}}
  = \vq{\omega} \cross \vq{v}.
\end{equation}
The solution of this equation is
\begin{equation}
  \vq{v}(t)
  = e^{-t \vq{\omega} \vdot \vecopr{L}} \vq{v}(0),
\end{equation}
where the components of $\vecopr{L}$ are generators of three-dimensional
rotations which satisfy the commutation relations\footnote{%
  That is, $\opr{L}^x \vq{r} = \vq{x} \cross \vq{r}$, etc. Contrary to the norm
  in quantum mechanics, this representation of $\mathfrak{so}(3)$ does not use
  the imaginary unit $\im$.
}
\begin{equation}
  \comm{\opr{L}^a}{\opr{L}^b}
  = \varepsilon_{abc} \opr{L}^c.
\end{equation}
Thus the most general time evolution for the two-dimensional system is
precession of the Bloch vector around $\vu{\omega}$ with angular frequency
$\omega$.


\section{Relaxation of the two-dimensional system}

Now what happens when a two-dimensional system is coupled to a thermal bath? The
most relevant application is the electronic states of an atom coupled to the
electromagnetic field. We will approximate the atom as just two energy levels: a
ground and first excited state. For simplicity we will consider the system
Hamiltonian
\begin{equation}
  \ham_S
  = \frac{\omega}{2}\pauli^z,
\end{equation}
where $\omega$ is the frequency of light emitted from the transition. The
weak-coupling limit for the electric dipole interaction with the bath is similar
to that for the magnetic dipole interaction considered later in
\cref{sec:weak-ising}, so will simply state the results here and focus on
investigating the dynamics.\footnote{%
  A full derivation is given in~\cite[pp.~141--149]{opensys}.
}
To focus on the relaxation of the system, we ignore the Lamb shift and consider
only the effect of the dissipator. The jump operators are
\begin{equation}
  \pauli^\pm
  = \frac{\pauli^x \pm \im\pauli^y}{2}
\end{equation}
and the simplified Lindblad equation $\dot{\dop} = \sopr{D}\dop$ is
\begin{equation}
  \dot{\dop}
  = \gamma_0(N + 1) \qty(\pauli^- \dop \pauli^+ -
  \frac{1}{2}\acomm{\pauli^+\pauli^-}{\dop})
  + \gamma_0 N \qty(\pauli^+ \dop \pauli^- -
  \frac{1}{2}\acomm{\pauli^-\pauli^+}{\dop}),
  \label{eq:atom-lindblad}
\end{equation}
where\footnote{%
  The rate $\gamma_0$ is also known as the \term{Einstein $A$
  coefficient}~\cite[p.~417]{griffithsIntroductionQuantumMechanics2018}
}
\begin{equation}
  \gamma_0
  = \frac{4}{3}\abs{\vq{d}}^2 \omega^3 
  \label{eq:einstein-A}
  \qc
  N
  = \frac{1}{e^{\beta\omega} - 1},
\end{equation}
and $\vq{d}$ is the transition matrix element of the dipole operator from the
excited to ground state. \Cref{eq:atom-lindblad} describes stimulated emission
with rate $\gamma_0 N$, spontaneous emission with rate $\gamma_0$, and
absorption with rate $\gamma_0 N$. Substituting the density operator in the form
of \cref{eq:bloch-dop} into the Lindblad equation gives the differential
equations
\begin{align}
  \dot{v}_x
  &= -\frac{\gamma}{2} v_x, \\
  \dot{v}_y
  &= -\frac{\gamma}{2} v_y, \\
  \dot{v}_z
  &= -\gamma v_z - \gamma_0,
\end{align}
where the total transition rate is
\begin{equation}
  \gamma
  = \gamma_0(2N + 1).
\end{equation}
The equilibrium solution is seen to be
\begin{equation}
  v_x^\text{eq}
  = v_y^\text{eq}
  = 0
  \qand
  v_z^\text{eq}
  = -\frac{1}{2N + 1},
\end{equation}
so the stationary population of the excited state is
\begin{equation}
  n_e^\text{eq}
  = \frac{1}{2}\qty(1 + v_z)
  = \frac{N}{2N + 1}
\end{equation}
and that of the ground state is
\begin{equation}
  n_g^\text{eq}
  = \frac{1}{2}\qty(1 - v_z)
  = \frac{N + 1}{2N + 1}.
\end{equation}
Thus
\begin{equation}
  \frac{n_e^\text{eq}}{n_g^\text{eq}}
  = \frac{N}{N + 1}
  = e^{-\beta\omega},
\end{equation}
which is the expected Boltzmann factor from statistical mechanics. We explicitly
see that $v_z$ decays to its equilibrium value with rate $\gamma$ because
\begin{equation}
  v_z(t)
  = v_z^\text{eq} - e^{-\gamma t}\qty(v_z^\text{eq} - v_z(0)).
\end{equation}


\section{The transverse-field Ising chain}

We now move to studying interacting spins. Of the many interacting spin models to
study, the \term{transverse-field Ising model}\footnote{%
  The translation to Pfeuty's constants is $g = \lambda^{-1} = J /
  2\Gamma$~\cite{pfeutyOnedimensionalIsingModel1970}.
}
\begin{equation}
  \ham
  = -J\sum_{i \in \ZZ_N} \qty(%
  \opr{S}_i^x\opr{S}_{i+1}^x
  + \frac{g}{2} \opr{S}_i^z
  ) \label{eq:transverse-ising}
\end{equation}
is an interesting candidate because it undergoes a so-called quantum phase
transition. The model is that $N$ spins with an anisotropic nearest-neighbor
interaction in the $x$-direction are subject to an external magnetic field in
the $z$-direction. We impose periodic boundary conditions so that $\pauli^a_{N}
= \pauli^a_0$. The parameter $g$ characterizes the strength of the magnetic
field relative to the interaction energy $J$. To get a sense for the scale of
$J$, take iron for reference.\footnote{%
  $\alpha$-Fe has $T_c = \SI{1044}{\K}$.
}
The interaction energy for two electron spins $a = \SI{1}{\AA}$ apart
is~\cite[p.~292]{griffithsIntroductionElectrodynamics2017}
\begin{equation}
  J_e
  = \frac{\mu_B^2 \mu_0}{4\pi a^3}
  \approx \SI{0.05}{\meV},
\end{equation}
while the corresponding exchange energy
is around~\cite[p.~325]{kittelIntroductionSolidState2004}
\begin{equation}
  J
  \approx \frac{1}{2} k_B T_c
  \approx \SI{45}{\meV}.
\end{equation}
We will keep the larger value in mind. This model has also been simulated
experimentally with trapped
ions~\cite{blattQuantumSimulationsTrapped2012,kimQuantumSimulationTransverse2011}.

The solution of the model by diagonalizing the Hamiltonian is straightforward
but somewhat involved, so it is given in \cref{ch:ising}. The idea is to
formally transform the interacting spins into free fermions. Each of the $N$
free fermion modes may be occupied or not, and is indexed by numbers $k$ that
are evenly spaced across the interval of the first Brillouin zone $[-\pi,\,
\pi]$ according to \cref{eq:fourierk}. Each mode has energy
\begin{equation}
  E_k
  = \sqrt{g^2 + 2g\cos k + 1}
  \label{eq:excitation-spectrum}
\end{equation}
relative to $J/2$, so the total energy in a given occupation state
$\op{\tq{n}}$ is
\begin{equation}
  E
  = \sum_k n_k E_k,
\end{equation}
where each $n_k \in \{0, 1\}$. Since setting only one $n_k = 1$ excites a single
mode from the ground state, \cref{eq:excitation-spectrum} also gives the
energies of elementary excitations from the ground state. It is plotted for
different values of $g$ in \cref{fig:ising-modes}.
\notebook{ising-modes}
From this, Pfeuty~\cite{pfeutyOnedimensionalIsingModel1970} has shown that the
ground state in the continuum limit $N \to \infty$ has
\begin{align}
  \ev{\pauli_i^z}
  &= G(0) \\
  \ev{\pauli_i^z\pauli_{i+n}^z}
  &= -G(n) G(-n),
  \label{eq:sz-corr}
\end{align}
where
\begin{align}
  G(n)
  &= L(n) + \frac{1}{g} L(n + 1) \\
  L(n)
  &= \frac{1}{\pi}\int_0^\pi \dd{k} \frac{\cos(kn)}{E_k}.
\end{align}
Pfeuty's $\ev{\pauli_i^z}$ is shown in \cref{fig:pfeuty-sz}.
\begin{figure}[ht]
  \centering
  \includegraphics[width=0.75\linewidth]{pfeuty}
  \caption{%
    $M_z = \ev{\pauli_i^z} / 2$ as a function of $g/2 = \Gamma /
    J$~\cite{pfeutyOnedimensionalIsingModel1970}. The dotted line is the result
    from mean-field theory. \textit{\small%
      Reprinted under
      \href{https://www.stm-assoc.org/intellectual-property/permissions/permissions-guidelines/}{\textsc{stm}
      guidelines}~\cite{STMPermissionsGuidelines}.
    }
  }\label{fig:pfeuty-sz}
\end{figure}
Similar investigation of \cref{eq:sz-corr} shows evidence of a quantum phase
transition at zero temperature.
\begin{itemize}
  \item For $0 \le g < 1$, the system is in a ferromagnetic ordered phase. The
    exchange interaction dominates and in the extreme case when $g = 0$, all
    spins align along the $x$-axis, giving the ground state twofold degeneracy.

  \item For $g = 1$, the system is said to undergo a quantum phase transition.
    The energy gap for excitations in \cref{fig:ising-modes} vanishes.

  \item For $g > 1$, the system is in a disordered phase. The transverse field
    dominates and in the extreme case where $g \to \infty$, all spins are
    aligned with the field along the $z$-axis.
\end{itemize}


\section{The weak-coupling limit for the Ising chain}\label{sec:weak-ising}

Does the presence of a quantum critical point influence the relaxation of the
Ising spin chain at finite temperature, where the phase transition vanishes?
What about at zero temperature, where the system may still relax due to
interactions with the electromagnetic vacuum? To approach these questions, we
carry out the weak-coupling limit for magnetic dipole interactions with the
electromagnetic field as the bath.

Before we begin, the system Hamiltonian \cref{eq:transverse-ising} requires
modification. We would like to study the effect of varying the parameter $g$ on
the relaxation rate of the open system. This means that we must be able to
compare the rates for two system Hamiltonians $\ham_1$ and $\ham_2$. However, we
are not interested in the absolute rate, but rather the rate relative to the
time scale of the system. For example, if $\ham_2 = 2\ham_1$, then a system with
$\ham_2$ will relax at a different rate than one with $\ham_1$. Both
Hamiltonians describe the same dynamics at different time scales. Therefore, we
first need to normalize the Hamiltonians with a map $\sopr{N}$ such that
\begin{equation}
  \sopr{N}(\ham_S)
  = \sopr{N}(\alpha\ham_S)
\end{equation}
for $\alpha > 0$. A related problem with \cref{eq:transverse-ising} is that we
must consider $g \to \infty$ which causes the system energy to diverge. To
enable comparison between different values of $g$, it would be convenient to
consider a parameter in the interval $[0,\, 1]$. Both of these problems are
solved by an appropriate interpolation between the extremal Hamiltonians
\begin{equation}
  \ham_x
  = -\sum_{i \in \ZZ_N}
  \pauli_i^x \pauli_{i+1}^x
  \qand
  \ham_z
  = -\sum_{i \in \ZZ_N}
  \pauli_i^z
\end{equation}
of the form
\begin{equation}
  \ham_S
  = -J(g)\sum_{i \in \ZZ_N}
  \qty(f(1-g) \pauli_i^x \pauli_{i+1}^x + f(g) \pauli_i^z).
\end{equation}
The map $\sopr{N}$ is implemented by the normalization $J(g)$ and the
compactification is implemented by the function $f(g)$. A natural choice is to
redefine $g$ and use spherical linear interpolation to rotate between $\ham_x$
and $\ham_z$ with
\begin{equation}
  \ham_S(g)
  = \sqrt{\abs{\ham_x}\cdot\abs{\ham_z}} \hat{\ham}_\text{int}(g)\qc
  0 \le g \le 1,
  \label{eq:Hinterp}
\end{equation}
where
\begin{equation}
  \ham_\text{int}(g)
  = -\qty(%
  \frac{\sin((1-g)\theta)}{\sin\theta} \hat{\ham}_x
  + \frac{\sin(g\theta)}{\sin\theta} \hat{\ham}_z
  ),
\end{equation}
a normalized operator $\hat{\opr{A}} = \opr{A} / \abs{A}$, and the angle
$\theta$ between $\ham_x$ and $\ham_z$ are defined with respect to the inner
product
\begin{equation}
  \lprod{\opr{A}}{\opr{B}}
  = \tr(\opr{A}^\dag \opr{B}).
\end{equation}
An overall magnitude $\sqrt{\abs{\ham_x}\cdot\abs{\ham_z}}$ is reintroduced so
that systems with different numbers of spins may be more easily compared below.
Because the energies are symmetric about zero ($\sum_i \tilde{E}_i = 0$), this
normalization of the Hamiltonians ensures that the energy variance $\sum_i
\tilde{E}_i^2$ is the same for all $g$. Thus the heat capacities
\begin{equation}
  C
  = \frac{1}{k_B T^2}\qty(\ev{\ham_S^2} - \ev{\ham_S}^2)
\end{equation}
of any two systems for different $g$ become the same as $\beta \to
0$~\cite[p.~112]{kardarStatisticalPhysicsParticles2007}. It is in this sense
that the normalization in \cref{eq:Hinterp} makes different systems behave
similarly.

Now we may add in the interaction with the bath. The field Hamiltonian is
\begin{equation}
  \ham_B
  = \sum_{\vq{k},\, \lambda} \omega_{\vq{k}}
  \opr{a}^\dag_{\vq{k},\lambda}\opr{a}_{\vq{k},\lambda},
  \label{eq:bath-hamiltonian}
\end{equation}
which is a collection of harmonic oscillators: one for each wavenumber $\vq{k}$
and polarization $\lambda$ of photons that make up the electromagnetic
field.\footnote{%
  The polarization $\lambda = \pm 1$ marks either right or left circular
  polarization.
}
The vacuum energy $\sum_{\vq{k},\, \lambda} \omega_{\vq{k}} / 2$ is dropped,
since it diverges in the continuum limit.

The interaction Hamiltonian for electron spins in a magnetic field is\footnote{%
  An electron has spin $s = 1/2$ and $g$-factor $g_e \approx 2$, so its magnetic
  moment $\mu_e = s g \mu_B$ is approximately the Bohr magneton $\mu_B$. The
  time dependence of the field is absorbed into the operators
  $\opr{a}_{\vq{k},\lambda}$, and the prefactor is chosen so that these
  operators are dimensionless, but $\vecopr{B}$ is not.
}
\begin{align}
  \ham_I
  &= -\int\dd{\vq{r}} \vecopr{\mu} \vdot \vecopr{B}
  \\
  \begin{split}
  &= -\int\dd{\vq{r}} \sum_i \mu_B \delta(\vq{r}_i) \vecopr{\pauli}_i \\
  &\vdot
  \im\sum_{\vq{k},\, \lambda} \sqrt{\frac{1}{2V\omega_{\vq{k}}}}
  \qty(%
  \qty(\vq{k} \cross \vq{e}_{\vq{k},\lambda})
  e^{\im\vq{k}\vdot\vq{r}} \opr{a}_{\vq{k},\lambda}
  - \qty(\vq{k} \cross \vq{e}^*_{\vq{k},\lambda})
  e^{-\im\vq{k}\vdot\vq{r}} \opr{a}^\dag_{\vq{k},\lambda}
  )
  \end{split}
  \\
  &= -\sum_{i,\,\mu} \pauli_i^\mu \opr{B}_i^\mu,
  \label{eq:mudotB}
\end{align}
where the components of the \term{magnetic field operator} evaluated at
$\vq{r}_i$ are\footnote{%
  This operator arises from quantizing the electromagnetic field. The
  polarization vector $\vq{e}_{\vq{k},\lambda}$ is $\vq{e}_{\vq{k},\pm 1} =
  \mp\qty(\vu{e}_{\vq{k},x} \pm \im\vu{e}_{\vq{k},y}) / \sqrt{2}$, with
  $\vu{e}_{\vq{k},x}$ and $\vu{e}_{\vq{k},y}$ perpendicular to $\vq{k}$. See the
  discussion of field quantization
  in~\cite[p.~506]{shankarPrinciplesQuantumMechanics2012} for more details.
}
\begin{equation}
  \opr{B}_i^\mu
  = \im\mu_B\sum_{\vq{k},\, \lambda} \sqrt{\frac{2\pi}{V\omega_{\vq{k}}}}
  \qty(%
  {\qty(\vq{k} \cross \vq{e}_{\vq{k},\lambda})}_\mu
  e^{\im\vq{k}\vdot\vq{r}_i} \opr{a}_{\vq{k},\lambda}
  - {\qty(\vq{k} \cross \vq{e}^*_{\vq{k},\lambda})}_\mu
  e^{-\im\vq{k}\vdot\vq{r}_i} \opr{a}^\dag_{\vq{k},\lambda}
  ).
  \label{eq:Bint-schrodinger}
\end{equation}
This magnetic field operator includes the electron's magnetic moment so that the
final rate $\gamma$ (\cref{eq:gamma-full}) carries the dimensions of frequency
in the dissipator (\cref{eq:dissipator}).

To summarize, the composite Hamiltonian that we will study is
\begin{align}
  \ham
  &= \ham_S \tp \idopr + \idopr \tp \ham_B + \ham_I \\
  \begin{split}
  &= -J(g)\sum_{i \in \ZZ_N}
  \qty(f(1-g) \pauli_i^x \pauli_{i+1}^x + f(g) \pauli_i^z) \tp \idopr
  + \idopr \tp \sum_{\vq{k},\, \lambda} \omega_{\vq{k}}
  \opr{a}^\dag_{\vq{k},\lambda}\opr{a}_{\vq{k},\lambda}
  - \sum_{i,\,\mu} \pauli_i^\mu \tp \opr{B}_i^\mu,
  \end{split}
  \label{eq:fullham}
\end{align}
where
\begin{itemize}
  \item $J(g) > 0$ for $0 \le g \le 1$,
  \item $f(g)$ is monotonic for $0 \le g \le 1$, $f(0) = 0$, and $f(1) = 1$,
    like for $f(g) = g$,
  \item The bath is a 3\textsc{d} continuum of modes according to
    \cref{eq:continuum-limit},
  \item The spin positions in \cref{eq:Bint-schrodinger} are $\vq{r}_i =
    ai\vu{z}$ for some spacing $a$, 
  \item The dipole approximation is valid. This requires that $a \ll
    c/\omega_{\vq{k}}$ for $\omega_{\vq{k}}$ up to the largest transition energy
    possible for $\ham_S$.
\end{itemize}
We will later nondimensionalize energy and time with respect to the system
energy scale $J(g)$ so that the rates $\gamma$ and $S$ in the Lindblad equation
(\cref{eq:microlindblad}) are relative to the system timescale $\tau_S = 1 /
J(g)$.

The magnetic field operator in the interaction picture is
\begin{align}
  \opr{B}_i^\mu(t)
  &= e^{\im\ham_B t} \opr{B}_i^\mu e^{-\im\ham_B t} \\
  &= \im\sum_{\vq{k},\, \lambda} \sqrt{\frac{2\pi\mu_B^2}{V\omega_{\vq{k}}}}
  \qty(%
  {\qty(\vq{k} \cross \vq{e}_{\vq{k},\lambda})}_\mu
  e^{\im(\vq{k}\vdot\vq{r}_i - \omega_{\vq{k}} t)} \opr{a}_{\vq{k},\lambda}
  - {\qty(\vq{k} \cross \vq{e}^*_{\vq{k},\lambda})}_\mu
  e^{-\im(\vq{k}\vdot\vq{r}_i - \omega_{\vq{k}} t)} \opr{a}^\dag_{\vq{k},\lambda}
  ).
  \label{eq:Bint-interaction}
\end{align}
Thus the spectral correlation tensor is
\begin{align}
  \Gamma_{i\mu,j\nu}(\omega)
  &= \int_0^\infty \dd{s} e^{\im\omega s}
  \ev{\opr{B}_i^\mu{(t)}^\dag \opr{B}_j^\nu(t - s)}
  \\
  &= -\frac{2\pi\mu_B^2}{V} \int_0^\infty \dd{s}
  \sum_{\vq{k},\, \vq{k}',\, \lambda,\, \lambda'}
  \sqrt{\frac{1}{\omega_{\vq{k}} \omega_{\vq{k'}}}}
  \\
  &\quad\left({\qty(\vq{k} \cross \vq{e}_{\vq{k},\lambda})}_\mu
    {\qty(\vq{k}' \cross \vq{e}_{\vq{k}',\lambda'})}_\nu
    e^{\im(\vq{k}\vdot\vq{r}_i - \omega_{\vq{k}} t + \vq{k}'\vdot\vq{r}_j -
    \omega_{\vq{k}'} (t-s) + \omega s)}
  \ev{%
    \opr{a}_{\vq{k},\lambda}\opr{a}_{\vq{k}',\lambda'}
  }\right.
  \nonumber \\
  &\quad-
    {\qty(\vq{k} \cross \vq{e}_{\vq{k},\lambda})}_\mu
    {\qty(\vq{k}' \cross \vq{e}^*_{\vq{k}',\lambda'})}_\nu
    e^{\im(\vq{k}\vdot\vq{r}_i - \omega_{\vq{k}} t - \vq{k}'\vdot\vq{r}_j +
    \omega_{\vq{k}'} (t-s) + \omega s)}
  \ev{%
    \opr{a}_{\vq{k},\lambda}\opr{a}^\dag_{\vq{k}',\lambda'}
  }
  \nonumber \\
  &\quad-
    {\qty(\vq{k} \cross \vq{e}^*_{\vq{k},\lambda})}_\mu
    {\qty(\vq{k}' \cross \vq{e}_{\vq{k}',\lambda'})}_\nu
    e^{-\im(\vq{k}\vdot\vq{r}_i - \omega_{\vq{k}} t - \vq{k}'\vdot\vq{r}_j +
    \omega_{\vq{k}'} (t-s) - \omega s)}
  \ev{%
    \opr{a}^\dag_{\vq{k},\lambda}\opr{a}_{\vq{k}',\lambda'}
  }
  \nonumber \\
  &\quad+ \left.
    {\qty(\vq{k} \cross \vq{e}^*_{\vq{k},\lambda})}_\mu
    {\qty(\vq{k} \cross \vq{e}^*_{\vq{k}',\lambda'})}_\nu
    e^{-\im(\vq{k}\vdot\vq{r}_i - \omega_{\vq{k}} t + \vq{k}'\vdot\vq{r}_j -
    \omega_{\vq{k}'} (t-s) - \omega s)}
  \ev{%
    \opr{a}^\dag_{\vq{k},\lambda}\opr{a}^\dag_{\vq{k}',\lambda'}
  }\right).
  \nonumber
\end{align}

We will consider the bath in the thermal state
\begin{equation}
  \dop_B
  = \frac{e^{-\beta\ham_B}}{\tr e^{-\beta\ham_B}}
  = \prod_{\vq{k},\,\lambda} \qty(1 - e^{-\beta\omega_{\vq{k}}})
  e^{-\beta\omega_{\vq{k}} \opr{a}^\dag_{\vq{k},\lambda}
  \opr{a}_{\vq{k},\lambda}}.
\end{equation}
Since $\comm{\opr{a}_{\vq{k},\lambda}}{\opr{a}^\dag_{\vq{k}',\lambda'}} =
\delta_{\vq{k}\vq{k}'}\delta_{\lambda\lambda'} \idopr$,
\begin{align}
  \ev{\opr{a}^\dag_{\vq{k},\lambda} \opr{a}_{\vq{k}',\lambda'}}
  &= {\tr(e^{-\beta\ham_B})}^{-1} \tr(e^{-\beta\ham_B}
  \opr{a}^\dag_{\vq{k},\lambda} \opr{a}_{\vq{k}',\lambda'})
  \\
  &= {\tr(e^{-\beta\ham_B})}^{-1} \tr(e^{-\beta\ham_B}
  \opr{a}_{\vq{k}',\lambda'} \opr{a}^\dag_{\vq{k},\lambda})
  - \delta_{\vq{k}\vq{k}'}\delta_{\lambda\lambda'}
  \\
  &= {\tr(e^{-\beta\ham_B})}^{-1} \tr(
  e^{\beta \omega_{\vq{k}}}
  \opr{a}_{\vq{k}',\lambda'} e^{-\beta\ham_B} \opr{a}^\dag_{\vq{k},\lambda})
  - \delta_{\vq{k}\vq{k}'}\delta_{\lambda\lambda'}
  \\
  &= e^{\beta \omega_{\vq{k}}}
  \ev{\opr{a}^\dag_{\vq{k},\lambda} \opr{a}_{\vq{k}',\lambda'}}
  - \delta_{\vq{k}\vq{k}'}\delta_{\lambda\lambda'},
\end{align}
so we find that
\begin{equation}
  \ev{\opr{a}^\dag_{\vq{k},\lambda} \opr{a}_{\vq{k}',\lambda'}}
  = \delta_{\vq{k}\vq{k}'}\delta_{\lambda\lambda'} n_B(\omega_{\vq{k}}),
\end{equation}
where
\begin{equation}
  n_B(\omega)
  = \frac{1}{e^{\beta\omega} - 1}.
  \label{eq:bose-number}
\end{equation}
Similarly,
\begin{align}
  \ev{\opr{a}_{\vq{k},\lambda} \opr{a}_{\vq{k}',\lambda'}}
  &= \ev{\opr{a}^\dag_{\vq{k},\lambda} \opr{a}^\dag_{\vq{k}',\lambda'}}
  = 0
  \\
  \ev{\opr{a}_{\vq{k},\lambda} \opr{a}^\dag_{\vq{k}',\lambda'}}
  &= \delta_{\vq{k}\vq{k}'} \delta_{\lambda\lambda'}
  \qty(1 + n_B(\omega_{\vq{k}})).
\end{align}
Then for a thermal bath, the spectral correlation tensor becomes
\begin{align}
  \begin{split}
  \Gamma_{i\mu,j\nu}(\omega)
  &= \frac{2\pi\mu_B^2}{V} \int_0^\infty \dd{s}
  \sum_{\vq{k},\,\lambda}
  \frac{1}{\omega_{\vq{k}}}
  \\
  &\quad\left(
    {\qty(\vq{k} \cross \vq{e}_{\vq{k},\lambda})}_\mu
    {\qty(\vq{k} \cross \vq{e}^*_{\vq{k},\lambda})}_\nu
    e^{\im(\vq{k}\vdot(\vq{r}_i - \vq{r}_j) + s(\omega - \omega_{\vq{k}}))}
    \qty(1 + n_B(\omega_{\vq{k}}))
    \right.
  \\
  &\quad+ \left.
    {\qty(\vq{k} \cross \vq{e}^*_{\vq{k},\lambda})}_\mu
    {\qty(\vq{k} \cross \vq{e}_{\vq{k},\lambda})}_\nu
    e^{-\im(\vq{k}\vdot(\vq{r}_i - \vq{r}_j) - s(\omega + \omega_{\vq{k}}))}
    n_B(\omega_{\vq{k}})
    \right).
\end{split}\label{eq:Gamma-precontinuum}
\end{align}

To evaluate \cref{eq:Gamma-precontinuum}, we now consider a chain of $N$ spins
along the $z$-axis, so that $\vq{r}_i = r_i \vu{z}$.\footnote{We could consider
any axis given the spherical symmetry, but the $z$-axis is the simplest to
evaluate.} Then $\vq{k} \vdot (\vq{r}_i - \vq{r}_j) = k_z \Delta r_{ij}$.

In the continuum limit,
\begin{equation}
  \frac{1}{V}\sum_{\vq{k}}
  \mapsto
  \int \frac{\dd{\vq{k}}}{{(2\pi)}^3}
  = \frac{1}{{(2\pi)}^3} \int_0^\infty \dd{\omega_k} \omega_k^2
  \int\dd{\Omega},
  \label{eq:continuum-limit}
\end{equation}
where the integral over solid angle is
\begin{equation}
  \int\dd{\Omega}
  = \int\dd{\phi}\int\dd{\theta}\sin\theta.
  \label{eq:solid-angle}
\end{equation}
To apply this limit to \cref{eq:Gamma-precontinuum}, we first note that
\begin{align}
  \sum_\lambda {\qty(\vq{k} \cross \vq{e}_{\vq{k},\lambda})}_\mu
  {\qty(\vq{k} \cross \vq{e}^*_{\vq{k},\lambda})}_\nu
  &= \sum_{abcd} \varepsilon_{\mu ab}\varepsilon_{\nu cd} k^a k^c 
  \sum_\lambda e_{\vq{k},\lambda}^b {\qty(e_{\vq{k},\lambda}^d)}^*
  \\
  &= \sum_{abcd} \varepsilon_{\mu ab}\varepsilon_{\nu cd} k^a k^c 
  \qty(\delta_{bd} - \frac{k^b k^d}{k^2})
  \\
  &= \sum_{abc} \varepsilon_{\mu ab}\varepsilon_{\nu cb} k^a k^c 
  \\
  &= \sum_{ac} \qty(\delta_{\mu\nu}\delta_{ac} - \delta_{\mu c}\delta_{a\nu}) k^a k^c 
  \\
  &= k^2 \delta_{\mu\nu} - k^\mu k^\nu.
\end{align}
Thus
\begin{equation}
  \int\dd{\Omega} e^{\pm\im k_z \Delta r_{ij}}
  \sum_\lambda {\qty(\vq{k} \cross \vq{e}_{\vq{k},\lambda})}_\mu
  {\qty(\vq{k} \cross \vq{e}^*_{\vq{k},\lambda})}_\nu
  = \frac{8\pi\omega_k^2}{3} \delta_{\mu\nu}
  G_{\nu}\qty(\omega_k\Delta r_{ij}),
  \label{eq:angular-integral}
\end{equation}
where
\begin{equation}
  G_{\nu}(u)
  = 3\qty(\delta_{\nu z} - \frac{\delta_{\nu x} + \delta_{\nu y}}{2})
  \frac{\sinc u - \cos u}{u^2}
  + \frac{3}{2}\qty(\delta_{\nu x} + \delta_{\nu y}) \sinc u.
\end{equation}
Note that in the dipole approximation
\begin{equation}
  \lim_{u \to 0} G_\nu(u)
  = \qty(\delta_{\nu z} - \frac{\delta_{\nu x} + \delta_{\nu y}}{2})
  + \frac{3}{2}\qty(\delta_{\nu x} + \delta_{\nu y})
  = 1.
\end{equation}
Now \cref{eq:angular-integral} gives that the continuum limit of the spectral
correlation tensor for the spin chain is
\begin{align}
  \begin{split}
  \Gamma_{i\mu,j\nu}(\omega)
  &= \delta_{\mu\nu} \frac{2\mu_B^2}{3\pi}
  \int_0^\infty \dd{\omega_k} \omega_k^3
  G_{\nu}\qty(\omega_k \Delta r_{ij})
  \\
  &\quad\qty(%
  \qty(1 + n_B(\omega_k)) \int_0^\infty \dd{s} e^{\im s(\omega - \omega_k)}
  + n_B(\omega_k) \int_0^\infty \dd{s} e^{\im s(\omega + \omega_k)}
  ).
  \end{split}
\end{align}
We now use that
\begin{equation}
  n_B(-\omega)
  = -\qty(1 + n_B(\omega))
\end{equation}
and 
\begin{equation}
  \int_0^\infty \dd{s} e^{-\im\omega s}
  = \pi\delta(\omega) - \im\pv\frac{1}{\omega},
\end{equation}
where $\pv$ denotes the Cauchy principal value, to find
\begin{align}
  \Gamma_{i\mu,j\nu}(\omega)
  &= \frac{1}{2}\gamma_{i\mu,j\nu}(\omega) + \im S_{i\mu,j\nu}(\omega),
  \\
  \shortintertext{where}
  \gamma_{i\mu,j\nu}(\omega)
  &= \delta_{\mu\nu} \frac{4\mu_B^2 \omega^3}{3}
  G_{\nu}\qty(\abs{\omega} \Delta r_{ij})
  (1 + n_B(\omega))
  \label{eq:gamma-full}
  \\
  S_{i\mu,j\nu}(\omega)
  &= \delta_{\mu\nu} \frac{2\mu_B^2}{3\pi}
  \pv\int_0^\infty \dd{\omega_k} \omega_k^3
  G_{\nu}\qty(\omega_k \Delta r_{ij})
  \qty(\frac{1 + n_B(\omega_k)}{\omega - \omega_k}
  + \frac{n_B(\omega_k)}{\omega + \omega_k}).
  \label{eq:S-full}
\end{align}

We will now nondimensionalize $\gamma$ and $S$. Starting with $\gamma$, we first
apply the dipole approximation to find $\gamma_{i\mu,j\nu}(\omega) = \delta_{ij}
\delta_{\mu\nu} \gamma(\omega)$, where\footnote{%
  At zero temperature, $\gamma(\omega)$ becomes the \term{Einstein $A$
  coefficient}. Compare this to the analogous result (\cref{eq:einstein-A}) for
  electric dipole radiation.
}
\begin{equation}
  \gamma(\omega)
  = \frac{4}{3}\mu_B^2 \omega^3 (1 + n_B(\omega)).
  \label{eq:gamma-dipole}
\end{equation}
Letting $\tilde{\gamma} = \gamma \tau_S$ and $\tilde{\omega} = \omega\tau_S$
gives
\begin{equation}
  \tilde{\gamma}(\tilde{\omega})
  = {\qty(\frac{\tau_0}{\tau_S})}^2 \tilde{\omega}^3
  \qty(1 + \frac{1}{e^{\tau_B \tilde{\omega} / \tau_S} - 1})
  \label{eq:gamma-nondim}
\end{equation}
in terms of the thermal correlation time $\tau_B = \beta$ and the
\term{vacuum magnetic timescale}
\begin{equation}
  \tau_0
  = \sqrt{\frac{4}{3}\mu_B^2}
  = \SI{6.35e-23}{\s}.
  \label{eq:tau0}
\end{equation}
In the high-temperature limit where $\tau_B\tilde{\omega}/\tau_S \ll 1$,
$\tilde{\gamma}$ becomes
\begin{equation}
  \tilde{\gamma}(\tilde{\omega})
  \approx \frac{\tau_0^2}{\tau_B \tau_S} \tilde{\omega}^2.
\end{equation}
The other limit where $\tau_B\tilde{\omega}/\tau_S \gg 1$ motivates the
definition of the typical zero-temperature decay rate
\begin{equation}
  \tilde{\gamma}_0
  = \lim_{\tau_B \to \infty} \tilde{\gamma}(1)
  = {\qty(\frac{\tau_0}{\tau_S})}^2.
\end{equation}

What about $\tilde{S} = S\tau_S$? Applying the dipole approximation and letting
$\tilde{\omega} = \omega\tau_S$ gives 
\begin{equation}
  \tilde{S}(\tilde{\omega})
  = \frac{1}{2\pi} {\qty(\frac{\tau_0}{\tau_S})}^2
  \pv\int_0^\infty \dd{\tilde{\omega}_k} \tilde{\omega}_k^3
  \qty(\frac{1 + \tilde{n}_B(\tilde{\omega}_k)}{\tilde{\omega} - \tilde{\omega}_k}
  + \frac{\tilde{n}_B(\tilde{\omega}_k)}{\tilde{\omega} + \tilde{\omega}_k}),
  \label{eq:Snondim}
\end{equation}
where
\begin{equation}
  \tilde{n}_B(\tilde{\omega})
  = \frac{1}{e^{\tau_B \tilde{\omega} / \tau_S} - 1}.
\end{equation}
To avoid the divergence of \cref{eq:Snondim}, we introduce an upper frequency
cutoff $\tilde{\Omega}$. Physically, one expects the coupling to high-frequency
modes of the bath to weaken.\footnote{%
  This is not the case for the electromagnetic field, but in other contexts, one
  may view the bath as merely an effective model, and this cutoff is used to
  describe the effective interaction.
}
For simplicity, we just set the upper limit of the integral to
$\tilde{\Omega}$.\footnote{%
  This hard cutoff of setting the upper limit to $\tilde{\Omega}$ is performed
  in~\cite[p.~265]{opensys}. In general, one considers a function called the
  spectral density that is related to the coupling in the interaction
  Hamiltonian. When this function includes
  \begin{equation}
    \frac{\tilde{\Omega}^2}{\tilde{\Omega}^2 + {\omega}^2},
  \end{equation}
  it is said to have a Lorentz-Drude cutoff, which is sometimes considered
  instead of the hard cutoff~\cite[p.~175]{opensys}. The low-frequency behavior
  of our spectral density classifies it as an Ohmic spectral density, which is
  the most common type to be considered because it arises from electromagnetic
  interactions in three dimensions. The exact functional form of the cutoff does
  not matter much because we ensure that the system transitions at frequencies
  far below the cutoff.
}
If we set the frequency cutoff $\tilde{\Omega} \gg \tilde{\omega}$ to be
far above any system frequency, we have the following limits.
\begin{itemize}
  \item In the high-temperature limit where $\tau_B\tilde{\Omega} / \tau_S \ll
    1$, $\tilde{S}$ simplifies to\footnote{%
      To evaluate these principal value integrals, it suffices to just ignore
      the pole at $u = 1$.
    }
    \begin{align}
      \tilde{S}(\tilde{\omega})
      &\approx \frac{1}{2\pi} \frac{\tau_0^2}{\tau_B\tau_S}
      \pv\int_0^{\tilde{\Omega}} \dd{\tilde{\omega}_k} \tilde{\omega}_k^2
      \qty(\frac{1}{\tilde{\omega} - \tilde{\omega}_k}
      + \frac{1}{\tilde{\omega} + \tilde{\omega}_k}) \\
      &= \frac{\tilde{\omega}^2}{\pi} \frac{\tau_0^2}{\tau_B\tau_S}
      \pv\int_0^{\tilde{\Omega} / \tilde{\omega}} \dd{u}
      \frac{u^2}{1 - u^2} \\
      &= \frac{\tau_0^2}{\tau_B\tau_S} \frac{\tilde{\omega}}{\pi}
      \qty(\tilde{\omega}\acoth\qty(\frac{\tilde{\omega}}{\tilde{\Omega}})
      - \tilde{\Omega}) \\
      &\approx -\frac{\tau_0^2}{\tau_B\tau_S}
      \frac{\tilde{\Omega}}{\pi}\tilde{\omega}.
    \end{align}

  \item In the low-temperature limit where $\tau_B\tilde{\Omega} / \tau_S \gg
    1$, $\tilde{S}$ simplifies to
    \begin{align}
      \tilde{S}(\tilde{\omega})
      &\approx \frac{1}{2\pi} {\qty(\frac{\tau_0}{\tau_S})}^2
      \pv\int_0^{\tilde{\Omega}} \dd{\tilde{\omega}_k}
      \frac{\tilde{\omega}_k^2}{\tilde{\omega} - \tilde{\omega}_k} \\
      &= \frac{\tilde{\omega}^2}{2\pi} {\qty(\frac{\tau_0}{\tau_S})}^2
      \pv\int_0^{\tilde{\Omega} / \tilde{\omega}} \dd{u}
      \frac{u^2}{1 - u} \\
      &= -\frac{1}{2\pi} {\qty(\frac{\tau_0}{\tau_S})}^2 \qty(%
      \frac{\tilde{\Omega}^2}{2} + \tilde{\Omega}\tilde{\omega} -
      \tilde{\omega}^2
      \log\qty(\frac{\tilde{\omega}}{\tilde{\Omega} - \tilde{\omega}})
      ) \\
      &\approx -\frac{1}{2\pi} {\qty(\frac{\tau_0}{\tau_S})}^2 \qty(%
      \frac{\tilde{\Omega}^2}{2} + \tilde{\Omega}\tilde{\omega}
      ).
    \end{align}
\end{itemize}
In both cases, $\tilde{S}$ is small if $\tau_0\Omega \ll 1$, which we will
assume. Because the Lamb-shift Hamiltonian (\cref{eq:lamb-shift}) scales with
$\tilde{S}$, it may then be neglected.

Now that we have calculated $\gamma$, we may verify that the weak-coupling limit
is valid. We already know that the bath is changed little by the system due to
its infinite size, that $\ev{\opr{B}_i^\mu(t)} = 0$, and that the secular
approximation holds for the numbers of spins that we will consider because
$\tau_0$ is so small. It remains to verify that the reservoir correlation
functions decay over a time scale that is much less than the relaxation time
$\tau_R$.\footnote{%
  Analogous results for the electric field are given in~\cite[p.~574]{opensys}.
}
The point of this thesis is to find the relaxation time $\tau_R$, so we cannot
make a direct comparison. However, on physical grounds we do expect that the
relaxation time is much slower than the system time $\tau_S$ for all realistic
temperatures.\footnote{%
  Also, if we instead suppose that this is true, then we expect the relaxation
  time to be on the order of $\gamma^{-1}(1)$. For the case to be otherwise, we
  must have $\tau_B \approx \tau_0$, which corresponds to a temperature of
  \SI{1.2e11}{\K} or higher.
}
Recall that the rate $\gamma(\omega)$ is given by the Fourier transform
(\cref{eq:gammas}) of the reservoir correlation functions
$\ev{\opr{B}_i^\mu{(s)}^\dag\opr{B}_j^\nu(0)}$, which are the same for all sites
and polarizations. The correlation functions are then the inverse Fourier
transform of $\gamma(\omega)$:
\begin{equation}
  \ev{\opr{B}_i^\mu{(s)}^\dag\opr{B}_j^\nu(0)}
  = \frac{1}{2\pi}\int_{-\infty}^\infty \dd{\omega} e^{-\im\omega s}
  \gamma(\omega).
\end{equation}
Using the dimensionless time $\tilde{s} = s / \tau_S$ and introducing the
parameter $\eta = \tau_B / \tau_S$, we find that
\begin{align}
  \frac{1}{\tau_S^2}\ev{\opr{B}_i^\mu{(\tilde{s})}^\dag\opr{B}_j^\nu(0)}
  &= \frac{1}{2\pi}\int_{-\infty}^\infty \dd{\tilde{\omega}}
  e^{-\im\tilde{\omega} \tilde{s}}
  \tilde{\gamma}(\tilde{\omega})
  \\
  &= {\qty(\frac{\tau_0}{\tau_S})}^2 \frac{1}{2\pi}
  \int_{-\infty}^\infty \dd{\tilde{\omega}}
  e^{-\im\tilde{\omega} \tilde{s}}
  \tilde{\omega}^3
  \qty(1 + \frac{1}{e^{\eta\tilde{\omega}} - 1})
  \\
  &= {\qty(\frac{\tau_0}{\tau_S})}^2 
  \im^3 \dv[3]{\tilde{s}}
  \frac{1}{2\pi} \int_{-\infty}^\infty \dd{\tilde{\omega}}
  e^{-\im\tilde{\omega} \tilde{s}}
  \frac{e^{\eta \tilde{\omega}}}{e^{\eta \tilde{\omega}} - 1}
  \\
  &= {\qty(\frac{\tau_0}{\tau_S})}^2 
  \im^3 \dv[3]{\tilde{s}}
  \frac{1}{2\pi} \int_{-\infty}^\infty \dd{\tilde{\omega}}
  e^{-\im\tilde{\omega} (\tilde{s} + \im\eta / 2)}
  \frac{1}{2}\csch(\frac{\eta \tilde{\omega}}{2})
  \\
  &= {\qty(\frac{\tau_0}{\tau_S})}^2 
  \im^3 \dv[3]{\tilde{s}} \qty(%
  -\frac{\im}{2\eta}\tanh(\frac{\pi}{\eta}\qty(\tilde{s} + \frac{\im\eta}{2}))
  )
  \\
  &= {\qty(\frac{\tau_0}{\tau_S})}^2 
  \im^3 \dv[3]{\tilde{s}}
  \qty(-\frac{\im}{2\eta}\coth(\frac{\pi\tilde{s}}{\eta}))
  \\
  &= {\qty(\frac{\tau_0}{\tau_S})}^2 \frac{\pi^3}{\eta^4}
  \sinh{\qty(\frac{\pi \tilde{s}}{\eta})}^{-4} \qty(%
  1 + 2\cosh{\qty(\frac{\pi\tilde{s}}{\eta})}^2
  ).
\end{align}
At low temperatures $\tilde{s} \ll \eta$ and
\begin{align}
  \frac{1}{\tau_S^2}\ev{\opr{B}_i^\mu{(\tilde{s})}^\dag\opr{B}_j^\nu(0)}
  &= {\qty(\frac{\tau_0}{\tau_S})}^2 \frac{3}{\pi \tilde{s}^4},
\end{align}
while at high temperatures $\tilde{s} \gg \eta$ and
\begin{align}
  \frac{1}{\tau_S^2}\ev{\opr{B}_i^\mu{(\tilde{s})}^\dag\opr{B}_j^\nu(0)}
  &= {\qty(\frac{\tau_0}{\tau_S})}^2 \frac{8\pi^3}{\eta^4}
  e^{-2\pi\tilde{s} / \eta} \\
  &= \qty(\frac{\tau_0^2}{\tau_B \tau_S}) \frac{8\pi^3}{\eta^3}
  e^{-2\pi\tilde{s} / \eta}.
\end{align}
Thus if $\tau_B \ll \tau_S$, the reservoir correlation time is $\tau_C = \tau_B
/ 2\pi$, which is comparable to the thermal correlation time. In either case,
the correlation functions decay quickly for $\tilde{s} > 1$. Since $\tau_S \ll
\tau_R$, this establishes the validity of the weak-coupling limit.

\end{document}

