\chapter*{Conclusion}
\addcontentsline{toc}{chapter}{Conclusion}
\chaptermark{Conclusion}
\markboth{Conclusion}{Conclusion}

We will first discuss the time evolution, energy levels, and spaghetti diagrams.
The time evolution across all systems looks just as one expects. Motion is
generally oscillatory like for the two-level atom, but as $N$ increases higher
frequencies are available for the system evolution and manifest like the wiggles
in \cref{fig:time-evolution-5}.

The exponential fits at $\eta = 10$ are reasonable, but the discrepancies from
the true envelopes serve as a reminder that the spin relaxation plots to follow
are to be taken with a grain of salt.

The energy levels have a parity dependence as expected from the solution of the
full transverse-field Ising model, but generally show that the energy levels for
$\hat{\ham}_x$ as $g \to 0$ are far more degenerate than those for
$\hat{\ham}_z$ as $g \to 1$. This offers an explanation of the trends for the
relaxation rates. Take \cref{fig:energy-levels-5} for example. For small $g$,
roughly half of the frequencies are zero and half of the frequencies are two
(neglecting the lowest eigenvalue below $-4$). For large $g$, the energies are
divided up more. For large $\eta$, \cref{eq:gamma-nondim} is like
$\tilde{\omega}^3$, and we see that the small $g$ case has the larger sum of
cubes. This difference in the energy levels for different $g$ may explain the
general shape of the limiting rates discussed below as $\eta \to \infty$. The
effect only arises as $N$ grows because there are an insufficient number of
eigenvalues to split at small $N$. At $N = 2$, the maximal similarity between
the different sides of the energy level diagram may explain the comparable decay
rates found as well.

The spaghetti diagrams are difficult to decipher. Again, since the
eigenoperators $\opr{V}_i$ of $\sopr{L}$ are not even states, it is difficult to
ascribe meaning to a particular eigenvalue (noodle). Looking to
\cref{fig:spin-spectrum-3}, the overlaid spin relaxation rate for $\eta = 10$
starts to blend in with the spaghetti near $g = 1$, but behaves differently
elsewhere. Here and in later spaghetti diagrams, the overlaid rate appears to be
influenced by the presence of noodles much higher, but tends to remain within an
order of magnitude of the value at $g = 1$. The large magnitude of the spaghetti
for $N = 4$ appears to be a consequence of the parity dependence in the
transverse-field Ising model. The full diagram for $N = 6$ could not be
computed, but its largest value at $g = 0.01$ was found to be $586.9$, which is
similar to the general magnitude for $N = 4$.

Overall, the single-spin relaxation plots demonstrate that relaxation rates do
depend on the system being considered at all temperatures. It should be
reiterated that these plots show the dimensionless rates, so while the
difference between $g = 0$ and $g = 1$ at high temperature in
\cref{fig:spin-relaxation-4} is about a factor of six, this is a multiplier for
an overall rate $\tilde{\gamma}_0$ that may be quite large for any practical
purposes. Given that caveat, we identify two general types of behavior depending
on temperature. Across all numbers of spins, the high-temperature relaxation
rates approach a fixed shape that varies for each $N$:
\begin{itemize}
  \item For $N = 2$ there is a minimum relaxation rate around $g = 0.65$.
  \item For $N = 3$ there is a maximum relaxation rate around $g = 0.40$.
  \item For $N = 4$ there is a remnant of a peak but the rate generally
    decreases with increasing $g$.
  \item For $N = 5$ there is a maximum relaxation rate around $g = 0.45$.
\end{itemize}

The story at low temperature is a bit different.
\begin{itemize}
  \item For $N = 2$ the rates appear to quickly reach a parabola-like shape for
    $\eta \approx 10$. Afterwards, the fitting becomes unstable as demonstrated
    by the reversal of the temperature dependence for the rates and the large
    error ribbons in the last plot.
  \item For $N = 3$ rates decrease somewhat linearly until $g = 0.5$ and then
    level out for $\eta \approx 10$. The later plots at lower temperature
    demonstrate the convergence of the rates for small $g$ but show instability
    in the fit for $g$ near $0.5$.
  \item For $N = 4$ the convergence to the previous line-flat shape is quicker
    and the transition from low to high $g$ across $g = 0.5$ is smoothed. The
    fit is stable and the shape remains for very low temperatures in the last
    plot.
  \item For $N = 5$ the flat part of the curve has lifted to reveal a parabola
    shape with a minimum at about $g = 0.45$.
\end{itemize}

In short, the answer to the question ``How much does varying the parameter $g$
change the relaxation rate of $\ev{\pauli_1^z}$?'' is by at most a factor of
ten. Given the deviation of the single-spin relaxation rate from the rest of the
spaghetti, it is unclear how well this result generalizes to other systems.
Extensions of this project to larger systems and a more systematic exploration
of other interesting systems or random ensembles of Hamiltonians are left as
directions for future work.

