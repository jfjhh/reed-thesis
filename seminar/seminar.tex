\documentclass[12pt,c]{beamer}

\usepackage[normalem]{ulem}
\usepackage{multicol}
\usepackage{wasysym}

\usepackage{mathtools} % Load before newtxmath
\mathtoolsset{%
  mathic,
}
\usepackage[%
math-style=ISO,
bold-style=ISO,
sans-style=italic,
nabla=upright,
partial=upright,
]{unicode-math}


\definecolor{rubric}{rgb}{0.7,0.05,0.0}

\usecolortheme[snowy]{owl}
\setbeamercolor{title}{fg=rubric}
\setbeamerfont{frametitle}{shape=\itshape}
\setbeamercolor{frametitle}{bg=white}
\setbeamertemplate{frametitle}[default]%[center]
\setbeamertemplate{caption}[numbered]
\setbeamertemplate{caption label separator}{: }
\setbeamercolor{caption name}{fg=normal text.fg}
\setbeamercolor{section in toc}{fg=normal text.fg}
\setbeamercolor{footnote}{fg=gray}
\setbeamercolor{alerted text}{fg=rubric}
\setbeamercolor*{bibliography item}{fg=normal text.fg}
\setbeamercolor*{bibliography entry author}{fg=normal text.fg}
\setbeamercolor*{bibliography entry title}{fg=rubric}
\setbeamercolor*{bibliography entry location}{fg=normal text.fg}
\setbeamercolor*{bibliography entry note}{fg=normal text.fg}
\addtobeamertemplate{navigation symbols}{}{%
    \usebeamerfont{footline}%
    \color{normal text.fg}
    \hspace{1em}%
    \insertframenumber\,/\,\inserttotalframenumber%
}

\usepackage[sorting=none,doi=false,url=false,isbn=false]{biblatex}
\addbibresource{references.bib}
\renewcommand*{\bibfont}{\footnotesize}

% Fonts
\usefonttheme[stillsansserifsmall]{serif} % So beamer won't change fonts
\usepackage[full]{textcomp} % to get the right copyright, etc.
\usepackage{libertinus-otf}
\usepackage[T1]{fontenc} % LY1 also works
\setmainfont[Numbers={OldStyle,Proportional}]{fbb}
\setsansfont[Scale=0.95]{Cabin}
\usepackage[supstfm=fbb-Regular-sup-t1]{superiors}
% \setmonofont[Scale=MatchLowercase]{DejaVu Sans Mono}
\setmonofont[Scale=MatchLowercase]{Iosevka} % We need unicode
\usepackage[cal=boondoxo,bb=boondox,frak=boondox]{mathalfa}

\newcommand{\spacedsc}{\scshape\addfontfeatures{LetterSpace=5}}

% Use microtype if available
\IfFileExists{microtype.sty}{\usepackage{microtype}}{}
\hypersetup{%
  pdfborder={0 0 0},
  breaklinks=true,
}
\urlstyle{same} % don't use monospace font for urls

% \setlength{\parindent}{0pt}
% \setlength{\parskip}{6pt plus 2pt minus 1pt}
% \setlength{\emergencystretch}{3em}  % prevent overfull lines
\providecommand{\tightlist}{%
  \setlength{\itemsep}{0pt}\setlength{\parskip}{0pt}}
\setcounter{secnumdepth}{0}

\usepackage{physics}
\usepackage{siunitx}
\usepackage{pgfplots}
\usepackage{tikz}
\usetikzlibrary{arrows}
\tikzset{% Animate with beamer overlays
  invisible/.style={opacity=0},
  visible on/.style={alt={#1{}{invisible}}},
  alt/.code args={<#1>#2#3}{%
    \alt<#1>{\pgfkeysalso{#2}}{\pgfkeysalso{#3}} % \pgfkeysalso doesn't change the path
  },
}

\graphicspath{{./figs/}}

%% Math packages
\usepackage{physics}
\usepackage{siunitx}


%% Math commands

% Semantic commands
\newcommand{\defeq}{\equiv}
\newcommand{\iso}{\cong}
\newcommand{\by}[1]{\qq{#1}}
\newcommand{\byeq}[1]{\by{\cref{#1}}}

% Variant symbols
\renewcommand{\le}{\leqslant}
\renewcommand{\ge}{\geqslant}
\renewcommand\phi\varphi%
\renewcommand\epsilon\varepsilon%

% Common algebraic objects
\newcommand\ZZ{\mathbb{Z}}
\newcommand\QQ{\mathbb{Q}}
\newcommand\CC{\mathbb{C}}
\newcommand\RR{\mathbb{R}}

% Named operations and special constants
\newcommand\im{\mathrm{i}\mkern1mu}
% \newcommand\im{i\mkern1mu}

% Operators
\newcommand\tp\otimes%
\newcommand\opr\symsf%
\newcommand\vecopr\symbfsf%
% \newcommand{\opr}[1]{\mathnormal{#1}}%
\newcommand\idopr{\opr{1}}
\newcommand\zopr{\opr{0}}
\newcommand\sopr\mathcal%
\newcommand\idsopr{\mathbb{1}}
\newcommand\zsopr{\mathbb{0}}
\newcommand\herm{\operatorname{\mathrm{He}}}
\newcommand\sgn{\operatorname{\mathrm{sgn}}}
\newcommand\sinc{\operatorname{\mathrm{sinc}}}
\newcommand\atanh{\operatorname{\mathrm{atanh}}}
\newcommand\acoth{\operatorname{\mathrm{acoth}}}

% Categories
\newcommand\cat\symsf%

% Linear algebra
\newcommand\tq[1]{\symbf{#1}} % Tuples
\newcommand\vq[1]{\symbf{#1}} % Vectors
\providecommand\vu{}
\renewcommand\vu[1]{\hat{\vq{#1}}} % Unit vectors
\newcommand\mq[1]{\symbf{#1}} % Matrices (not transformations)
\newcommand\trans{{\symsfup{T}}} % this is the one case when we want it upright

% Quantum mechanics
\newcommand\hilb{\mathcal{H}}
\newcommand\liou{\mathcal{L}}
\newcommand\bdd{\mathcal{B}}
\newcommand\ham{\opr{H}}
\newcommand\dop{\opr{ρ}}
\newcommand\pauli{\opr{σ}}
\newcommand\paulivec{\vecopr{σ}}
\newcommand\spin{\opr{S}}
\newcommand\spinvec{\vecopr{S}}
\newcommand\tprod\otimes%
\newcommand\lprod\ip%
% \newcommand\lprod[2]{\qty(#1,#2)}
\newcommand\ensavg[2][\dop]{\ev{#2}_{\mkern-1.5mu{#1}}}
\newcommand\sensavg[2][\dop]{\ev*{#2}_{\mkern-1.5mu{#1}}}



\renewcommand\footnoterule{}
\renewcommand\thefootnote{\textcolor{gray}{\arabic{footnote}}}
% \setbeamertemplate{footnote}{%
%   \scriptsize\noindent\raggedright$^*$\insertfootnotetext\par%
% }

\title{Decay the quantum way}
\subtitle{Save an open system from guessing today}
\author{Alex Striff \\ Advisor: Darrell Schroeter}
\institute{\textit{Reed College \\ Thesis Seminar}}
\date{March 24, 2021}

\begin{document}

{%
  \beamertemplatenavigationsymbolsempty%
  \begin{frame}
    \begin{tikzpicture}[remember picture,overlay,opacity=0.125]
      \node[at=(current page.center)] {%
          \includegraphics[keepaspectratio,
          width=\paperwidth,
          height=\paperheight]{spins}
        };
    \end{tikzpicture}
    \titlepage%
  \end{frame}
}

\section{What decay?}

\begin{frame}{What happens to the spins?}
  \begin{tikzpicture}[remember picture,overlay,opacity=1]
    \node[at=(current page.center)] {%
        \includegraphics[keepaspectratio,
        width=\paperwidth,
        height=\paperheight]{spins}
      };
  \end{tikzpicture}
\end{frame}

\begin{frame}{Example: The transverse-field Ising model}
  \begin{center}
    \alt<2->{\includegraphics[width=\textwidth]{open}}{%
    \includegraphics[width=\textwidth]{closed}}
  \end{center}
\end{frame}

\section{What is an open quantum system?}

\begin{frame}{The quantum state}
  A \alert{quantum state} assigns probabilities to outcomes.
  \begin{center}
    \includegraphics[width=0.3333\textwidth]{stern-gerlach}
  \end{center}
  $\ket{\psi} \in \hilb$ does not assign even odds for all orientations.
  \frownie%
  
  How to represent half-$\ket{\uparrow}$ and half-$\ket{\downarrow}$?
  \begin{align*}
    \dop
    &= \frac{1}{2}\op{\uparrow} + \frac{1}{2}\op{\downarrow} \\
    &= \sum_i p_i \op{\psi_i}
  \end{align*}
  A \alert{density operator} $\dop$ represents a general quantum state.
\end{frame}

\begin{frame}{Open system with weak coupling to a thermal bath}
  \begin{center}
    $
    \vcenter{\hbox{\includegraphics[width=0.3333\textwidth]{spins}}}
    \overset{\ham_I}{\longleftrightarrow}
    \quad
    \vcenter{\hbox{\includegraphics[width=0.3333\textwidth]{thermal-bath}}}
    $
  \end{center}
  \begin{itemize}
    \item Start with $\dop(0) = \dop_S(0) \tp \dop_B$.

    \item The combined system has unitary evolution from the Hamiltonian
      \[
        \ham
        = \ham_S + \ham_B + \ham_I.
      \]

    \item Then throw away the bath: $\dop_S(t) = \tr_B\dop(t)$.

    \item How to find $\dop_S(t)$ \alert{directly} from $\dop_S(0)$?
  \end{itemize}
\end{frame}

\begin{frame}{Time evolution}
  Whatever it is, the result must also be a density operator.
  \[
    \ket{\Psi(t)} = \opr{U}(t) \ket{\Psi(0)}
    \quad\mapsto\quad
    \dop(t)
    = \sopr{V}(t)\dop(0)
  \]
  \pause%
  Usually $\dop(t + \dd{t})$ depends only on $\dop(t)$, not the past.
  \[
    \sopr{V}(t + \dd{t})\dop(0)
    = \sopr{V}(\dd{t})\sopr{V}(t)\dop(0)
  \]
  Then there is a generator of time evolution.
  \[
    \ket{\dot{\Psi}} = -\im\ham\ket{\Psi}
    \quad\mapsto\quad
    \dot{\dop} = \sopr{L}\dop
  \]
  \pause%
  Any \textcolor{gray}{\textsc{cptp}} $\sopr{L}$ may be put into
  \alert{Lindblad form}.
  \[
    \sopr{L}\dop
    = -\im\comm{\ham}{\dop}
    + \sum_i \gamma_i \qty(%
    \opr{J}_i\dop\opr{J}_i^\dag
    - \frac{1}{2}\acomm{\opr{J}_i^\dag\opr{J}_i}{\dop}
    )
  \]
\end{frame}

\begin{frame}{Example: Atomic emission}
  The eigenstates of hydrogen are \alert{not} stable.
  \textcolor{gray}{(E\&M vacuum)}
  \[
    \vcenter{\hbox{\includegraphics[width=0.5\textwidth]{hydrogen}}}
    \quad
    \vcenter{\hbox{\includegraphics[width=0.125\textwidth]{two-level.png}}}
    \quad
    \ham = \frac{\omega}{2}\,\pauli_z
  \]
  \begin{center}
    \includegraphics[width=0.75\textwidth]{populations}
  \end{center}
\end{frame}

\begin{frame}{Example: Two spins in a transverse magnetic field}
  \begin{center}
    \includegraphics[width=0.3333\textwidth]{spins}
  \end{center}
  \begin{itemize}
    \item Multiple spins are typically assumed to decay independently.
    \item But there are non-local effects from interactions between spins.
  \end{itemize}
  The simplest case is two spins.
  \[
    \ham
    = \alpha \pauli_x^1 \pauli_x^2 + \qty(\pauli_z^1 + \pauli_z^2)
  \]
  Find dissipation rates by diagonalizing $\sopr{L}$.
\end{frame}

\begin{frame}{How to compare different systems?}
  \begin{center}
    \includegraphics[width=0.9\textwidth]{relative}
  \end{center}
\end{frame}

\begin{frame}{What happens to the spins?}
  \vspace{-\baselineskip}
  \[
    \vcenter{\hbox{\includegraphics[width=0.3\textwidth]{spins}}}
    \quad
    \vcenter{\hbox{\includegraphics[width=0.6\textwidth]{open}}}
  \]
  \begin{itemize}
    \item Interaction with the bath causes decay to equilibrium.
    \item I wrote code that finds the equation for decay.
    \item Contrary to practice, spin interactions change the decay.
    \item Next: continue to quantify this effect.
  \end{itemize}
\end{frame}

\section*{References}

\begin{frame}{References}
  \nocite{opensys,liebTwoSolubleModels1961,pfeutyOnedimensionalIsingModel1970}
  \printbibliography%
\end{frame}

\end{document}

